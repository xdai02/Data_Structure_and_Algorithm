\chapter{动态规划}

\section{动态规划}

\subsection{动态规划(Dynamic Programming)}

动态规划在数学上属于运筹学的分支,是求解决策过程最优化的数学方法,同时也是计算机科学与技术领域中一种常见的算法思想。\\

动态规划算法的基本思想与分治法类似,也是将带求解的问题分解为若干个子问题,按顺序求解子问题。前一子问题的解,为后一子问题的求解提供了有用的信息。\\

在求解任一子问题时,列出各种可能的局部解,通过决策保留那些有可能达到最优的局部解,丢弃其它局部解。依次解决各子问题,最后一个子问题就是初始问题的解。\\

动态规划的本质是对问题状态的定义和状态转移方程的定义。动态规划通过拆分问题,定义问题状态和状态之间的关系,使得问题能够以递推的方式去解决。因此在一个典型的动态规划问题上,需要定义问题状态以及写出状态转移方程,这样对于问题的解答就会一目了然。\\

\subsection{爬楼梯}

有一座高度是10级台阶的楼梯,从下往上走,每跨一步只能向上1级或者2级台阶,要求求出一共有多少种走法。\\

比如,每次走1级台阶,一共走10步,这是其中一种走法,可以简写成[1, 1, 1, 1, 1, 1, 1, 1, 1, 1]。再比如,每次走2级台阶,一共走5步,这是另一种走法,可以简写成[2, 2, 2, 2, 2]。当然,除此之外,还有很多很多种走法。\\

暴力枚举的算法利用排列组合的思想,通过多重循环遍历出所有的可能性。但是暴力枚举的时间复杂度是指数级的,有没有更高效的解法呢?\\

要不找个楼梯走一下试试吧!正好能减肥!\\

动态规划是一种分阶段求解决策问题的数学思想,它不止用于编程领域,也应用于管理学、经济学、生物学等。总的来说就是大事化小,小事化了。\\

在爬楼梯问题中,假设你只差最后一步就走到第10级台阶,这时候会出现几种情况?\\

当然是两种喽,因为每一步只许走1级或2级,所以最后一步要么是从第9级走到第10级,要么是从第8级走到第10级。\\

接下来就引申出了一个新的问题,如果已知从第0级走到第9级的走法有X种,从第0级走到第8级的走法有Y种,那么从第0级走到第10级的走法就有X + Y种。\\

为了方便表达,我们把10级台阶的走法数量简写为F(10),此时F(10) = F(9) + F(8)。那么如何计算F(9)和F(8)呢?\\

利用刚才的思路可以很容易地推断出F(9) = F(8) + F(7), F(8) = F(7) + F(6)。这样我们就把一个复杂的问题分阶段进行简化,逐步简化成简单的问题,这就是动态规划的思想。\\

当只有1级台阶和2级台阶的时候,显然分别只有1种和2种走法。由此可以归纳出递推公式:

\vspace{-0.5cm}

\begin{align*}
	F(n) = \begin{cases}
		1               & n = 1   \\
		1               & n = 2   \\
		F(n-1) + F(n-2) & n \ge 3
	\end{cases}
\end{align*}

动态规划当中包含了三个重要的概念:

\begin{itemize}
	\item 最优子结构
	\item 边界
	\item 状态转移方程
\end{itemize}

刚才分析得出的F(10) = F(9) + F(8)中,F(9)和F(8)就是F(10)的最优子结构。\\

当只有1级台阶或2级台阶时,我们可以直接得出结果,无需继续简化。因此F(1)和F(2)就是问题的边界。如果一个问题没有边界,将永远无法得到有限的结果。\\

F(n) = F(n-1) + F(n-2)是阶段与阶段之间的状态转移方程,这是动态规划的核心,决定了问题的每一个阶段与下一阶段的关系。\\

既然已经归纳出了F(n) = F(n-1) + F(n-2),又知道了边界,那就可以直接用递归的思路实现。\\

\mybox{爬楼梯(递归)}

\begin{lstlisting}[language=C]
int climbStairs(int n) {
    if(n <= 0) {
        return 0;
    } else if(n == 1) {
        return 1;
    } else if(n == 2) {
        return 2;
    }
    return climbStairs(n-1) + climbStairs(n-2);
}
\end{lstlisting}

递归的确可以计算出最终答案,可是其时间复杂度却是指数级的。

\begin{figure}[H]
	\centering
	\begin{tikzpicture}[
			level distance=2.5cm,
			level 1/.style={sibling distance=8cm},
			level 2/.style={sibling distance=4cm},
			level 3/.style={sibling distance=2cm}
		]
		\node {F(n)}
		child {
				node {F(n-1)}
				child {
						node {F(n-2)}
						child {node {F(n-3)}}
						child {node {F(n-4)}}
					}
				child {
						node {F(n-3)}
						child {node {F(n-4)}}
						child {node {F(n-5)}}
					}
			}
		child {
				node {F(n-2)}
				child {
						node {F(n-3)}
						child {node {F(n-4)}}
						child {node {F(n-5)}}
					}
				child {
						node {F(n-4)}
						child {node {F(n-5)}}
						child {node {F(n-6)}}
					}
			};
	\end{tikzpicture}
	\caption{递归树}
\end{figure}

二叉树的结点个数就是递归方法所需要计算的次数,因此递归方法的时间复杂度为$ O(2^n) $。\\

可以发现在递归树中,有些相同的参数被重复计算了,越往下走,重复的越多。\\

可是一定要对F(n)自顶向下做递归运算吗?采用自底向上,用迭代的方法也可以推导出结果。\\

之前已经得出结论,F(1) = 1,F(2) = 2:

\begin{table}[H]
	\centering
	\setlength{\tabcolsep}{5mm}{
		\begin{tabular}{|c|c|c|c|c|c|c|}
			\hline
			\textbf{台阶数} & \textbf{1} & \textbf{2} & \textbf{3} & \textbf{4} & \textbf{5} & \textbf{6} \\
			\hline
			\textbf{走法数} & 1          & 2          &            &            &            &            \\
			\hline
		\end{tabular}
	}
\end{table}

因为F(3) = F(1) + F(2),经过一次迭代可以计算出F(3):

\begin{table}[H]
	\centering
	\setlength{\tabcolsep}{5mm}{
		\begin{tabular}{|c|c|c|c|c|c|c|}
			\hline
			\textbf{台阶数} & \textbf{1} & \textbf{2} & \textbf{3} & \textbf{4} & \textbf{5} & \textbf{6} \\
			\hline
			\textbf{走法数} & 1          & 2          & 3          &            &            &            \\
			\hline
		\end{tabular}
	}
\end{table}

第二次迭代,由于F(4)只依赖于F(2)和F(3),而F(3)已经在上一次迭代计算得出,无需重复计算:

\begin{table}[H]
	\centering
	\setlength{\tabcolsep}{5mm}{
		\begin{tabular}{|c|c|c|c|c|c|c|}
			\hline
			\textbf{台阶数} & \textbf{1} & \textbf{2} & \textbf{3} & \textbf{4} & \textbf{5} & \textbf{6} \\
			\hline
			\textbf{走法数} & 1          & 2          & 3          & 5          &            &            \\
			\hline
		\end{tabular}
	}
\end{table}

\mybox{爬楼梯(动态规划)}

\begin{lstlisting}[language=C]
int climbStairs(int n) {
    if(n <= 0) {
        return 0;
    } else if(n == 1) {
        return 1;
    } else if(n == 2) {
        return 2;
    }
    int num1 = 1;
    int num2 = 2;
    int sum;
    for(int i = 3; i <= n; i++) {
        sum = num1 + num2;
        num1 = num2;
        num2 = sum;
    }
    return sum;
}
\end{lstlisting}

\newpage

\section{硬币找零}

\subsection{硬币找零}

有三种硬币,面值分别是2元、5元、7元,每种硬币都有足够多。买一个物品需要27元,如何用最少的硬币组合正好付清?\\

要让硬币最少,应该尽量用面值大的硬币,也就是贪心算法的结果为7 + 7 + 7 + 5 = 26,呃……\\

那么需要改变一下策略,尽量用面值大的硬币,最后如果可以用一种硬币付清就行。7 + 7 + 7 + 2 + 2 + 2 = 27,一共6枚硬币,应该对了吧……\\

但是正确答案是7 + 5 + 5 + 5 + 5 = 27,一共5枚硬币。\\

状态在动态规划中的作用属于定海神针,确定状态需要两个意识:最后一步和子问题。\\

虽然目前不知道最优策略是什么,但是最优策略肯定是k枚硬币$ a_1, a_2, \dots, a_k $面值加起来是27,所以一定存在最后一枚硬币$ a_k $。除了这枚硬币,前面硬币的面值加起来是$ 27 - a_k $。因为是最优策略,所以拼出$ 27 - a_k $的硬币数一定要最少。\\

\begin{figure}[H]
	\centering
	\begin{tikzpicture}
		\draw[fill=cyan] (0,0) rectangle node{$ 27 - a_k $} (7,1);
		\draw[fill=yellow] (7,0) rectangle node{$ a_k $} (10,1);
		\draw (3.5,-0.5) node{k-1枚硬币};
		\draw (8.5,-0.5) node{最后一枚硬币};
	\end{tikzpicture}
	\caption{最后一步}
\end{figure}

因此问题就变成了最少用多少枚硬币可以拼出$ 27 - a_k $。这样就将原问题转化成了一个子问题,而且规模更小。\\

假设状态F(x)表示拼出x元的所需的最少硬币数,最后那枚硬币$ a_k $只可能是2元、5元或7元:

\begin{itemize}
	\item 如果$ a_k $是2元,F(27) = F(27-2) + 1(加上最后一枚2元硬币)。

	\item 如果$ a_k $是5元,F(27) = F(27-5) + 1(加上最后一枚5元硬币)。

	\item 如果$ a_k $是7元,F(27) = F(27-7) + 1(加上最后一枚7元硬币)。
\end{itemize}

由此可得到递归公式:

\vspace{-0.5cm}

$$
	F(27) = min\{F(27-2)+1,\ F(27-5)+1,\ F(27-7)+1\}
$$

\vspace{0.5cm}

\mybox{硬币找零(递归)}

\begin{lstlisting}[language=Java]
public static int getMinCoins(int price) {
    // 0元钱只需要0枚硬币
    if(price == 0) {
        return 0;
    }
    // 初始化为无穷大
    int coinNum = Integer.MAX_VALUE - 1;
    // 最后一枚硬币是2元
    if(price >= 2) {
        coinNum = Math.min(getMinCoins(price-2) + 1, coinNum);
    }
    // 最后一枚硬币是5元
    if(price >= 5) {
        coinNum = Math.min(getMinCoins(price-5) + 1, coinNum);
    }
    // 最后一枚硬币是7元
    if(price >= 7) {
        coinNum = Math.min(getMinCoins(price-7) + 1, coinNum);
    }
    return coinNum;
}
\end{lstlisting}

\begin{figure}[H]
	\centering
	\begin{tikzpicture}[
			level distance=2.5cm,
			level 1/.style={sibling distance=5cm},
			level 2/.style={sibling distance=1.5cm},
			level 3/.style={sibling distance=1cm}
		]
		\node {F(27)}
		child {
				node {F(25)}
				child {
						node {F(23)}
						child {node { }}
						child {node { }}
					}
				child {
						node {F(20)}
						child {node { }}
						child {node { }}
					}
				child {
						node {F(18)}
						child {node { }}
						child {node { }}
					}
			}
		child {
				node {F(22)}
				child {
						node {F(20)}
						child {node { }}
						child {node { }}
					}
				child {
						node {F(17)}
						child {node { }}
						child {node { }}
					}
				child {
						node {F(15)}
						child {node { }}
						child {node { }}
					}
			}
		child {
				node {F(20)}
				child {
						node {F(18)}
						child {node { }}
						child {node { }}
					}
				child {
						node {F(15)}
						child {node { }}
						child {node { }}
					}
				child {
						node {F(13)}
						child {node { }}
						child {node { }}
					}
			};
	\end{tikzpicture}
	\caption{递归树}
\end{figure}

动态规划的另一个组成部分就是状态转移方程:

\vspace{-0.5cm}

$$
	F[X] = min\{F[X-2]+1,\ F[X-5]+1,\ F[X-7]+1\}
$$

其次需要考虑初始条件和边界情况。如果不能拼出i元,则定义$ F[i] = \infty $。例如当$ x - 2 $、$ x - 5 $、$ x - 7 $小于0时:

\vspace{-0.5cm}

$$
	F[-1] = F[-2] = \dots = \infty
$$

状态转移方程的初始状态为F[0] = 0,因为无需任何硬币就能拼成0元。\\

\begin{table}[H]
	\centering
	\setlength{\tabcolsep}{3mm}{
		\begin{tabular}{|c|c|c|c|c|c|c|c|c|c|c|}
			\hline
			\textbf{$ \dots $} & \textbf{F[-1]} & \textbf{F[0]} & \textbf{F[1]} & \textbf{F[2]} & \textbf{F[3]} & \textbf{F[4]} & \textbf{F[5]} & \textbf{F[6]} & \textbf{$ \dots $} & \textbf{F[27]} \\
			\hline
			$ \infty $         & $ \infty $     & 0             &               &               &               &               &               &               &                    &                \\
			\hline
		\end{tabular}
	}
	\caption{初始状态}
\end{table}

\begin{table}[H]
	\centering
	\setlength{\tabcolsep}{3mm}{
		\begin{tabular}{|c|c|c|c|c|c|c|c|c|c|c|}
			\hline
			\textbf{$ \dots $} & \textbf{F[-1]} & \textbf{F[0]} & \textbf{F[1]} & \textbf{F[2]} & \textbf{F[3]} & \textbf{F[4]} & \textbf{F[5]} & \textbf{F[6]} & \textbf{$ \dots $} & \textbf{F[27]} \\
			\hline
			$ \infty $         & $ \infty $     & 0             & $ \infty $    &               &               &               &               &               &                    &                \\
			\hline
		\end{tabular}
	}
	\caption{$ F[1] = min\{F[-1] + 1, F[-4] + 1, F[-6] + 1\} = \infty $}
\end{table}

\begin{table}[H]
	\centering
	\setlength{\tabcolsep}{3mm}{
		\begin{tabular}{|c|c|c|c|c|c|c|c|c|c|c|}
			\hline
			\textbf{$ \dots $} & \textbf{F[-1]} & \textbf{F[0]} & \textbf{F[1]} & \textbf{F[2]} & \textbf{F[3]} & \textbf{F[4]} & \textbf{F[5]} & \textbf{F[6]} & \textbf{$ \dots $} & \textbf{F[27]} \\
			\hline
			$ \infty $         & $ \infty $     & 0             & $ \infty $    & 1             &               &               &               &               &                    &                \\
			\hline
		\end{tabular}
	}
	\caption{$ F[2] = min\{F[0] + 1, F[-3] + 1, F[-5] + 1\} = 1 $}
\end{table}

\begin{table}[H]
	\centering
	\setlength{\tabcolsep}{3mm}{
		\begin{tabular}{|c|c|c|c|c|c|c|c|c|c|c|}
			\hline
			\textbf{$ \dots $} & \textbf{F[-1]} & \textbf{F[0]} & \textbf{F[1]} & \textbf{F[2]} & \textbf{F[3]} & \textbf{F[4]} & \textbf{F[5]} & \textbf{F[6]} & \textbf{$ \dots $} & \textbf{F[27]} \\
			\hline
			$ \infty $         & $ \infty $     & 0             & $ \infty $    & 1             & $ \infty $    &               &               &               &                    &                \\
			\hline
		\end{tabular}
	}
	\caption{$ F[3] = min\{F[1] + 1, F[-2] + 1, F[-4] + 1\} = \infty $}
\end{table}

\begin{table}[H]
	\centering
	\setlength{\tabcolsep}{3mm}{
		\begin{tabular}{|c|c|c|c|c|c|c|c|c|c|c|}
			\hline
			\textbf{$ \dots $} & \textbf{F[-1]} & \textbf{F[0]} & \textbf{F[1]} & \textbf{F[2]} & \textbf{F[3]} & \textbf{F[4]} & \textbf{F[5]} & \textbf{F[6]} & \textbf{$ \dots $} & \textbf{F[27]} \\
			\hline
			$ \infty $         & $ \infty $     & 0             & $ \infty $    & 1             & $ \infty $    & 2             &               &               &                    &                \\
			\hline
		\end{tabular}
	}
	\caption{$ F[4] = min\{F[2] + 1, F[-1] + 1, F[-3] + 1\} = 2 $}
\end{table}

\begin{table}[H]
	\centering
	\setlength{\tabcolsep}{3mm}{
		\begin{tabular}{|c|c|c|c|c|c|c|c|c|c|c|}
			\hline
			\textbf{$ \dots $} & \textbf{F[-1]} & \textbf{F[0]} & \textbf{F[1]} & \textbf{F[2]} & \textbf{F[3]} & \textbf{F[4]} & \textbf{F[5]} & \textbf{F[6]} & \textbf{$ \dots $} & \textbf{F[27]} \\
			\hline
			$ \infty $         & $ \infty $     & 0             & $ \infty $    & 1             & $ \infty $    & 2             & 1             & 3             & $ \dots $          & 5              \\
			\hline
		\end{tabular}
	}
	\caption{$ F[27] = min\{F[25] + 1, F[-22] + 1, F[-20] + 1\} = 5 $}
\end{table}

\mybox{硬币找零(动态规划)}

\begin{lstlisting}[language=Python]
def get_min_coins(coins, price):
    f = [INF] * (price + 1)

    f[0] = 0
    for i in range(1, price+1):
        for j in range(len(coins)):
            if i >= coins[j] and f[i - coins[j]] != INF:
                f[i] = min(f[i - coins[j]] + 1, f[i])
        
    if f[price] == INF:
        f[price] = -1
    return f[price]
\end{lstlisting}

\newpage

\section{路径问题}

\subsection{路径问题}

有一个机器人位于一个m行n列的网格的左上角$ (0, 0) $,机器人每次只能向下或向右移动一步,问有多少种方法可以走到右下角。

\begin{table}[H]
	\centering
	\setlength{\tabcolsep}{3mm}{
		\begin{tabular}{|c|c|c|c|c|c|c|c|c|c|}
			\hline
			           & \textbf{0}         & \textbf{1} & \textbf{2} & \textbf{3} & \textbf{4} & \textbf{5} & \textbf{6} & \textbf{7} \\
			\hline
			\textbf{0} & \textcolor{red}{R} &            &            &            &            &            &            &            \\
			\hline
			\textbf{1} &                    &            &            &            &            &            &            &            \\
			\hline
			\textbf{2} &                    &            &            &            &            &            &            &            \\
			\hline
			\textbf{3} &                    &            &            &            &            &            &            &            \\
			\hline
		\end{tabular}
	}
	\caption{起点}
\end{table}

\begin{table}[H]
	\centering
	\setlength{\tabcolsep}{3mm}{
		\begin{tabular}{|c|c|c|c|c|c|c|c|c|c|}
			\hline
			           & \textbf{0} & \textbf{1} & \textbf{2} & \textbf{3} & \textbf{4} & \textbf{5} & \textbf{6}                       & \textbf{7}                      \\
			\hline
			\textbf{0} &            &            &            &            &            &            &                                  &                                 \\
			\hline
			\textbf{1} &            &            &            &            &            &            &                                  &                                 \\
			\hline
			\textbf{2} &            &            &            &            &            &            &                                  & \textcolor{red}{$ \downarrow $} \\
			\hline
			\textbf{3} &            &            &            &            &            &            & \textcolor{red}{$ \rightarrow $} & \textcolor{red}{R}              \\
			\hline
		\end{tabular}
	}
	\caption{终点}
\end{table}

无论机器人用何种方式到达右下角,总有最后挪动的一步。右下角的坐标为$ (m-1, n-1) $,那么前一步机器人一定在$ (m-2, n-1) $或$ (m-1, n-2) $的位置。\\

如果机器人有X种方式从左上角走到$ (m-2, n-1) $,有Y种方式从左上角走到$ (m-1, n-2) $,那么机器人一共有X + Y种方式从左上角走到$ (m-1, n-1) $。原问题就转换为了机器人有多少种方式从左上角走到$ (m-2, n-1) $和$ (m-1, n-2) $。\\

那么可以得出转移方程f[i][j] = f[i-1][j] + f[i][j-1],其中f[i][j]表示机器人有多少种方式走到$ (i, j) $。\\

初始条件为f[0][0] = 1,因为机器人只有1种方式到达左上角。\\

边界情况为当i = 0或j = 0,则前一步只能有一个方向到达,因此f[i][j] = 1。

\begin{table}[H]
	\centering
	\setlength{\tabcolsep}{3mm}{
		\begin{tabular}{|c|c|c|c|c|c|c|c|c|c|}
			\hline
			           & \textbf{0}                      & \textbf{1}                       & \textbf{2}                       & \textbf{3}                       & \textbf{4}                       & \textbf{5}                       & \textbf{6}                       & \textbf{7}         \\
			\hline
			\textbf{0} &                                 & \textcolor{red}{$ \rightarrow $} & \textcolor{red}{$ \rightarrow $} & \textcolor{red}{$ \rightarrow $} & \textcolor{red}{$ \rightarrow $} & \textcolor{red}{$ \rightarrow $} & \textcolor{red}{$ \rightarrow $} & \textcolor{red}{R} \\
			\hline
			\textbf{1} & \textcolor{red}{$ \downarrow $} &                                  &                                  &                                  &                                  &                                  &                                  &                    \\
			\hline
			\textbf{2} & \textcolor{red}{$ \downarrow $} &                                  &                                  &                                  &                                  &                                  &                                  &                    \\
			\hline
			\textbf{3} & \textcolor{red}{R}              &                                  &                                  &                                  &                                  &                                  &                                  &                    \\
			\hline
		\end{tabular}
	}
	\caption{边界情况}
\end{table}

\mybox{路径问题}

\begin{lstlisting}[language=C]
int uniquePath(int m, int n) {
    int f[m][n];
    for(int i = 0; i < m; i++) {
        for(int j = 0; j < n; j++) {
            if(i == 0 || j == 0) {
                f[i][j] = 1;
            } else {
                f[i][j] = f[i-1][j] + f[i][j-1];
            }
        }
    }
    return f[m-1][n-1];
}
\end{lstlisting}

\newpage

\section{跳跃游戏}

\subsection{跳跃游戏}

有n块石头分别在$ 0, 1, \dots, n-1 $的位置,一只青蛙在石头0,想跳到石头n - 1。如果青蛙在第i块石头上,每块石头的元素表示可以跳跃的最长距离。问青蛙能否跳到石头n - 1。\\

\textbf{示例1}

输入:s = [2, 3, 1, 1, 4]

输出:True

解释:可以先跳1步,从石头0达到石头1,然后再从石头1跳3步到达目标。\\

\textbf{示例2}

输入:s = [3, 2, 1, 0, 4]

输出:False

解释:无论怎样,总会达到石头3,但该石头的最大跳跃长度是0,所以永远不可能到达目标。\\

如果青蛙能跳到最后一块石头n - 1,那么它一定是从石头i跳过来的(i < n - 1)。\\

这需要两个条件同时满足:

\begin{enumerate}
	\item 青蛙可以跳到石头i。
	\item 最后一跳不能超过可以跳跃的最大距离。
\end{enumerate}

那么问题转化为了青蛙能否跳到石头i。假设f[j]表示青蛙能否跳到石头j,可以得出转移方程:

\vspace{-0.5cm}

$$
	f[j] = OR_{0 \le i < j}(f[i]\ AND\ i + a[i] \ge j)
$$

\begin{itemize}
	\item $ f[j] $:青蛙能否跳到石头$ j $。

	\item $ OR_{0 \le i < j} $:枚举上一个跳到的石头$ i $。

	\item $ f[i] $:青蛙能否跳到石头$ i $。

	\item $ a[i] $:最后一步的距离不能超过$ a_i $。
\end{itemize}

初始条件为f[0] = True,因为青蛙一开始就在石头0。\\

\mybox{跳跃游戏}

\begin{lstlisting}[language=Java]
public static boolean canJump(int[] stone) {
    int n = stone.length;
    boolean[] f = new boolean[n];
    f[0] = true;

    for(int j = 1; j < n; j++) {
        f[j] = false;
        for(int i = 0; i < j; i++) {
            if(f[i] && i + stone[i] >= j) {
                f[j] = true;
                break;
            }
        }
    }
    return f[n-1];
}
\end{lstlisting}

\newpage

\section{0-1背包}

\subsection{0-1背包(0-1 Knapsack)}

有一个小偷带了一个能够装C = 20公斤物品的背包到商店里面偷东西,请问他要怎么偷才能使价值最高?

\begin{table}[H]
	\centering
	\setlength{\tabcolsep}{5mm}{
		\begin{tabular}{|c|c|c|}
			\hline
			\textbf{物品} & \textbf{重量W} & \textbf{价格V} \\
			\hline
			\textbf{0}    & 2              & 3              \\
			\hline
			\textbf{1}    & 3              & 4              \\
			\hline
			\textbf{2}    & 4              & 5              \\
			\hline
			\textbf{3}    & 5              & 8              \\
			\hline
			\textbf{4}    & 9              & 10             \\
			\hline
		\end{tabular}
	}
	\caption{物品信息}
\end{table}

假设用B(k, C)表示当背包容量还剩下C的时候,在前k件物品中能偷到的最大价值。

\vspace{-0.5cm}

\begin{align*}
	B(k, C) = \begin{cases}
		B(k-1, C) & \text{当第}k\text{件太重} \\
		max \begin{cases}
			B(k-1, C-w_k) + v_k & \text{偷}   \\
			B(k-1, C)           & \text{不偷} \\
		\end{cases}
	\end{cases}
\end{align*}

\begin{figure}[H]
	\centering
	\begin{tikzpicture}[
			level distance=2.5cm,
			level 1/.style={sibling distance=8cm},
			level 2/.style={sibling distance=4cm},
			level 3/.style={sibling distance=3cm}
		]
		\node {B(4,20)}
		child {
				node {B(3,11)+10}
				child {
						node {B(2,6)+18}
						child {
								node {B(1,2)+23}
								child[missing] {}
								child {
										node {B(0,2)+23}
										child {node {3+23=26}}
										child[missing] {}
									}
							}
						child {node {B(1,6)+18}}
					}
				child {node {B(2,11)+10}}
			}
		child {node {B(3,20)}};

		\draw (-2.2,-1) node{偷};
		\draw (2.2,-1) node{不偷};

		\draw (-5.2,-3.5) node{偷};
		\draw (-2.8,-3.5) node{不偷};

		\draw (-7,-6) node{偷};
		\draw (-4.8,-6) node{不偷};

		\draw (-6.3,-8.5) node{不偷};

		\draw (-7,-11) node{偷};
	\end{tikzpicture}
	\caption{决策树}
\end{figure}

边界情况为B(i, 0) = B(0, j) = 0,即当背包容量为0或者没有物品可偷的情况下,最大价值为0。

\begin{table}[H]
	\centering
	\setlength{\tabcolsep}{1.3mm}{
		\begin{tabular}{|c|c|c|c|c|c|c|c|c|c|c|c|c|c|c|c|c|c|c|c|c|c|}
			\hline
			\textbf{Capacity} & \textbf{0} & \textbf{1} & \textbf{2} & \textbf{3} & \textbf{4} & \textbf{5} & \textbf{6} & \textbf{7} & \textbf{8} & \textbf{9} & \textbf{10} & \textbf{11} & \textbf{12} & \textbf{13} & \textbf{14} & \textbf{15} & \textbf{16} & \textbf{17} & \textbf{18} & \textbf{19} & \textbf{20} \\
			\hline
			\textbf{No Item}  & 0          & 0          & 0          & 0          & 0          & 0          & 0          & 0          & 0          & 0          & 0           & 0           & 0           & 0           & 0           & 0           & 0           & 0           & 0           & 0           & 0           \\
			\hline
			\textbf{Item 0}   & 0          & 0          & 3          & 3          & 3          & 3          & 3          & 3          & 3          & 3          & 3           & 3           & 3           & 3           & 3           & 3           & 3           & 3           & 3           & 3           & 3           \\
			\hline
			\textbf{Item 1}   & 0          & 0          & 3          & 4          & 4          & 7          & 7          & 7          & 7          & 7          & 7           & 7           & 7           & 7           & 7           & 7           & 7           & 7           & 7           & 7           & 7           \\
			\hline
			\textbf{Item 2}   & 0          & 0          & 3          & 4          & 5          & 7          & 8          & 9          & 9          & 12         & 12          & 12          & 12          & 12          & 12          & 12          & 12          & 12          & 12          & 12          & 12          \\
			\hline
			\textbf{Item 3}   & 0          & 0          & 3          & 4          & 5          & 8          & 8          & 11         & 12         & 13         & 15          & 16          & 17          & 17          & 20          & 20          & 20          & 20          & 20          & 20          & 20          \\
			\hline
			\textbf{Item 4}   & 0          & 0          & 3          & 4          & 5          & 8          & 8          & 11         & 12         & 13         & 15          & 16          & 17          & 17          & 20          & 20          & 21          & 22          & 23          & 25          & 26          \\
			\hline
		\end{tabular}
		\caption{动态规划表格法}
	}
\end{table}

\mybox{0-1背包}

\begin{lstlisting}[language=C]
#define ITEM_NUM 5
#define CAPACITY 20

int getMaxValue(int *weight, int *value) {
    int b[ITEM_NUM+1][CAPACITY+1] = {{0}};

    for(int k = 1; k <= ITEM_NUM; k++) {
        for(int c = 1; c <= CAPACITY; c++) {
            if(weight[k] > c) {
                b[k][c] = b[k-1][c];
            } else {
                b[k][c] = max(
                b[k-1][c-weight[k]] + value[k],
                b[k-1][c]
                );
            }
        }
    }
    return b[ITEM_NUM][CAPACITY];
}
\end{lstlisting}

\newpage