\chapter{数组}

\section{查找算法}

\subsection{顺序查找(Sequence Search)}

顺序查找也称线性查找,是一种按照序列原有顺序进行遍历比较的查找算法。\\

对于任意一个序列以及一个需要查找的元素(关键字),将关键字与序列中元素依次比较,直到找出与给定关键字相同的元素,或者将序列中的元素与其都比较完为止。若某个元素的值与关键字相等,则查找成功;如果直到最后一个元素,元素的值和关键字比较都不等时,则查找不成功。\\

最好的情况就是在第一个位置就找到,算法时间复杂度为$ O(1) $。最坏情况是关键字不存在,需要进行$ n $次比较,时间复杂度为$ O(n) $。\\

\mybox{顺序查找}

\begin{lstlisting}[language=C++]
template <typename T>
int sequence_search(T *arr, int n, T key) {
    for (int i = 0; i < n; i++) {
        if (arr[i] == key) {
            return i;
        }
    }
    return -1;
}
\end{lstlisting}

\vspace{0.5cm}

\subsection{二分查找(Binary Search)}

二分查找法也称折半查找,是一种效率较高的查找方法。折半查找要求线性表必须采用顺序存储结构,而且表中元素按关键字有序排列。\\

算法思想是假设表中元素是按升序排列,将表中间位置的关键字与查找关键字比较,如果两者相等,则查找成功;否则利用中间位置记录将表分成前、后两个子表,如果中间位置的关键字大于查找关键字,则进一步查找前一子表,否则进一步查找后一子表。重复以上过程,直到找到满足条件的记录,使查找成功,或直到子表不存在为止,此时查找不成功。\\

二分查找法的时间复杂度为$ O(logn) $。\\

\mybox{二分查找}

\begin{lstlisting}[language=C++]
template <typename T>
int binary_search(T *arr, int n, T key) {
    int start = 0;
    int end = n - 1;

    while (start <= end) {
        int mid = (start + end) / 2;
        if (arr[mid] == key) {
            return mid;
        } else if (arr[mid] < key) {
            start = mid + 1;
        } else {
            end = mid - 1;
        }
    }
    return -1;
}
\end{lstlisting}

\newpage

\section{数组}

\subsection{数组(Array)}

数组是数据结构中最简单的结构,很多编程语言都内置数组。数组是有限个相同类型的变量所组成的集合,数组中的每一个变量被称为元素。\\

创建数组时会在内存中划分出一块连续的内存,将数据按顺序进行存储,数组中的每一个元素都有着自己的下标(index),下标从0开始。\\

对于数组来说,读取元素是最简单的操作。由于数组在内存中顺序存储,所以只要给出数组的下标,就可以读取到对应位置的元素。像这种根据下标读取元素的方式叫作随机读取。数组读取元素的时间复杂度是$ O(1) $。\\

数组的劣势体现在插入和删除元素方面。由于数组元素连续紧密地存储在内存中,插入、删除元素都会导致大量元素被迫移动,影响效率。总的来说,数组所适合的是读操作多、写操作少的场景。\\

\subsection{插入元素}

在数组中插入元素存在3种情况:

\begin{figure}[H]
    \centering
    \begin{tikzpicture}[font=\ttfamily,
            array/.style={matrix of nodes,nodes={draw, minimum size=10mm, fill=green!30},column sep=-\pgflinewidth, row sep=0.5mm, nodes in empty cells,
                    row 1/.style={nodes={draw=none, fill=none, minimum size=5mm}},
                }]

        \matrix[array] (array) {
            0    & 1    & 2    & 3    & 4    & 5    & 6    & 7    \\
            data & data & data & data & data & data & data & data \\
        };
    \end{tikzpicture}
    \caption{数组}
\end{figure}

\subsubsection{尾部插入}

直接把插入的元素放在数组尾部的空闲位置即可。

\subsubsection{中间插入}

首先把插入位置及后面的元素向后移动,腾出位置,再把要插入的元素放入该位置上。\\

\subsubsection{扩容}

数组的长度在创建时就已经确定了,要实现数组的扩容,只能创建一个新数组,长度是旧数组的2倍,再把旧数组中的元素全部复制过去。\\

数组插入元素最好情况是尾部插入,无需移动任何元素,时间复杂度为$ O(1) $。最坏情况是在第一个位置插入,这样就需要移动后面所有$ n - 1 $个元素,时间复杂度为$ O(n) $。\\

\mybox{插入元素}

\begin{lstlisting}[language=Java]
public void add(T elem) {
    if (size == capacity) {
        resize();
    }
    data[size++] = elem;
}

public void add(int index, T elem) throws IndexOutOfBoundsException {
    if (index < 0 || index > size) {
        throw new IndexOutOfBoundsException("Index out of bounds");
    }

    if (size == capacity) {
        resize();
    }

    for (int i = size - 1; i >= index; i--) {
        data[i + 1] = data[i];
    }

    data[index] = elem;
    size++;
}
\end{lstlisting}

\subsection{删除元素}

数组的删除操作与插入操作过程相反,如果被删除的元素位于数组中间,其后的元素都需要向前挪动一位。\\

\mybox{删除元素}

\begin{lstlisting}[language=Java]
public T remove(int index) throws IndexOutOfBoundsException {
    if (index < 0 || index >= size) {
        throw new IndexOutOfBoundsException("Index out of bounds");
    }

    T elem = data[index];
    for (int i = index; i < size - 1; i++) {
        data[i] = data[i + 1];
    }
    size--;
    return elem;
}
\end{lstlisting}

数组的删除操作,由于涉及元素的移动,时间复杂度为$ O(n) $。\\

\newpage