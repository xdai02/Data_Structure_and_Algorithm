\chapter{链表}

\section{链表种类}

\subsection{单向链表(Singly Linked List)}

为避免元素的移动,采用线性表的另一种存储方式:链式存储结构。链表是一种在物理上非连续、非顺序的数据结构,由若干结点(node)所组成。\\

单向链表的每一个结点又包含两部分,一部分是存放数据的数据域data,另一部分是指向下一个结点的指针域next。结点可以在运行时动态生成。

\vspace{-0.5cm}

\begin{lstlisting}[language=Python]
class Node:
    def __init__(self, elem):
        self.data = elem
        self.next = None
\end{lstlisting}

链表的第一个结点被称为头结点,最后一个节点被称为尾结点,尾结点的next指针指向NULL。\\

\begin{figure}[H]
	\centering
	\begin{tikzpicture}[list/.style={rectangle split, rectangle split parts=2,
					draw, rectangle split horizontal}, >=stealth, start chain]
		\node[list,on chain] (A) {data};
		\node[list,on chain] (B) {data};
		\node[list,on chain] (C) {data};
		\node[on chain,draw,inner sep=6pt] (D) {};
		\draw (D.north east) -- (D.south west);
		\draw (D.north west) -- (D.south east);
		\draw[*->] let \p1 = (A.two), \p2 = (A.center) in (\x1,\y2) -- (B);
		\draw[*->] let \p1 = (B.two), \p2 = (B.center) in (\x1,\y2) -- (C);
		\draw[*->] let \p1 = (C.two), \p2 = (C.center) in (\x1,\y2) -- (D);
	\end{tikzpicture}
	\caption{单向链表}
\end{figure}

与数组按照下标来随机寻找元素不同,对于链表的其中一个结点A,只能根据结点A的next指针来找到该结点的下一个结点B,再根据结点B的next指针找到下一个节点C……\\

数组在内存中的存储方式是顺序存储,链表在内存中的存储方式则是随机存储。链表采用了见缝插针的方式,每一个结点分布在内存的不同位置,依靠next指针关联起来。这样可以灵活有效地利用零散的碎片空间。\\

\subsection{双向链表(Doubly Linked List)}

那么,通过链表的一个结点,如何能快速找到它的前一个结点呢?要想让每个结点都能回溯到它的前置结点,可以使用双向链表。\\

双向链表比单向链表稍微复杂一点,它的每一个结点除了拥有data和next指针,还拥有指向前置结点的prev指针。

\vspace{-0.5cm}

\begin{lstlisting}[language=Python]
class Node:
    def __init__(self, elem):
        self.data = elem
        self.prev = None
        self.next = None
\end{lstlisting}

\begin{figure}[H]
	\centering
	\begin{tikzpicture}[list/.style={rectangle split, rectangle split parts=3,
					draw, rectangle split horizontal}, >=stealth, start chain]
		\node[on chain,draw,inner sep=6pt] (NULL1) {};
		\node[list,on chain] (A) {prev \nodepart{second} data \nodepart{third} next};
		\node[list,on chain] (B) {prev \nodepart{second} data \nodepart{third} next};
		\node[list,on chain] (C) {prev \nodepart{second} data \nodepart{third} next};
		\node[on chain,draw,inner sep=6pt] (NULL2) {};
		\draw (NULL1.north east) -- (NULL1.south west);
		\draw (NULL1.north west) -- (NULL1.south east);
		\draw (NULL2.north east) -- (NULL2.south west);
		\draw (NULL2.north west) -- (NULL2.south east);
		\draw[<-] let \p1 = (A.west), \p2 = (A.center) in (NULL1) -- (\x1,\y2);
		\draw[<->] let \p1 = (A.east), \p2 = (A.center) in (\x1,\y2) -- (B);
		\draw[<->] let \p1 = (B.east), \p2 = (B.center) in (\x1,\y2) -- (C);
		\draw[->] let \p1 = (C.east), \p2 = (C.center) in (\x1,\y2) -- (NULL2);
	\end{tikzpicture}
	\caption{双向链表}
\end{figure}

单向链表只能从头到尾遍历,只能找到后继,无法找到前驱,因此遍历的时候不会死循环。而双向链表需要多分配一个指针的存储空间,每个结点有两个指针,分别指向直接前驱和直接后继。\\

\subsection{循环链表(Circular Linked List)}

除了单向链表和双向链表以外,还有循环链表。对于单向循环链表,尾结点的next指针指向头结点。对于双向循环链表,尾结点的next指针指向头结点,并且头结点的prev指针指向尾结点。\\

\begin{figure}[H]
	\centering
	\begin{tikzpicture}[
			list/.style={
					rectangle split,
					rectangle split parts=2,
					draw,
					rectangle split horizontal
				},
			dotarrow/.style={Circle-Stealth},
			start chain
		]
		\node[list,on chain] (A) {data};
		\node[list,on chain] (B) {data};
		\node[list,on chain] (C) {data};
		\draw[dotarrow] (A.two |- A.center) -- (B);
		\draw[dotarrow] (B.two |- B.center)  -- (C);
		\draw[dotarrow] (C.two |- C.center) to[out=-20,in=190,distance=4cm] (A);
	\end{tikzpicture}
	\caption{单向循环链表}
\end{figure}

\begin{figure}[H]
	\centering
	\begin{tikzpicture}[list/.style={
					rectangle split,
					rectangle split parts=3,
					draw,
					rectangle split horizontal
				},
			dotarrow/.style={Circle-Stealth},
			start chain]
		\node[list,on chain] (A) {prev \nodepart{second} data \nodepart{third} next};
		\node[list,on chain] (B) {prev \nodepart{second} data \nodepart{third} next};
		\node[list,on chain] (C) {prev \nodepart{second} data \nodepart{third} next};

		\draw[dotarrow] (A.center -| A.west) to[out=-220,in=20,distance=2cm] (C.north);
		\draw[<->] let \p1 = (A.east), \p2 = (A.center) in (\x1,\y2) -- (B);
		\draw[<->] let \p1 = (B.east), \p2 = (B.center) in (\x1,\y2) -- (C);
		\draw[dotarrow] (C.east |- C.center) to[out=-20,in=220,distance=2cm] (A.south);
	\end{tikzpicture}
	\caption{双向循环链表}
\end{figure}

\newpage

\section{链表}

\subsection{查找结点}

在查找元素时,链表不像数组那样可以通过下标快速进行定位,只能从头结点开始向后一个一个结点逐一查找。\\

链表中的数据只能按顺序进行访问,最坏的时间复杂度是$ O(n) $。\\

\mybox{查找结点}

\begin{lstlisting}[language=Java]
public T get(int index) throws IndexOutOfBoundsException {
	if (index < 0 || index >= size) {
		throw new IndexOutOfBoundsException("Index out of bounds");
	}

	Node<T> current = head;
	for (int i = 0; i < index; i++) {
		current = current.next;
	}
	return current.data;
}
\end{lstlisting}

\vspace{0.5cm}

\subsection{更新结点}

如果不考虑查找结点的过程,链表的更新过程会像数组那样简单,直接把旧数据替换成新数据即可。\\

\mybox{更新结点}

\begin{lstlisting}[language=Java]
public void set(int index, T elem) throws IndexOutOfBoundsException {
	if (index < 0 || index >= size) {
		throw new IndexOutOfBoundsException("Index out of bounds");
	}

	Node<T> current = head;
	for (int i = 0; i < index; i++) {
		current = current.next;
	}
	current.data = elem;
}
\end{lstlisting}

\vspace{0.5cm}

\subsection{插入结点}

链表插入结点,分为3种情况:

\subsubsection{尾部插入}

把最后一个结点的next指针指向新插入的结点。

\begin{figure}[H]
	\centering
	\begin{tikzpicture}[node distance=2cm, auto, scale=1,
			every node/.style={scale=1}]
		\node[head, label=below:head] (head) {};
		\node[data, right of=head] (A) {\data};
		\node[above of=A,node distance=0.5cm,label=above:a1] (a1){};
		\node[data, right of=A] (B) {\data};
		\node[above of=B,node distance=0.5cm,label=above:a2] (a2){};
		\node[data, right of=B] (C) {\data};
		\node[above of=C,node distance=0.5cm,label=above:a3] (a3){};
		\node[data, right of=C, xshift=0.1cm] (D) {\data};%, xshift=1.5cm
		\node[above of=D,node distance=0.5cm,label=above:a4] (a4){};
		\node[data, right of=D] (E) {\data};
		\node[above of=E,node distance=0.5cm,label=above:a5] (a5){};
		\node[data, right of=E] (last) {data \nodepart{second} \texttt{NULL}};
		\node[above of=last,node distance=0.5cm,label=above:new] (new){};

		\draw[fill] (head.center) circle (0.05);

		\path[ptr] (head.center) --++(right:7.5mm) |- (A.text west);
		\draw[fill] ($(A.south)!0.5!(A.text split)$) circle (0.05);
		\draw[ptr] ($(A.south)!0.5!(A.text split)$) --++(right:10mm) |- (B.text west);
		\draw[fill] ($(B.south)!0.5!(B.text split)$) circle (0.05);
		\draw[ptr] ($(B.south)!0.5!(B.text split)$) --++(right:10mm) |- (C.text west);
		\draw[fill] ($(C.south)!0.5!(C.text split)$) circle (0.05);
		\draw[ptr] ($(C.south)!0.5!(C.text split)$) --++(right:10mm) |- (D.text west);
		\draw[fill] ($(D.south)!0.5!(D.text split)$) circle (0.05);

		\draw[fill] ($(D.south)!0.5!(D.text split)$) circle (0.05);
		\draw[ptr] ($(D.south)!0.5!(D.text split)$) --++(right:10mm) |- (E.text west);
		\draw[fill] ($(E.south)!0.5!(E.text split)$) circle (0.05);
		\draw[ptr] ($(E.south)!0.5!(E.text split)$) --++(right:10mm) |- (last.text west);
	\end{tikzpicture}
	\caption{尾部插入}
\end{figure}

\subsubsection{头部插入}

先把新结点的next指针指向原先的头结点,再把新结点设置为链表的头结点。

\begin{figure}[H]
	\centering
	\begin{tikzpicture}[node distance=2cm, auto, scale=1,
			every node/.style={scale=1}]
		\node[head, label=below:head, xshift=-0.5cm] (head) {};
		\node[data, right of=head] (A) {\data};
		\node[above of=A,node distance=0.5cm,label=above:a1] (a1){};
		\node[data, right of=A] (B) {\data};
		\node[above of=B,node distance=0.5cm,label=above:a2] (a2){};
		\node[data, right of=B] (C) {\data};
		\node[above of=C,node distance=0.5cm,label=above:a3] (a3){};
		\node[data, below of=head, xshift=0.6cm, yshift=0.5cm] (NEW) {\data};
		\node[above of=NEW,node distance=0.3cm,label=above:new] (new){};
		\node[data, right of=C, xshift=0.1cm] (D) {\data};%, xshift=1.5cm
		\node[above of=D,node distance=0.5cm,label=above:a4] (a4){};
		\node[data, right of=D] (E) {\data};
		\node[above of=E,node distance=0.5cm,label=above:a5] (a5){};
		\node[data, right of=E] (last) {data \nodepart{second} \texttt{NULL}};
		\node[above of=last,node distance=0.5cm,label=above:a6] (a6){};

		\draw[fill] (head.center) circle (0.05);

		\path[ptr] (head.center) --++(left:7.5mm) |- (NEW.text west);
		\draw[fill] ($(NEW.south)!0.5!(NEW.text split)$) circle (0.05);
		\draw[ptr] ($(NEW.south)!0.5!(NEW.text split)$) --++(right:6mm) |- (A.text west);
		\draw[fill] ($(A.south)!0.5!(A.text split)$) circle (0.05);
		\draw[ptr] ($(A.south)!0.5!(A.text split)$) --++(right:10mm) |- (B.text west);
		\draw[fill] ($(B.south)!0.5!(B.text split)$) circle (0.05);
		\draw[ptr] ($(B.south)!0.5!(B.text split)$) --++(right:10mm) |- (C.text west);
		\draw[fill] ($(C.south)!0.5!(C.text split)$) circle (0.05);
		\draw[ptr] ($(C.south)!0.5!(C.text split)$) --++(right:10mm) |- (D.text west);
		\draw[fill] ($(D.south)!0.5!(D.text split)$) circle (0.05);
		\draw[fill] ($(D.south)!0.5!(D.text split)$) circle (0.05);
		\draw[ptr] ($(D.south)!0.5!(D.text split)$) --++(right:10mm) |- (E.text west);
		\draw[fill] ($(E.south)!0.5!(E.text split)$) circle (0.05);
		\draw[ptr] ($(E.south)!0.5!(E.text split)$) --++(right:10mm) |- (last.text west);
	\end{tikzpicture}
	\caption{头部插入}
\end{figure}

\subsubsection{中间插入}

先把新结点的next指针指向插入位置的结点,再将插入位置的前置结点的next指针指向新结点。

\begin{figure}[H]
	\centering
	\begin{tikzpicture}[node distance=2cm, auto, scale=1,
			every node/.style={scale=1}]
		\node[head, label=below:head] (head) {};
		\node[data, right of=head] (A) {\data};
		\node[above of=A,node distance=0.5cm,label=above:a1] (a1){};
		\node[data, right of=A] (B) {\data};
		\node[above of=B,node distance=0.5cm,label=above:a2] (a2){};
		\node[data, right of=B] (C) {\data};
		\node[above of=C,node distance=0.5cm,label=above:a3] (a3){};
		\node[data, below of=C, xshift=0.8cm, yshift=0.5cm] (NEW) {\data};
		\node[above of=NEW,node distance=0.3cm,label=above:new] (new){};
		\node[data, right of=C, xshift=0.1cm] (D) {\data};%, xshift=1.5cm
		\node[above of=D,node distance=0.5cm,label=above:a4] (a4){};
		\node[data, right of=D] (E) {\data};
		\node[above of=E,node distance=0.5cm,label=above:a5] (a5){};
		\node[data, right of=E] (last) {data \nodepart{second} \texttt{NULL}};
		\node[above of=last,node distance=0.5cm,label=above:a6] (a6){};

		\draw[fill] (head.center) circle (0.05);

		\path[ptr] (head.center) --++(right:7.5mm) |- (A.text west);
		\draw[fill] ($(A.south)!0.5!(A.text split)$) circle (0.05);
		\draw[ptr] ($(A.south)!0.5!(A.text split)$) --++(right:10mm) |- (B.text west);
		\draw[fill] ($(B.south)!0.5!(B.text split)$) circle (0.05);
		\draw[ptr] ($(B.south)!0.5!(B.text split)$) --++(right:10mm) |- (C.text west);
		\draw[fill] ($(C.south)!0.5!(C.text split)$) circle (0.05);

		\draw[ptr] ($(C.south)!0.5!(C.text split)$) |- (NEW.text west);
		\draw[fill] ($(NEW.south)!0.5!(NEW.text split)$) circle (0.05);
		\draw[ptr] ($(NEW.south)!0.5!(NEW.text split)$) --++(right:6mm) |- (D.text west);
		\draw[fill] ($(D.south)!0.5!(D.text split)$) circle (0.05);
		\draw[ptr] ($(D.south)!0.5!(D.text split)$) --++(right:10mm) |- (E.text west);
		\draw[fill] ($(E.south)!0.5!(E.text split)$) circle (0.05);
		\draw[ptr] ($(E.south)!0.5!(E.text split)$) --++(right:10mm) |- (last.text west);
	\end{tikzpicture}
	\caption{中间插入}
\end{figure}

只要内存空间允许,能够插入链表的元素是无穷无尽的,不需要像数组考虑扩容的问题。如果不考虑插入之前的查找元素的过程,只考虑纯粹的插入操作,时间复杂度是$ O(1) $。\\

\mybox{插入结点}

\begin{lstlisting}[language=Java]
public void add(T elem) {
	Node<T> newNode = new Node<T>(elem, null);
	if (isEmpty()) {
		head = newNode;
		tail = newNode;
	} else {
		tail.next = newNode;
		tail = newNode;
	}
	size++;
}

public void add(int index, T elem) throws IndexOutOfBoundsException {
	if (index < 0 || index > size) {
		throw new IndexOutOfBoundsException("Index out of bounds");
	}

	Node<T> newNode = new Node<T>(elem, null);
	if (index == 0) {
		newNode.next = head;
		head = newNode;
	} else {
		Node<T> prev = head;
		for (int i = 0; i < index - 1; i++) {
			prev = prev.next;
		}
		newNode.next = prev.next;
		prev.next = newNode;
	}
	size++;
}
\end{lstlisting}

\vspace{0.5cm}

\subsection{删除结点}

链表的删除操作也分3种情况:

\subsubsection{尾部删除}

把倒数第二个结点的next指针指向空。

\begin{figure}[H]
	\centering
	\begin{tikzpicture}[node distance=2cm, auto, scale=1,
			every node/.style={scale=1}]
		\node[head, label=below:head] (head) {};
		\node[data, right of=head] (A) {\data};
		\node[above of=A,node distance=0.5cm,label=above:a1] (a1){};
		\node[data, right of=A] (B) {\data};
		\node[above of=B,node distance=0.5cm,label=above:a2] (a2){};
		\node[data, right of=B] (C) {\data};
		\node[above of=C,node distance=0.5cm,label=above:a3] (a3){};
		\node[data, right of=C, xshift=0.1cm] (D) {\data};
		\node[above of=D,node distance=0.5cm,label=above:a4] (a4){};
		\node[data, right of=D] (E) {\data};
		\node[above of=E,node distance=0.5cm,label=above:a5] (a5){};

		\draw[fill] (head.center) circle (0.05);

		\path[ptr] (head.center) --++(right:7.5mm) |- (A.text west);
		\draw[fill] ($(A.south)!0.5!(A.text split)$) circle (0.05);
		\draw[ptr] ($(A.south)!0.5!(A.text split)$) --++(right:10mm) |- (B.text west);
		\draw[fill] ($(B.south)!0.5!(B.text split)$) circle (0.05);
		\draw[ptr] ($(B.south)!0.5!(B.text split)$) --++(right:10mm) |- (C.text west);
		\draw[fill] ($(C.south)!0.5!(C.text split)$) circle (0.05);
		\draw[ptr] ($(C.south)!0.5!(C.text split)$) --++(right:10mm) |- (D.text west);
		\draw[fill] ($(D.south)!0.5!(D.text split)$) circle (0.05);

		\draw[fill] ($(D.south)!0.5!(D.text split)$) circle (0.05);
		\draw[ptr,red] ($(D.south)!0.5!(D.text split)$) -- (8.1,-1.7) -- (11.5,-1.7) -- (last.text west);
		\draw[fill] ($(E.south)!0.5!(E.text split)$) circle (0.05);

		\draw (12.3,0.3) node {NULL};

		\draw[-, red] (9.3,1) -- (11,-1);
	\end{tikzpicture}
	\caption{尾部删除}
\end{figure}

\subsubsection{头部删除}

把链表的头结点设置为原先头结点的next指针。

\begin{figure}[H]
	\centering
	\begin{tikzpicture}[node distance=2cm, auto, scale=1,
			every node/.style={scale=1}]
		\node[head, label=below:head, xshift=-0.5cm] (head) {};
		\node[data, right of=head] (A) {\data};
		\node[above of=A,node distance=0.5cm,label=above:a1] (a1){};
		\node[data, right of=A] (B) {\data};
		\node[above of=B,node distance=0.5cm,label=above:a2] (a2){};
		\node[data, right of=B] (C) {\data};
		\node[above of=C,node distance=0.5cm,label=above:a3] (a3){};
		\node[data, right of=C, xshift=0.1cm] (D) {\data};
		\node[above of=D,node distance=0.5cm,label=above:a4] (a4){};
		\node[data, right of=D] (E) {\data \nodepart{second} \texttt{NULL}};
		\node[above of=E,node distance=0.5cm,label=above:a5] (a5){};

		\draw[fill] (head.center) circle (0.05);
		\draw[fill] ($(A.south)!0.5!(A.text split)$) circle (0.05);
		\draw[fill] ($(B.south)!0.5!(B.text split)$) circle (0.05);
		\draw[ptr] ($(B.south)!0.5!(B.text split)$) --++(right:10mm) |- (C.text west);
		\draw[fill] ($(C.south)!0.5!(C.text split)$) circle (0.05);
		\draw[ptr] ($(C.south)!0.5!(C.text split)$) --++(right:10mm) |- (D.text west);
		\draw[fill] ($(D.south)!0.5!(D.text split)$) circle (0.05);
		\draw[fill] ($(D.south)!0.5!(D.text split)$) circle (0.05);
		\draw[ptr] ($(D.south)!0.5!(D.text split)$) --++(right:10mm) |- (E.text west);

		\draw[-, red] (0.5,1) -- (2.5,-1);
		\draw[->, red] (-0.5,0) -- (-0.5,2) -- (3,2) -- (3,0.6);
	\end{tikzpicture}
	\caption{头部删除}
\end{figure}

\subsubsection{中间删除}

把要删除的结点的前置结点的next指针,指向要删除结点的下一个结点。

\begin{figure}[H]
	\centering
	\begin{tikzpicture}[node distance=2cm, auto, scale=1,
			every node/.style={scale=1}]
		\node[head, label=below:head, xshift=-0.5cm] (head) {};
		\node[data, right of=head] (A) {\data};
		\node[above of=A,node distance=0.5cm,label=above:a1] (a1){};
		\node[data, right of=A] (B) {\data};
		\node[above of=B,node distance=0.5cm,label=above:a2] (a2){};
		\node[data, right of=B] (C) {\data};
		\node[above of=C,node distance=0.5cm,label=above:a3] (a3){};
		\node[data, right of=C, xshift=0.1cm] (D) {\data};
		\node[above of=D,node distance=0.5cm,label=above:a4] (a4){};
		\node[data, right of=D] (E) {data \nodepart{second} \texttt{NULL}};
		\node[above of=E,node distance=0.5cm,label=above:a5] (a5){};

		\draw[fill] (head.center) circle (0.05);
		\path[ptr] (head.center) --++(right:7.5mm) |- (A.text west);
		\draw[fill] ($(A.south)!0.5!(A.text split)$) circle (0.05);
		\draw[fill] ($(B.south)!0.5!(B.text split)$) circle (0.05);
		\draw[fill] ($(C.south)!0.5!(C.text split)$) circle (0.05);
		\draw[ptr] ($(C.south)!0.5!(C.text split)$) --++(right:10mm) |- (D.text west);
		\draw[fill] ($(D.south)!0.5!(D.text split)$) circle (0.05);
		\draw[fill] ($(D.south)!0.5!(D.text split)$) circle (0.05);
		\draw[ptr] ($(D.south)!0.5!(D.text split)$) --++(right:10mm) |- (E.text west);

		\draw[-, red] (2.5,1) -- (4.5,-1);
		\draw[->, red] (1.5,-0.3) -- (1.5,-2) -- (5.5,-2) -- (5.5,-0.3);
	\end{tikzpicture}
	\caption{中间删除}
\end{figure}

许多高级语言,如Java,拥有自动化的垃圾回收机制,所以不用刻意去释放被删除的结点,只要没有外部引用指向它们,被删除的结点会被自动回收。\\

如果不考虑删除操作之前的查找的过程,只考虑纯粹的删除操作,时间复杂度是$ O(1) $。\\

\mybox{删除结点}

\begin{lstlisting}[language=Java]
public T remove(int index) throws IndexOutOfBoundsException {
	if (index < 0 || index >= size) {
		throw new IndexOutOfBoundsException("Index out of bounds");
	}

	T data;
	if (index == 0) {
		data = head.data;
		head = head.next;
	} else {
		Node<T> prev = head;
		for (int i = 0; i < index - 1; i++) {
			prev = prev.next;
		}
		data = prev.next.data;
		prev.next = prev.next.next;
	}
	size--;
	return data;
}
\end{lstlisting}

\vspace{0.5cm}

\subsection{反转链表}

反转一个链表需要调整链表中指针的方向。\\

递归反转法的实现思想是从链表的尾结点开始,依次向前遍历,遍历过程依次改变各结点的指向,即另其指向前一个结点。\\

而迭代反转法的实现思想非常直接,就是从当前链表的首结点开始,一直遍历至链表尾部,期间会逐个改变所遍历到的结点的指针域,另其指向前一个结点。\\

\begin{figure}[H]
	\centering
	\begin{tikzpicture}[list/.style={rectangle split, rectangle split parts=2,
					draw, rectangle split horizontal}, >=stealth, start chain]
		\node[list,on chain] (A) {data};
		\node[list,on chain] (B) {data};
		\node[list,on chain] (C) {data};
		\node[on chain,draw,inner sep=6pt] (D) {};
		\draw (D.north east) -- (D.south west);
		\draw (D.north west) -- (D.south east);
		\draw[*->] let \p1 = (A.two), \p2 = (A.center) in (\x1,\y2) -- (B);
		\draw[*->] let \p1 = (B.two), \p2 = (B.center) in (\x1,\y2) -- (C);
		\draw[*->] let \p1 = (C.two), \p2 = (C.center) in (\x1,\y2) -- (D);
	\end{tikzpicture}
	\begin{tikzpicture}[
			list/.style={
					rectangle split,
					rectangle split parts=2,
					draw,
					rectangle split horizontal
				},
			dotarrow/.style={Circle-Stealth},
			start chain
		]
		\node[on chain,draw,inner sep=6pt] (D) {};
		\draw (D.north east) -- (D.south west);
		\draw (D.north west) -- (D.south east);
		\node[list,on chain] (A) {data};
		\node[list,on chain] (B) {data};
		\node[list,on chain] (C) {data};

		\draw[dotarrow] (A.two |- A.center) to[out=100,in=60,distance=2cm] (D);
		\draw[dotarrow] (B.two |- B.center) to[out=100,in=60,distance=2cm] (A.east);
		\draw[dotarrow] (C.two |- C.center) to[out=100,in=60,distance=2cm] (B.east);
	\end{tikzpicture}
	\caption{反转链表}
\end{figure}

\mybox{反转链表(递归)}

\begin{lstlisting}[language=Java]
private Node<T> reverseRecursive(Node<T> current) {
	if (current == null || current.next == null) {
		return current;
	}
	Node<T> newHead = reverse(current.next);
	current.next.next = current;
	current.next = null;
	return newHead;
}

public void reverseRecursive() {
	head = reverseRecursive(head);
}
\end{lstlisting}

\vspace{0.5cm}

\mybox{反转链表(迭代)}

\begin{lstlisting}[language=Java]
public void reverse() {
	Node<T> prev = null;
	Node<T> current = head;
	Node<T> next = null;
	while (current != null) {
		next = current.next;
		current.next = prev;
		prev = current;
		current = next;
	}
	head = prev;
}
\end{lstlisting}

\newpage