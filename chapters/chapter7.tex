\chapter{分治法}

\section{分治法}

\subsection{分治法(Divide and Conquer)}

分治策略是将原问题分解为k个子问题,并对k个子问题分别求解。如果子问题的规模仍然不够小,则再划分为k个子问题,如此递归的进行下去,直到问题规模足够小,很容易求出其解为止。\\

\begin{figure}[H]
	\centering
	\begin{tikzpicture}[
			level distance=2.5cm,
			level 1/.style={sibling distance=6cm},
			level 2/.style={sibling distance=3cm},
			level 3/.style={sibling distance=2cm}
		]
		\node {problem}
		child {
				node {subproblem}
				child {node {compute}}
				child {node {compute}}
			}
		child {
				node {subproblem}
				child {node {compute}}
				child {node {compute}}
			};
	\end{tikzpicture}
	\caption{分治法}
\end{figure}

将求出的小规模的问题的解合并为一个更大规模的问题的解,自底向上逐步求出原来问题的解。\\

分治法的适用条件有以下四点:

\begin{enumerate}
	\item 该问题的规模缩小到一定的程度就可以容易地解决。

	\item 该问题可以分解为若干个规模较小的相同问题,即该问题具有最优子结构性质。

	\item 利用该问题分解出的子问题的解可以合并为该问题的解。

	\item 该问题所分解出的各个子问题是相互独立的,即子问题之间不包含公共的子问题(如果各子问题是不独立的,则分治法要做许多不必要的工作,重复地解公共的子问题,此时虽然也可用分治法,但一般用动态规划较好)。
\end{enumerate}

人们在大量事件中发现,在用分治法设计算法时,最好使子问题的规模大致相同。即将一个问题分成大小相等的k个子问题的处理方法是行之有效的。这种使子问题规模大致相等的做法是出自一种平衡子问题的思想,它几乎总是比子问题规模不等的做法要好。\\

\subsection{二分搜索}

一个装有16个硬币的袋子,16个硬币中有一个是伪造的,并且那个伪造的硬币比真的硬币要轻一些,要求找出这个伪造的硬币。只提供一台可用来比较两组硬币重量的仪器,利用这台仪器,可以知道两组硬币的重量是否相同。\\

算法思想是将16个硬币等分成A、B两份,将A放仪器的一边,B放另一边,如果A袋轻,则表明伪币在A,解子问题A即可,否则解子问题B。\\

二分搜索每执行一次循环,待搜索数组的大小减少一半,在最坏情况下,循环被执行了$ O(logn) $次,循环体内运算需要$ O(1) $时间。因此,整个算法在最坏情况下的计算时间复杂性为$ O(logn) $。

\newpage

\section{大整数加法}

\subsection{大整数加法}

如果有两个很大的整数,如何求出它们的和?\\

这还不简单?直接用long类型存储,在程序里相加不就行了?\\

C/C++中的int类型能表示的范围是$ -2^{31} \sim 2^{31} - 1 $,unsigned类型能表示的范围是$ 0 \sim 2^{32} - 1 $,所以int和unsigned类型变量都不能保存超过10位的整数。\\

有时需要参与运算的数可能会远远不止10位,例如计算$ 100! $的精确值。即便使用能表示很大数值范围的double变量,但是由于double变量只有64位,精度也不足以表示一个超过100位的整数。我们称这种基本数据类型无法表示的整数为大整数。\\

在小学的时候,老师教我们用列竖式的方式计算两个整数的和。

\begin{table}[H]
	\centering
	\begin{tabular}{cD{.}{.}{3}}
		  & 426709752318 \\
		+ & 95481253129  \\
		\hline
		= & 522191005447
	\end{tabular}
\end{table}

不仅仅是人脑,对于计算机来说同样如此。对于大整数,我们无法一步到位直接算出结果,所以不得不把计算拆解成一个一个子步骤。\\

可是,既然大整数已经超出了long类型的范围,我们如何来存储这样的整数呢?\\

存放大整数最简单的方法就是使用数组,可以用数组的每一个元素存储整数的每一个数位。如果给定大整数的最长位数是n,那么按位计算的时间复杂度是$ O(n) $。\\

\mybox{大整数加法}

\begin{lstlisting}[language=Python]
def big_int_add(num1, num2):
    # 其中一个数为0,直接返回另一个数
    if num1 == "0":
        return num2
    elif num2 == "0":
        return num1
    
    # 计算两个数中较长的整数位数
    max_len = max(len(num1), len(num2))
    # 让位数较短的整数前面补0对齐
    num1 = '0' * (max_len - len(num1)) + num1
    num2 = '0' * (max_len - len(num2)) + num2

    result = ""         # 结果
    carry = 0           # 保存进位
    # 从右往左逐位相加
    for i in range(max_len - 1, -1, -1):
        s = int(num1[i]) + int(num2[i]) + carry
        result = str(s % 10) + result
        carry = s // 10
    
    # 判断最高位是否有进位
    if carry > 0:
        result = str(carry) + result
    
    # 去除结果前面多余的0
    i = 0
    while result[i] == '0':
        i += 1
    return result[i:]
\end{lstlisting}

这种思路其实还存在一个可优化的地方。我们之前是把大整数按照每一个十进制数位来拆分,比如较大整数的长度有50位,那么需要创建一个51位的数组,数组的每个元素存储其中一位。\\

真的有必要把原整数拆分得那么细吗?显然不需要,只需要拆分到可以被直接计算的程度就够了。int类型的取值范围是$ -2147483648 \sim 2147483647 $,最多有10位整数。为了防止溢出,可以把大整数的每9位作为数组的一个元素,进行加法运算。如此一来,占用空间和运算次数,都被压缩了9倍。

\begin{figure}[H]
	\centering
	\begin{tikzpicture}[
			level distance=2.5cm,
			level 1/.style={sibling distance=2cm}
		]
		\node {50位}
		child {node {9位}}
		child {node {9位}}
		child {node {9位}}
		child {node {9位}}
		child {node {9位}}
		child {node {9位}};
	\end{tikzpicture}
	\caption{大整数加法优化}
\end{figure}

在Java中,工具类BigInteger和BidDecimal的底层实现也是把大整数拆分成数组进行运算,与此处的思路大体类似。

\newpage