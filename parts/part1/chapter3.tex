\chapter{链表}

\section{链表}

\subsection{单向链表(Singly Linked List)}

为避免元素的移动,采用线性表的另一种存储方式:链式存储结构。链表是一种在物理上非连续、非顺序的数据结构,由若干结点(node)所组成。 \\

单向链表的每一个结点又包含两部分,一部分是存放数据的数据域data,另一部分是指向下一个结点的指针域next。结点可以在运行时动态生成。

\vspace{-0.5cm}

\begin{lstlisting}[language=C]
typedef struct Node {
    dataType data;          // 数据域
    struct Node *next;		// 指针域
} Node;
\end{lstlisting}

链表的第一个结点被称为头结点,最后一个节点被称为尾结点,尾结点的next指针指向空NULL。 \\

\begin{figure}[H]
	\centering
	\begin{tikzpicture}[list/.style={rectangle split, rectangle split parts=2,
					draw, rectangle split horizontal}, >=stealth, start chain]
		\node[list,on chain] (A) {data};
		\node[list,on chain] (B) {data};
		\node[list,on chain] (C) {data};
		\node[on chain,draw,inner sep=6pt] (D) {};
		\draw (D.north east) -- (D.south west);
		\draw (D.north west) -- (D.south east);
		\draw[*->] let \p1 = (A.two), \p2 = (A.center) in (\x1,\y2) -- (B);
		\draw[*->] let \p1 = (B.two), \p2 = (B.center) in (\x1,\y2) -- (C);
		\draw[*->] let \p1 = (C.two), \p2 = (C.center) in (\x1,\y2) -- (D);
	\end{tikzpicture}
	\caption{单向链表}
\end{figure}

与数组按照下标来随机寻找元素不同,对于链表的其中一个结点A,只能根据结点A的next指针来找到该结点的下一个结点B,再根据结点B的next指针找到下一个节点C…… \\

数组在内存中的存储方式是顺序存储,链表在内存中的存储方式则是随机存储。链表采用了见缝插针的方式,每一个结点分布在内存的不同位置,依靠next指针关联起来。这样可以灵活有效地利用零散的碎片空间。

\subsection{双向链表(Doubly Linked List)}

那么,通过链表的一个结点,如何能快速找到它的前一个结点呢?要想让每个结点都能回溯到它的前置结点,可以使用双向链表。 \\

双向链表比单向链表稍微复杂一点,它的每一个结点除了拥有data和next指针,还拥有指向前置结点的prev指针。 \\

\begin{figure}[H]
	\centering
	\begin{tikzpicture}[list/.style={rectangle split, rectangle split parts=3,
					draw, rectangle split horizontal}, >=stealth, start chain]
		\node[on chain,draw,inner sep=6pt] (NULL1) {};
		\node[list,on chain] (A) {prev \nodepart{second} data \nodepart{third} next};
		\node[list,on chain] (B) {prev \nodepart{second} data \nodepart{third} next};
		\node[list,on chain] (C) {prev \nodepart{second} data \nodepart{third} next};
		\node[on chain,draw,inner sep=6pt] (NULL2) {};
		\draw (NULL1.north east) -- (NULL1.south west);
		\draw (NULL1.north west) -- (NULL1.south east);
		\draw (NULL2.north east) -- (NULL2.south west);
		\draw (NULL2.north west) -- (NULL2.south east);
		\draw[<-] let \p1 = (A.west), \p2 = (A.center) in (NULL1) -- (\x1,\y2);
		\draw[<->] let \p1 = (A.east), \p2 = (A.center) in (\x1,\y2) -- (B);
		\draw[<->] let \p1 = (B.east), \p2 = (B.center) in (\x1,\y2) -- (C);
		\draw[->] let \p1 = (C.east), \p2 = (C.center) in (\x1,\y2) -- (NULL2);
	\end{tikzpicture}
	\caption{双向链表}
\end{figure}

单向链表只能从头到尾遍历,只能找到后继,无法找到前驱,因此遍历的时候不会死循环。而双向链表需要多分配一个指针的存储空间,每个结点有两个指针,分别指向直接前驱和直接后继。

\subsection{循环链表(Circular Linked List)}

除了单向链表和双向链表以外,还有循环链表。对于单向循环链表,尾结点的next指针指向头结点。对于双向循环链表,尾结点的next指针指向头结点,并且头结点的prev指针指向尾结点。 \\

\begin{figure}[H]
	\centering
	\begin{tikzpicture}[
			list/.style={
					rectangle split,
					rectangle split parts=2,
					draw,
					rectangle split horizontal
				},
			dotarrow/.style={Circle-Stealth},
			start chain
		]
		\node[list,on chain] (A) {data};
		\node[list,on chain] (B) {data};
		\node[list,on chain] (C) {data};
		\draw[dotarrow] (A.two |- A.center) -- (B);
		\draw[dotarrow] (B.two |- B.center)  -- (C);
		\draw[dotarrow] (C.two |- C.center) to[out=-20,in=190,distance=4cm] (A);
	\end{tikzpicture}
	\caption{单向循环链表}
\end{figure}

\begin{figure}[H]
	\centering
	\begin{tikzpicture}[list/.style={
					rectangle split,
					rectangle split parts=3,
					draw,
					rectangle split horizontal
				},
			dotarrow/.style={Circle-Stealth},
			start chain]
		\node[list,on chain] (A) {prev \nodepart{second} data \nodepart{third} next};
		\node[list,on chain] (B) {prev \nodepart{second} data \nodepart{third} next};
		\node[list,on chain] (C) {prev \nodepart{second} data \nodepart{third} next};

		\draw[dotarrow] (A.center -| A.west) to[out=-220,in=20,distance=2cm] (C.north);
		\draw[<->] let \p1 = (A.east), \p2 = (A.center) in (\x1,\y2) -- (B);
		\draw[<->] let \p1 = (B.east), \p2 = (B.center) in (\x1,\y2) -- (C);
		\draw[dotarrow] (C.east |- C.center) to[out=-20,in=220,distance=2cm] (A.south);
	\end{tikzpicture}
	\caption{双向循环链表}
\end{figure}

\newpage

\section{链表的增删改查}

\subsection{查找结点}

在查找元素时,链表不像数组那样可以通过下标快速进行定位,只能从头结点开始向后一个一个结点逐一查找。 \\

链表中的数据只能按顺序进行访问,最坏的时间复杂度是$ O(n) $。 \\

\mybox{查找结点}

\begin{lstlisting}[language=C]
Node* search(List *head, dataType val) {
    // 查找元素位置
    Node *temp = head;
    while(temp) {
        if(temp->data == val) {
            return temp;
        }
        temp = temp->next;
    }
    return NULL;        // 未找到
}
\end{lstlisting}

\subsection{更新结点}

如果不考虑查找结点的过程,链表的更新过程会像数组那样简单,直接把旧数据替换成新数据即可。 \\

\mybox{更新结点}

\begin{lstlisting}[language=C]
void replace(List *head, int pos, dataType val) {
    // 找到元素位置
    Node *temp = head;
    for(int i = 0; i < pos; i++) {
        temp = temp->next;
    }
    temp->data = val;
}
\end{lstlisting}

\subsection{插入结点}

链表插入结点,分为3种情况:

\subsubsection{尾部插入}

把最后一个结点的next指针指向新插入的结点。

\begin{figure}[H]
	\centering
	\begin{tikzpicture}[node distance=2cm, auto, scale=1,
			every node/.style={scale=1}]
		\node[head, label=below:head] (head) {};
		\node[data, right of=head] (A) {\data};
		\node[above of=A,node distance=0.5cm,label=above:a1] (a1){};
		\node[data, right of=A] (B) {\data};
		\node[above of=B,node distance=0.5cm,label=above:a2] (a2){};
		\node[data, right of=B] (C) {\data};
		\node[above of=C,node distance=0.5cm,label=above:a3] (a3){};
		\node[data, right of=C, xshift=0.1cm] (D) {\data};%, xshift=1.5cm
		\node[above of=D,node distance=0.5cm,label=above:a4] (a4){};
		\node[data, right of=D] (E) {\data};
		\node[above of=E,node distance=0.5cm,label=above:a5] (a5){};
		\node[data, right of=E] (last) {data \nodepart{second} \texttt{NULL}};
		\node[above of=last,node distance=0.5cm,label=above:new] (new){};

		\draw[fill] (head.center) circle (0.05);

		\path[ptr] (head.center) --++(right:7.5mm) |- (A.text west);
		\draw[fill] ($(A.south)!0.5!(A.text split)$) circle (0.05);
		\draw[ptr] ($(A.south)!0.5!(A.text split)$) --++(right:10mm) |- (B.text west);
		\draw[fill] ($(B.south)!0.5!(B.text split)$) circle (0.05);
		\draw[ptr] ($(B.south)!0.5!(B.text split)$) --++(right:10mm) |- (C.text west);
		\draw[fill] ($(C.south)!0.5!(C.text split)$) circle (0.05);
		\draw[ptr] ($(C.south)!0.5!(C.text split)$) --++(right:10mm) |- (D.text west);
		\draw[fill] ($(D.south)!0.5!(D.text split)$) circle (0.05);

		\draw[fill] ($(D.south)!0.5!(D.text split)$) circle (0.05);
		\draw[ptr] ($(D.south)!0.5!(D.text split)$) --++(right:10mm) |- (E.text west);
		\draw[fill] ($(E.south)!0.5!(E.text split)$) circle (0.05);
		\draw[ptr] ($(E.south)!0.5!(E.text split)$) --++(right:10mm) |- (last.text west);
	\end{tikzpicture}
	\caption{尾部插入}
\end{figure}

\subsubsection{头部插入}

先把新结点的next指针指向原先的头结点,再把新结点设置为链表的头结点。

\begin{figure}[H]
	\centering
	\begin{tikzpicture}[node distance=2cm, auto, scale=1,
			every node/.style={scale=1}]
		\node[head, label=below:head, xshift=-0.5cm] (head) {};
		\node[data, right of=head] (A) {\data};
		\node[above of=A,node distance=0.5cm,label=above:a1] (a1){};
		\node[data, right of=A] (B) {\data};
		\node[above of=B,node distance=0.5cm,label=above:a2] (a2){};
		\node[data, right of=B] (C) {\data};
		\node[above of=C,node distance=0.5cm,label=above:a3] (a3){};
		\node[data, below of=head, xshift=0.6cm, yshift=0.5cm] (NEW) {\data};
		\node[above of=NEW,node distance=0.3cm,label=above:new] (new){};
		\node[data, right of=C, xshift=0.1cm] (D) {\data};%, xshift=1.5cm
		\node[above of=D,node distance=0.5cm,label=above:a4] (a4){};
		\node[data, right of=D] (E) {\data};
		\node[above of=E,node distance=0.5cm,label=above:a5] (a5){};
		\node[data, right of=E] (last) {data \nodepart{second} \texttt{NULL}};
		\node[above of=last,node distance=0.5cm,label=above:a6] (a6){};

		\draw[fill] (head.center) circle (0.05);

		\path[ptr] (head.center) --++(left:7.5mm) |- (NEW.text west);
		\draw[fill] ($(NEW.south)!0.5!(NEW.text split)$) circle (0.05);
		\draw[ptr] ($(NEW.south)!0.5!(NEW.text split)$) --++(right:6mm) |- (A.text west);
		\draw[fill] ($(A.south)!0.5!(A.text split)$) circle (0.05);
		\draw[ptr] ($(A.south)!0.5!(A.text split)$) --++(right:10mm) |- (B.text west);
		\draw[fill] ($(B.south)!0.5!(B.text split)$) circle (0.05);
		\draw[ptr] ($(B.south)!0.5!(B.text split)$) --++(right:10mm) |- (C.text west);
		\draw[fill] ($(C.south)!0.5!(C.text split)$) circle (0.05);
		\draw[ptr] ($(C.south)!0.5!(C.text split)$) --++(right:10mm) |- (D.text west);
		\draw[fill] ($(D.south)!0.5!(D.text split)$) circle (0.05);
		\draw[fill] ($(D.south)!0.5!(D.text split)$) circle (0.05);
		\draw[ptr] ($(D.south)!0.5!(D.text split)$) --++(right:10mm) |- (E.text west);
		\draw[fill] ($(E.south)!0.5!(E.text split)$) circle (0.05);
		\draw[ptr] ($(E.south)!0.5!(E.text split)$) --++(right:10mm) |- (last.text west);
	\end{tikzpicture}
	\caption{头部插入}
\end{figure}

\subsubsection{中间插入}

先把新结点的next指针指向插入位置的结点,再将插入位置的前置结点的next指针指向新结点。

\begin{figure}[H]
	\centering
	\begin{tikzpicture}[node distance=2cm, auto, scale=1,
			every node/.style={scale=1}]
		\node[head, label=below:head] (head) {};
		\node[data, right of=head] (A) {\data};
		\node[above of=A,node distance=0.5cm,label=above:a1] (a1){};
		\node[data, right of=A] (B) {\data};
		\node[above of=B,node distance=0.5cm,label=above:a2] (a2){};
		\node[data, right of=B] (C) {\data};
		\node[above of=C,node distance=0.5cm,label=above:a3] (a3){};
		\node[data, below of=C, xshift=0.8cm, yshift=0.5cm] (NEW) {\data};
		\node[above of=NEW,node distance=0.3cm,label=above:new] (new){};
		\node[data, right of=C, xshift=0.1cm] (D) {\data};%, xshift=1.5cm
		\node[above of=D,node distance=0.5cm,label=above:a4] (a4){};
		\node[data, right of=D] (E) {\data};
		\node[above of=E,node distance=0.5cm,label=above:a5] (a5){};
		\node[data, right of=E] (last) {data \nodepart{second} \texttt{NULL}};
		\node[above of=last,node distance=0.5cm,label=above:a6] (a6){};

		\draw[fill] (head.center) circle (0.05);

		\path[ptr] (head.center) --++(right:7.5mm) |- (A.text west);
		\draw[fill] ($(A.south)!0.5!(A.text split)$) circle (0.05);
		\draw[ptr] ($(A.south)!0.5!(A.text split)$) --++(right:10mm) |- (B.text west);
		\draw[fill] ($(B.south)!0.5!(B.text split)$) circle (0.05);
		\draw[ptr] ($(B.south)!0.5!(B.text split)$) --++(right:10mm) |- (C.text west);
		\draw[fill] ($(C.south)!0.5!(C.text split)$) circle (0.05);

		\draw[ptr] ($(C.south)!0.5!(C.text split)$) |- (NEW.text west);
		\draw[fill] ($(NEW.south)!0.5!(NEW.text split)$) circle (0.05);
		\draw[ptr] ($(NEW.south)!0.5!(NEW.text split)$) --++(right:6mm) |- (D.text west);
		\draw[fill] ($(D.south)!0.5!(D.text split)$) circle (0.05);
		\draw[ptr] ($(D.south)!0.5!(D.text split)$) --++(right:10mm) |- (E.text west);
		\draw[fill] ($(E.south)!0.5!(E.text split)$) circle (0.05);
		\draw[ptr] ($(E.south)!0.5!(E.text split)$) --++(right:10mm) |- (last.text west);
	\end{tikzpicture}
	\caption{中间插入}
\end{figure}

只要内存空间允许,能够插入链表的元素是无穷无尽的,不需要像数组考虑扩容的问题。如果不考虑插入之前的查找元素的过程,只考虑纯粹的插入操作,时间复杂度是$ O(1) $。

\subsection{删除结点}

链表的删除操作也分3种情况:

\subsubsection{尾部删除}

把倒数第二个结点的next指针指向空。

\begin{figure}[H]
	\centering
	\begin{tikzpicture}[node distance=2cm, auto, scale=1,
			every node/.style={scale=1}]
		\node[head, label=below:head] (head) {};
		\node[data, right of=head] (A) {\data};
		\node[above of=A,node distance=0.5cm,label=above:a1] (a1){};
		\node[data, right of=A] (B) {\data};
		\node[above of=B,node distance=0.5cm,label=above:a2] (a2){};
		\node[data, right of=B] (C) {\data};
		\node[above of=C,node distance=0.5cm,label=above:a3] (a3){};
		\node[data, right of=C, xshift=0.1cm] (D) {\data};
		\node[above of=D,node distance=0.5cm,label=above:a4] (a4){};
		\node[data, right of=D] (E) {\data};
		\node[above of=E,node distance=0.5cm,label=above:a5] (a5){};

		\draw[fill] (head.center) circle (0.05);

		\path[ptr] (head.center) --++(right:7.5mm) |- (A.text west);
		\draw[fill] ($(A.south)!0.5!(A.text split)$) circle (0.05);
		\draw[ptr] ($(A.south)!0.5!(A.text split)$) --++(right:10mm) |- (B.text west);
		\draw[fill] ($(B.south)!0.5!(B.text split)$) circle (0.05);
		\draw[ptr] ($(B.south)!0.5!(B.text split)$) --++(right:10mm) |- (C.text west);
		\draw[fill] ($(C.south)!0.5!(C.text split)$) circle (0.05);
		\draw[ptr] ($(C.south)!0.5!(C.text split)$) --++(right:10mm) |- (D.text west);
		\draw[fill] ($(D.south)!0.5!(D.text split)$) circle (0.05);

		\draw[fill] ($(D.south)!0.5!(D.text split)$) circle (0.05);
		\draw[ptr,red] ($(D.south)!0.5!(D.text split)$) -- (8.1,-1.7) -- (11.5,-1.7) -- (last.text west);
		\draw[fill] ($(E.south)!0.5!(E.text split)$) circle (0.05);

		\draw (12.3,0.3) node {NULL};

		\draw[-, red] (9.3,1) -- (11,-1);
	\end{tikzpicture}
	\caption{尾部删除}
\end{figure}

\subsubsection{头部删除}

把链表的头结点设置为原先头结点的next指针。

\begin{figure}[H]
	\centering
	\begin{tikzpicture}[node distance=2cm, auto, scale=1,
			every node/.style={scale=1}]
		\node[head, label=below:head, xshift=-0.5cm] (head) {};
		\node[data, right of=head] (A) {\data};
		\node[above of=A,node distance=0.5cm,label=above:a1] (a1){};
		\node[data, right of=A] (B) {\data};
		\node[above of=B,node distance=0.5cm,label=above:a2] (a2){};
		\node[data, right of=B] (C) {\data};
		\node[above of=C,node distance=0.5cm,label=above:a3] (a3){};
		\node[data, right of=C, xshift=0.1cm] (D) {\data};
		\node[above of=D,node distance=0.5cm,label=above:a4] (a4){};
		\node[data, right of=D] (E) {\data \nodepart{second} \texttt{NULL}};
		\node[above of=E,node distance=0.5cm,label=above:a5] (a5){};

		\draw[fill] (head.center) circle (0.05);
		\draw[fill] ($(A.south)!0.5!(A.text split)$) circle (0.05);
		\draw[fill] ($(B.south)!0.5!(B.text split)$) circle (0.05);
		\draw[ptr] ($(B.south)!0.5!(B.text split)$) --++(right:10mm) |- (C.text west);
		\draw[fill] ($(C.south)!0.5!(C.text split)$) circle (0.05);
		\draw[ptr] ($(C.south)!0.5!(C.text split)$) --++(right:10mm) |- (D.text west);
		\draw[fill] ($(D.south)!0.5!(D.text split)$) circle (0.05);
		\draw[fill] ($(D.south)!0.5!(D.text split)$) circle (0.05);
		\draw[ptr] ($(D.south)!0.5!(D.text split)$) --++(right:10mm) |- (E.text west);

		\draw[-, red] (0.5,1) -- (2.5,-1);
		\draw[->, red] (-0.5,0) -- (-0.5,2) -- (3,2) -- (3,0.6);
	\end{tikzpicture}
	\caption{头部删除}
\end{figure}

\subsubsection{中间删除}

把要删除的结点的前置结点的next指针,指向要删除结点的下一个结点。

\begin{figure}[H]
	\centering
	\begin{tikzpicture}[node distance=2cm, auto, scale=1,
			every node/.style={scale=1}]
		\node[head, label=below:head, xshift=-0.5cm] (head) {};
		\node[data, right of=head] (A) {\data};
		\node[above of=A,node distance=0.5cm,label=above:a1] (a1){};
		\node[data, right of=A] (B) {\data};
		\node[above of=B,node distance=0.5cm,label=above:a2] (a2){};
		\node[data, right of=B] (C) {\data};
		\node[above of=C,node distance=0.5cm,label=above:a3] (a3){};
		\node[data, right of=C, xshift=0.1cm] (D) {\data};
		\node[above of=D,node distance=0.5cm,label=above:a4] (a4){};
		\node[data, right of=D] (E) {data \nodepart{second} \texttt{NULL}};
		\node[above of=E,node distance=0.5cm,label=above:a5] (a5){};

		\draw[fill] (head.center) circle (0.05);
		\path[ptr] (head.center) --++(right:7.5mm) |- (A.text west);
		\draw[fill] ($(A.south)!0.5!(A.text split)$) circle (0.05);
		\draw[fill] ($(B.south)!0.5!(B.text split)$) circle (0.05);
		\draw[fill] ($(C.south)!0.5!(C.text split)$) circle (0.05);
		\draw[ptr] ($(C.south)!0.5!(C.text split)$) --++(right:10mm) |- (D.text west);
		\draw[fill] ($(D.south)!0.5!(D.text split)$) circle (0.05);
		\draw[fill] ($(D.south)!0.5!(D.text split)$) circle (0.05);
		\draw[ptr] ($(D.south)!0.5!(D.text split)$) --++(right:10mm) |- (E.text west);

		\draw[-, red] (2.5,1) -- (4.5,-1);
		\draw[->, red] (1.5,-0.3) -- (1.5,-2) -- (5.5,-2) -- (5.5,-0.3);
	\end{tikzpicture}
	\caption{中间删除}
\end{figure}

许多高级语言,如Java,拥有自动化的垃圾回收机制,所以不用刻意去释放被删除的结点,只要没有外部引用指向它们,被删除的结点会被自动回收。 \\

如果不考虑删除操作之前的查找的过程,只考虑纯粹的删除操作,时间复杂度是$ O(1) $。

\newpage

\section{带头结点的链表}

\subsection{带头结点的链表}

为了方便链表的插入、删除操作,在链表加上头结点之后,无论链表是否为空,头指针始终指向头结尾。因此对于空表和非空表的处理也统一了,方便了链表的操作,也减少了程序的复杂性和出现bug的机会。 \\

\mybox{插入结点}

\begin{lstlisting}[language=C]
void insert(List *head, int pos, dataType val) {
    Node *newNode = (Node *)malloc(sizeof(Node));
    newNode->data = val;
    newNode->next =  NULL;
    
    // 找到插入位置
    Node *temp = head;
    for(int i = 0; i < pos; i++) {
        temp = temp->next;
    }
    newNode->next = temp->next;
    temp->next = newNode;
}
\end{lstlisting}

\vspace{0.5cm}

\mybox{删除结点}

\begin{lstlisting}[language=C]
void delete(List *head, int pos) {
    Node *temp = head;
    for(int i = 0; i < pos; i++) {
        temp = temp->next;
    }
    Node *del = temp->next;
    temp->next = del->next;
    free(del);
    del = NULL;
}
\end{lstlisting}

\subsection{数组VS链表}

数据结构没有绝对的好与坏,数组和链表各有千秋。

\begin{table}[H]
	\centering
	\setlength{\tabcolsep}{5mm}{
		\begin{tabular}{|c|l|l|}
			\hline
			\textbf{比较内容} & \textbf{数组}          & \textbf{链表}                \\
			\hline
			基本              & 一组固定数量的数据项   & 可变数量的数据项             \\
			\hline
			大小              & 声明期间指定           & 无需指定,执行期间增长或收缩 \\
			\hline
			存储分配          & 元素位置在编译期间分配 & 元素位置在运行时分配         \\
			\hline
			元素顺序          & 连续存储               & 随机存储                     \\
			\hline
			访问元素          & 直接访问:索引、下标   & 顺序访问:指针遍历           \\
			\hline
			插入/删除         & 速度慢                 & 快速、高效                   \\
			\hline
			查找              & 线性查找、二分查找     & 线性查找                     \\
			\hline
			内存利用率        & 低效                   & 高效                         \\
			\hline
		\end{tabular}
	}
	\caption{数组VS链表}
\end{table}

数组的优势在于能够快速定位元素,对于读操作多、写操作少的场景来说,用数组更合适一些。 \\

相反,链表的优势在于能够灵活地进行插入和删除操作,如果需要频繁地插入、删除元素,用链表更合适一些。

\newpage

\section{倒数第k个结点}

\subsection{倒数第k个结点}

输入一个链表,输出该链表中倒数第k个结点。例如一个链表有6个结点[0, 1, 2, 3, 4, 5],这个链表的倒数第3个结点是值为3的结点。 \\

算法的思路是设置两个指针p1和p2,它们都从头开始出发,p2指针先出发k个结点,然后p1指针再进行出发,当p2指针到达链表的尾结点时,则p1指针的位置就是链表的倒数第k个结点。 \\

\mybox{倒数第k个结点}

\begin{lstlisting}[language=Java]
public static Node findLastKth(LinkedList list, int k) {
    Node p1 = list.getHead();
    Node p2 = list.getHead();

    int i = 0;
    while(p2 != null && i < k) {
        p2 = p2.next;
        i++;
    }
    while(p2 != null) {
        p1 = p1.next;
        p2 = p2.next;
    }
    return p1;
}
\end{lstlisting}

\newpage

\section{环形链表}

\subsection{环形链表}

一个单向链表中有可能出现环,不允许修改链表结构,如何在时间复杂度$ O(n) $、空间复杂度$ O(1) $内判断出这个链表是有环链表?如果带环,环的长度是多少?环的入口结点是哪个? \\

暴力算法首先从头结点开始,依次遍历单链表的每一个结点。对于每个结点都从头重新遍历之前的所有结点。如果发现当前结点与之前结点存在相同ID,则说明该结点被遍历过两次,链表有环。但是这种方法的时间复杂度太高。 \\

另一种方法就是利用快慢指针,首先创建两个指针p1和p2,同时指向头结点,然后让p1每次向后移动一个位置,让p2每次向后移动两个位置。在环中,快指针一定会反复遇到慢指针。比如在一个环形跑道上,两个运动员在同一地点起跑,一个运动员速度快,一个运动员速度慢。当两人跑了一段时间,速度快的运动员必然会从速度慢的运动员身后再次追上并超过。 \\

环的长度可以通过从快慢指针相遇的结点开始再走一圈,下一次回到该点的时的移动步数,即环的长度n。 \\

环的入口可以利用类似获取链表倒数第k个结点的方法,准备两个指针p1和p2,让p2先走n步,然后p1和p2一块走。当两者相遇时,即为环的入口处。 \\

\mybox{环形链表}

\begin{lstlisting}[language=Java]
public static Node cycleNode(LinkedList list) {
    Node p1 = list.getHead();
    Node p2 = list.getHead();

    while(p1 != null && p2 != null) {
        if(p2.next == null || p2.next.next == null) {
            return null;
        }
        p1 = p1.next;
        p2 = p2.next.next;
        if(p1 == p2) {
            return p1;
        }
    }
    return null;
}

public static int cycleLength(LinkedList list) {
    Node node = cycleNode(list);
    if(node == null) {
        return 0;
    }
    int len = 1;
    Node cur = node.next;
    while(cur != node) {
        cur = cur.next;
        len++;
    }
    return len;
}

public static Node cycleEntrance(LinkedList list) {
    int n = cycleLength(list);
    if(n == 0) {
        return null;
    }

    Node p1 = list.getHead();
    Node p2 = list.getHead();
    for(int i = 0; i < n; i++) {
        p2 = p2.next;
    }

    while(p1 != p2) {
        p1 = p1.next;
        p2 = p2.next;
    }
    return p1;
}
\end{lstlisting}

\newpage

\section{反转链表}

\subsection{逆序输出链表}

输入一个单链表,从尾到头打印链表每个结点的值。由于单链表的遍历只能从头到尾,所以可以通过递归达到链表尾部,然后在回溯时输出每个结点的值。 \\

\mybox{逆序输出链表}

\begin{lstlisting}[language=Java]
public static void inverse(Node head) {
    if(head != null) {
        inverse(head.next);
        System.out.print(head.data + " ");
    }
}
\end{lstlisting}

\subsection{反转链表}

反转一个链表需要调整链表中指针的方向。 \\

递归反转法的实现思想是从链表的尾结点开始,依次向前遍历,遍历过程依次改变各结点的指向,即另其指向前一个结点。 \\

而迭代反转法的实现思想非常直接,就是从当前链表的首结点开始,一直遍历至链表尾部,期间会逐个改变所遍历到的结点的指针域,另其指向前一个结点。 \\

\begin{figure}[H]
	\centering
	\begin{tikzpicture}[list/.style={rectangle split, rectangle split parts=2,
					draw, rectangle split horizontal}, >=stealth, start chain]
		\node[list,on chain] (A) {data};
		\node[list,on chain] (B) {data};
		\node[list,on chain] (C) {data};
		\node[on chain,draw,inner sep=6pt] (D) {};
		\draw (D.north east) -- (D.south west);
		\draw (D.north west) -- (D.south east);
		\draw[*->] let \p1 = (A.two), \p2 = (A.center) in (\x1,\y2) -- (B);
		\draw[*->] let \p1 = (B.two), \p2 = (B.center) in (\x1,\y2) -- (C);
		\draw[*->] let \p1 = (C.two), \p2 = (C.center) in (\x1,\y2) -- (D);
	\end{tikzpicture}
	\begin{tikzpicture}[
			list/.style={
					rectangle split,
					rectangle split parts=2,
					draw,
					rectangle split horizontal
				},
			dotarrow/.style={Circle-Stealth},
			start chain
		]
		\node[on chain,draw,inner sep=6pt] (D) {};
		\draw (D.north east) -- (D.south west);
		\draw (D.north west) -- (D.south east);
		\node[list,on chain] (A) {data};
		\node[list,on chain] (B) {data};
		\node[list,on chain] (C) {data};

		\draw[dotarrow] (A.two |- A.center) to[out=100,in=60,distance=2cm] (D);
		\draw[dotarrow] (B.two |- B.center) to[out=100,in=60,distance=2cm] (A.east);
		\draw[dotarrow] (C.two |- C.center) to[out=100,in=60,distance=2cm] (B.east);
	\end{tikzpicture}
	\caption{反转链表}
\end{figure}

\mybox{反转链表(递归)}

\begin{lstlisting}[language=Java]
public static Node reverseList(Node head) {
    if(head == null || head.next == null) {
        return head;
    }
    Node next = head.next;
    head.next = null;
    Node newHead = reverseList(next);
    next.next = head;
    return newHead;
}
\end{lstlisting}

\vspace{0.5cm}

\mybox{反转链表(迭代)}

\begin{lstlisting}[language=Java]
public static Node reverseListIterative(LinkedList list) {
    Node newHead = new Node(-1);
    Node head = list.getHead();
    while(head != null) {
        Node next = head.next;
        head.next = newHead.next;
        newHead.next = head;
        head = next;
    }
    return newHead.next;
}
\end{lstlisting}

\newpage

\section{跳表}

\subsection{跳表}

给定一个有序链表,如何根据给定元素的值进行高效率查找? \\

\begin{figure}[H]
	\centering
	\begin{tikzpicture}[list/.style={rectangle split, rectangle split parts=2,
					draw, rectangle split horizontal}, >=stealth, start chain]
		\node[list,on chain] (A) {2};
		\node[list,on chain] (B) {5};
		\node[list,on chain] (C) {9};
		\node[list,on chain] (D) {12};
		\node[list,on chain] (E) {20};
		\node[list,on chain] (F) {26};
		\node[on chain,draw,inner sep=6pt] (NULL) {};
		\draw (NULL.north east) -- (NULL.south west);
		\draw (NULL.north west) -- (NULL.south east);
		\draw[*->] let \p1 = (A.two), \p2 = (A.center) in (\x1,\y2) -- (B);
		\draw[*->] let \p1 = (B.two), \p2 = (B.center) in (\x1,\y2) -- (C);
		\draw[*->] let \p1 = (C.two), \p2 = (C.center) in (\x1,\y2) -- (D);
		\draw[*->] let \p1 = (D.two), \p2 = (D.center) in (\x1,\y2) -- (E);
		\draw[*->] let \p1 = (E.two), \p2 = (E.center) in (\x1,\y2) -- (F);
		\draw[*->] let \p1 = (F.two), \p2 = (F.center) in (\x1,\y2) -- (NULL);
	\end{tikzpicture}
	\caption{有序链表}
\end{figure}

是不是可以用二分查找,先找到中间元素? \\

链表和数组可不一样!数组能直接用下标快速定位,而链表只能从头结点一个一个元素向后找。对于链表,确实没办法像数组那样进行二分查找,但是可以在链表的基础上做一个小小的升级。 \\

如果给你一本书,要求你快速翻到书中的第5章,当然是首先查阅书的目录,根据目录提示,翻到第5章所对应的页码。 \\

其实一个链表也可以拥有自己的目录,或者说索引。 \\

\begin{figure}[H]
	\centering
	\begin{tikzpicture}[box/.style={on chain,draw,minimum width=1.5em,
					text=green!60!black,join=by abc},
			c/.style={on chain,circle,draw,inner sep=1ex},
			node distance=1.5em,abc/.style={stealth-}]
		\path[abc/.style={-stealth},start chain=going right,nodes={alias=1-\X}]
		foreach \X in {0,1,5,6,8,12,17,22,33,40,44} {\ifnum\X=0
					node[c]{}
					\else
					node[box]{\X}
				\fi};
		\begin{scope}[start branch=b0 going above]
			\chainin(1-0);
			\node[c,alias=2-0]{};
			\node[c,alias=3-0]{};
		\end{scope}
		\begin{scope}[start branch=b6 going above]
			\chainin(1-6);
			\node[box,alias=2-6]{6};
		\end{scope}
		\begin{scope}[start branch=b6 going above]
			\chainin(1-6);
			\node[box,alias=2-6]{6};
		\end{scope}
		\begin{scope}[start branch=b17 going above]
			\chainin(1-17);
			\node[box,alias=2-17]{17};
		\end{scope}
		\begin{scope}[start branch=b40 going above]
			\chainin(1-40);
			\node[box,alias=2-40]{40};
			\node[box,alias=3-40]{40};
		\end{scope}
		\path foreach \X in {1,2,3} {(\X-0) node[left=1em]{$h=\X$}};
		\path[every edge/.append style={-stealth}] (2-0) edge (2-6) edge (1-0)
		(3-0) edge (2-0) edge (3-40)
		(2-6) edge (2-17) (2-17) edge (2-40);
	\end{tikzpicture}
	\caption{跳表}
\end{figure}

在原始链表的基础上,增加了一个索引链表。原始链表的每三个结点,有一个结点在索引链表当中。当需要查找结点17的时候,不需要在原始链表中一个一个结点访问,而是首先访问索引链表。在索引链表找到结点17之后,顺着索引链表的结点向下,找到原始链表的结点17。 \\

这个过程,就像是先查阅了图书的目录,再翻到章节所对应的页码。由于索引链表的结点个数是原始链表的一半,查找结点所需的访问次数也相应减少了一半。 \\

这个过程,就像是先查阅了图书的目录,再翻到章节所对应的页码。由于索引链表的结点个数是原始链表的一半,查找结点所需的访问次数也相应减少了一半。基于原始链表的第1层索引,抽出了第2层更为稀疏的索引,结点数量是第1层索引的一半。这样的多层索引可以进一步提升查询效率。 \\

假设原始链表有n个结点,那么索引的层级就是$ logn - 1 $,在每一层的访问次数是常量,因此查找结点的平均时间复杂度是$ O(logn) $,这比起常规的线性查找效率要高得多。 \\

但相应的,这种基于链表的优化增加了额外的空间开销,各层索引的结点总数是$ {n \over 2} + {n \over 4} + {n \over 8} + {n \over 16} + \dots \approx n $。也就是说,优化之后的数据结构所占空间,是原来的2倍,这是典型的以空间换时间的做法。 \\

像这样基于链表改造的数据结构,有一个全新的名字,叫做跳表。

\subsection{跳表插入结点}

假设要插入的结点是10,首先按照跳表查找结点的方法,找到待插入结点的前置结点(仅小于待插入结点)。按照一般链表的插入方式,把结点10插入到结点8的下一个位置。 \\

\begin{figure}[H]
	\centering
	\begin{tikzpicture}[box/.style={on chain,draw,minimum width=1.5em,
					text=green!60!black,join=by abc},
			c/.style={on chain,circle,draw,inner sep=1ex},
			node distance=1.5em,abc/.style={stealth-}]
		\path[abc/.style={-stealth},start chain=going right,nodes={alias=1-\X}]
		foreach \X in {0,1,5,6,8,10,12,17,22,33,40} {\ifnum\X=0
					node[c]{}
					\else
					node[box]{\X}
				\fi};
		\begin{scope}[start branch=b0 going above]
			\chainin(1-0);
			\node[c,alias=2-0]{};
			\node[c,alias=3-0]{};
		\end{scope}
		\begin{scope}[start branch=b6 going above]
			\chainin(1-6);
			\node[box,alias=2-6]{6};
		\end{scope}
		\begin{scope}[start branch=b6 going above]
			\chainin(1-6);
			\node[box,alias=2-6]{6};
		\end{scope}
		\begin{scope}[start branch=b17 going above]
			\chainin(1-17);
			\node[box,alias=2-17]{17};
		\end{scope}
		\begin{scope}[start branch=b40 going above]
			\chainin(1-40);
			\node[box,alias=2-40]{40};
			\node[box,alias=3-40]{40};
		\end{scope}
		\path foreach \X in {1,2,3} {(\X-0) node[left=1em]{$h=\X$}};
		\path[every edge/.append style={-stealth}] (2-0) edge (2-6) edge (1-0)
		(3-0) edge (2-0) edge (3-40)
		(2-6) edge (2-17) (2-17) edge (2-40);
		\draw[-, very thick, red] (5.9,0.3) -- (6.6,0.3) -- (6.6,-0.3) -- (5.9,-0.3) -- (5.9,0.3);
	\end{tikzpicture}
	\caption{插入结点10}
\end{figure}

这样是不是插入工作就完成了呢?并不是。随着原始链表的新结点越来越多,索引会渐渐变得不够用了,因此索引结点也需要相应作出调整。 \\

调整索引的方法就是让新插入的结点随机晋升成为索引结点,晋升成功的几率是50\%。新结点在成功晋升之后,仍然有机会继续向上一层索引晋升。 \\

\begin{figure}[H]
	\centering
	\begin{tikzpicture}[box/.style={on chain,draw,minimum width=1.5em,
					text=green!60!black,join=by abc},
			c/.style={on chain,circle,draw,inner sep=1ex},
			node distance=1.5em,abc/.style={stealth-}]
		\path[abc/.style={-stealth},start chain=going right,nodes={alias=1-\X}]
		foreach \X in {0,1,5,6,8,10,12,17,22,33,40} {\ifnum\X=0
					node[c]{}
					\else
					node[box]{\X}
				\fi};
		\begin{scope}[start branch=b0 going above]
			\chainin(1-0);
			\node[c,alias=2-0]{};
			\node[c,alias=3-0]{};
		\end{scope}
		\begin{scope}[start branch=b6 going above]
			\chainin(1-6);
			\node[box,alias=2-6]{6};
		\end{scope}
		\begin{scope}[start branch=b6 going above]
			\chainin(1-6);
			\node[box,alias=2-6]{6};
		\end{scope}
		\begin{scope}[start branch=b10 going above]
			\chainin(1-10);
			\node[box,alias=2-10]{10};
			\node[box,alias=3-10]{10};
		\end{scope}
		\begin{scope}[start branch=b17 going above]
			\chainin(1-17);
			\node[box,alias=2-17]{17};
		\end{scope}
		\begin{scope}[start branch=b40 going above]
			\chainin(1-40);
			\node[box,alias=2-40]{40};
			\node[box,alias=3-40]{40};
		\end{scope}
		\path foreach \X in {1,2,3} {(\X-0) node[left=1em]{$h=\X$}};
		\path[every edge/.append style={-stealth}] (2-0) edge (2-6) edge (1-0)
		(3-0) edge (2-0) edge (3-10) (3-10) edge (3-40)
		(2-6) edge (2-10) (2-10) edge (2-17) (2-17) edge (2-40);
		\draw[-, very thick, red] (5.9,0.3) -- (6.6,0.3) -- (6.6,-0.3) -- (5.9,-0.3) -- (5.9,0.3);
		\draw[-, very thick, red] (5.9,1.45) -- (6.6,1.45) -- (6.6,0.9) -- (5.9,0.9) -- (5.9,1.45);
		\draw[-, very thick, red] (5.9,2.6) -- (6.6,2.6) -- (6.6,2.1) -- (5.9,2.1) -- (5.9,2.6);
	\end{tikzpicture}
	\caption{晋升结点}
\end{figure}

\subsection{跳表删除结点}

至于跳表删除结点的过程,则是相反的思路。 \\

假设要从跳表中删除结点10,首先按照跳表查找结点的方法,找到待删除的结点。按照一般链表的删除方式,把结点10从原始链表当中删除。这样是不是删除工作就完成了呢?并不是。还需要顺藤摸瓜,把索引当中的对应结点也一一删除。 \\

\begin{figure}[H]
	\centering
	\begin{tikzpicture}[box/.style={on chain,draw,minimum width=1.5em,
					text=green!60!black,join=by abc},
			c/.style={on chain,circle,draw,inner sep=1ex},
			node distance=1.5em,abc/.style={stealth-}]
		\path[abc/.style={-stealth},start chain=going right,nodes={alias=1-\X}]
		foreach \X in {0,1,5,6,8,10,12,17,22,33,40} {\ifnum\X=0
					node[c]{}
					\else
					node[box]{\X}
				\fi};
		\begin{scope}[start branch=b0 going above]
			\chainin(1-0);
			\node[c,alias=2-0]{};
			\node[c,alias=3-0]{};
		\end{scope}
		\begin{scope}[start branch=b6 going above]
			\chainin(1-6);
			\node[box,alias=2-6]{6};
		\end{scope}
		\begin{scope}[start branch=b6 going above]
			\chainin(1-6);
			\node[box,alias=2-6]{6};
		\end{scope}
		\begin{scope}[start branch=b10 going above]
			\chainin(1-10);
			\node[box,alias=2-10]{10};
			\node[box,alias=3-10]{10};
		\end{scope}
		\begin{scope}[start branch=b17 going above]
			\chainin(1-17);
			\node[box,alias=2-17]{17};
		\end{scope}
		\begin{scope}[start branch=b40 going above]
			\chainin(1-40);
			\node[box,alias=2-40]{40};
			\node[box,alias=3-40]{40};
		\end{scope}
		\path foreach \X in {1,2,3} {(\X-0) node[left=1em]{$h=\X$}};
		\path[every edge/.append style={-stealth}] (2-0) edge (2-6) edge (1-0)
		(3-0) edge (2-0) edge (3-10) (3-10) edge (3-40)
		(2-6) edge (2-10) (2-10) edge (2-17) (2-17) edge (2-40);
		\draw[-, very thick, red] (5.7,0.5) -- (6.8,-0.5);
		\draw[-, very thick, red] (5.7,1.65) -- (6.8,0.7);
		\draw[-, very thick, red] (5.7,2.8) -- (6.8,1.9);
	\end{tikzpicture}
	\caption{删除结点}
\end{figure}

在实际的程序中,跳表采用的是双向链表,无论前后结点还是上下结点,都各有两个指针相互指向彼此。并且跳表的每一层首尾各有一个空结点,左侧的空节点是负无穷大,右侧的空节点是正无穷大。 \\

\begin{figure}[H]
	\centering
	\begin{tikzpicture}
		\SetGraphUnit{1.5}
		\SetVertexNormal[Shape    = rectangle,MinSize=.8 cm]
		%%%%%%%%%%%%%%%%% vertices %%%%%%%%%%%%%%%%%%% 
		%  name of node x;y  x row y column

		\Vertex[L=$-\infty$] {0;0}

		\foreach \num [count=\n from 0] in  {1,2,3}
			{\NO[L=$-\infty$](0;\n){0;\num}}
		% No = north (initial node) {new node}  L=label 
		\foreach \num/\label [count=\n from 0] in
			{1/11,2/15,3/17,4/28,5/31,6/55,7/56,8/61,9/+\infty}
			{\EA[L=$\label$](\n;0){\num;0}}
		\foreach \num [count=\n from 0] in  {1,2,3}{%
				\NO[L=$+\infty$](9;\n){9;\num}}

		\foreach \num [count=\n from 0] in  {1,2,3} {%
				\NO[L=$31$](5;\n){5;\num}}

		\foreach \no/\label  in  {1/11,2/15,6/55,7/56} {%
				\NO[L=$\label$](\no;0){\no;1} }

		%%%%%%%%%%%%%%%%% edges %%%%%%%%%%%%%%%%%%%%%%%
		\foreach \num [count=\n from 1] in  {0,...,8}  {\Edge(\num;0)(\n;0)}

		\foreach \num [count=\n from 1] in  {0,...,2}
			{\Edge(0;\num)(0;\n)  \Edge(5;\num)(5;\n)  \Edge(9;\num)(9;\n)}

		\foreach \num  in  {1,2,6,7} {\Edge(\num;0)(\num;1)}

		\foreach \num [remember=\num as \lastnum (initially 0)] in  {5,9}
			{\Edge(\lastnum;3)(\num;3) }

		\foreach \num [remember=\num as \lastnum (initially 0)] in  {5,9}
			{\Edge(\lastnum;2)(\num;2)}

		\foreach \num [remember=\num as \lastnum (initially 0)] in  {1,2,5,6,7,9}
			{\Edge(\lastnum;1)(\num;1)}
	\end{tikzpicture}
	\caption{跳表}
\end{figure}

\newpage