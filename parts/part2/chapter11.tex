\chapter{计算复杂性理论}

\section{时间复杂度}

\subsection{输入规模}

算法的时间复杂度是针对指定基本运算,计算算法所做的运算次数。其中基本运算指的是比较、加法、乘法、置指针、交换等操作。\\

算法基本运算可以表示为跟输入规模相关的函数。常见的输入规模有数组的元素个数、调度问题的任务个数、图的顶点数和边数等。对于相同输入规模的不同实例,算法的基本运算次数有可能会不一样。\\

对于排序算法,输入规模为数组的元素个数,基本运算为元素之间的比较。\\

对于整数乘法,$ m $位整数与$ n $位整数相乘需要做$ m \times n $次乘法。\\

对于矩阵乘法,$ i \times j $矩阵与$ j \times k $矩阵相乘需要做$ i \times j \times k $次乘法。\\

\subsection{时间复杂度}

最好情况时间复杂度是指算法在最理想情况下的时间复杂度。例如在查找算法中,目标元素刚好在数组的第一个位置,那么只需要一次就能找到,时间复杂度是常量阶$ O(1) $。\\

最坏情况时间复杂度$ W(n) $是指算法在最坏情况下执行的时间复杂度。例如目标元素在数组最后一个位置或者不在数组中,那么需要遍历完整个数组才能得出结果,时间复杂度为$ O(n) $。\\

平均情况时间复杂度$ A(n) $是指用算法在所有情况下执行的次数的加权平均值表示,也就是算法在求解这类问题所需要的平均时间。\\

假设$ S $是规模n为实例集,实例的$ i \in S $概率是$ p_i $,算法对实例$ i $执行的基本运算次数是$ t_i $:

\vspace{-0.5cm}

$$
	A(n) = \sum_{i \in S} p_i t_i
$$

例如,利用顺序查找算法在一个长度为n的数组中查找元素x。假设x在数组中的概率是$ p $(x不在数组中的概率为$ 1 - p $),且每个位置概率相等:

\vspace{-1cm}

\begin{align*}
	A(n) & = \sum_{i=1}^{n} i {p \over n} + (1 - p)n \\
	     & = {p(n+1) \over 2} + (1 - p)n             \\
	\text{当}p = {1 \over 2},                        \\
	     & = {n+1 \over 4} + {n \over 2}             \\
	     & = {3 \over 4}n
\end{align*}

\vspace{0.5cm}

\subsection{大O符号}

设$ f $和$ g $是定义域为自然数集$ \mathbb{N} $上的函数,若存在正数$ c $和$ n_0 $,使得一切$ n \ge n_0 $满足

\vspace{-0.5cm}

$$
	0 \le f(n) \le cg(n)
$$

则称$ f(n) $的渐进上界是$ g(n) $,即$ f(n) $的阶不高于$ g(n) $的阶,记作:

\vspace{-0.5cm}

$$
	f(n) = O(g(n))
$$

\vspace{0.5cm}

\mybox{大O符号}

\vspace{-1cm}

\begin{align*}
	f(n) & = n^2 + n \\
	f(n) & = O(n^2)  \\
	f(n) & = O(n^3)
\end{align*}

\vspace{0.5cm}

\subsection{大$ \Omega $符号}

设$ f $和$ g $是定义域为自然数集$ \mathbb{N} $上的函数,若存在正数$ c $和$ n_0 $,使得一切$ n \ge n_0 $满足

\vspace{-0.5cm}

$$
	0 \le cg(n) \le f(n)
$$

则称$ f(n) $的渐进下界是$ g(n) $,即$ f(n) $的阶不低于$ g(n) $的阶,记作:

\vspace{-0.5cm}

$$
	f(n) = \Omega(g(n))
$$

\vspace{0.5cm}

\mybox{大$ \Omega $符号}

\vspace{-1cm}

\begin{align*}
	f(n) & = n^2 + n      \\
	f(n) & = \Omega(n^2)  \\
	f(n) & = \Omega(100n)
\end{align*}

\vspace{0.5cm}

\subsection{小o符号}

设$ f $和$ g $是定义域为自然数集$ \mathbb{N} $上的函数,若对于任意正数$ c $都存在$ n_0 $,使得一切$ n \ge n_0 $满足

\vspace{-0.5cm}

$$
	0 \le f(n) < cg(n)
$$

则称$ f(n) $的阶低于$ g(n) $的阶,记作:

\vspace{-0.5cm}

$$
	f(n) = o(g(n))
$$

\vspace{0.5cm}

\mybox{小o符号}

\vspace{-1cm}

\begin{align*}
	f(n) & = n^2 + n \\
	f(n) & = o(n^3)
\end{align*}

\vspace{0.5cm}

\subsection{小$ \omega $符号}

设$ f $和$ g $是定义域为自然数集$ \mathbb{N} $上的函数,若对于任意正数$ c $都存在$ n_0 $,使得一切$ n \ge n_0 $满足

\vspace{-0.5cm}

$$
	0 \le cg(n) < f(n)
$$

则称$ f(n) $的阶高于$ g(n) $的阶,记作:

\vspace{-0.5cm}

$$
	f(n) = \omega(g(n))
$$

\vspace{0.5cm}

\mybox{小$ \omega $符号}

\vspace{-1cm}

\begin{align*}
	f(n) & = n^2 + n   \\
	f(n) & = \omega(n)
\end{align*}

\vspace{0.5cm}

\subsection{$ \Theta $符号}

若$ f(n) = O(g(n)) $且$ f(n) = \Omega(g(n)) $,则称$ f(n) $的阶与$ g(n) $的阶相等,记作:

\vspace{-0.5cm}

$$
	f(n) = \Theta(g(n))
$$

\vspace{0.5cm}

\mybox{$ \Theta $符号}

\vspace{-1cm}

\begin{align*}
	f(n) & = n^2 + n      \\
	g(n) & = 100n^2       \\
	f(n) & = \Theta(g(n))
\end{align*}

\vspace{0.5cm}

\subsection{定理(Theorem)}

\begin{enumerate}
	\item 如果$ \lim\limits_{n \rightarrow \infty} {f(n) \over g(n)} $存在,并且等于某个常数$ c > 0 $,那么$ f(n) = \Theta(g(n)) $。

	\item 如果$ \lim\limits_{n \rightarrow \infty} {f(n) \over g(n)} = 0 $,那么$ f(n) = o(g(n)) $。

	\item 如果$ \lim\limits_{n \rightarrow \infty} {f(n) \over g(n)} = +\infty $,那么$ f(n) = \omega(g(n)) $。

	\item 多项式函数的阶低于指数函数的阶,即$ n^d = o(r^n),\ r > 1,\ d > 0 $。

	\item 对数函数的阶低于幂函数的阶,即$ ln(n) = o(n^d),\ d > 0 $。

	\item 函数的阶之间的关系具有传递性:
	      \begin{itemize}
		      \item 如果$ f = O(g),\ g = O(h) $,那么$ f = O(h) $。

		      \item 如果$ f = \Omega(g),\ g = \Omega(h) $,那么$ f = \Omega(h) $。

		      \item 如果$ f = \Theta(g),\ g = \Theta(h) $,那么$ f = \Theta(h) $。
	      \end{itemize}
\end{enumerate}

\vspace{0.5cm}

\mybox{证明}\\

$ f(n) = {1 \over 2}n^2 - 3n $,证明$ f(n) = \Theta(n^2) $

\vspace{-0.5cm}

\begin{align*}
	 & \lim\limits_{n \rightarrow \infty} {f(n) \over n^2}                  \\
	 & = \lim\limits_{n \rightarrow \infty} {{1 \over 2}n^2 - 3n \over n^2} \\
	 & = {1 \over 2}
\end{align*}

\vspace{0.5cm}

\mybox{证明}\\

证明多项式函数的阶低于指数函数的阶。

\vspace{-0.5cm}

\begin{align*}
	 & \lim\limits_{n \rightarrow \infty} {n^d \over r^n}                    \\
	 & = \lim\limits_{n \rightarrow \infty} {dn^{d-1} \over r^nln(r)}        \\
	 & = \lim\limits_{n \rightarrow \infty} {d(d-1)n^{d-2} \over r^nln(r)^2} \\
	 & = \dots                                                               \\
	 & = \lim\limits_{n \rightarrow \infty} {d! \over r^nln(r)^d}            \\
	 & = 0
\end{align*}

\vspace{0.5cm}

\mybox{证明}\\

证明对数函数的阶低于幂函数的阶。

\vspace{-0.5cm}

\begin{align*}
	 & \lim\limits_{n \rightarrow \infty} {ln(n) \over n^d}              \\
	 & = \lim\limits_{n \rightarrow \infty} {{1 \over n} \over dn^{d-1}} \\
	 & = \lim\limits_{n \rightarrow \infty} {1 \over dn^d}               \\
	 & = 0
\end{align*}

\vspace{0.5cm}

\mybox{排序}

\vspace{-1cm}

\begin{align*}
	 & f(n) = (n^2 + n) / 2 \\
	 & g(n) = 10n           \\
	 & h(n) = 1.5^n         \\
	 & t(n) = n^{1 \over 2}
\end{align*}

按照阶从高到低排序。

\vspace{-1cm}

\begin{align*}
	 & h(n) = \omega(f(n))       \\
	 & f(n) = \omega(g(n))       \\
	 & g(n) = \omega(t(n))       \\
	 & h(n) < f(n) < g(n) < t(n)
\end{align*}

\newpage

\section{均摊时间复杂度}

\subsection{均摊时间复杂度(Amortized Time Complexity)}

均摊时间复杂度也称摊还分析或分摊分析,均摊复杂度是一个更加高级的概念,它是一种特殊的情况,应用的场景也更加特殊和有限。

\vspace{-0.5cm}

\begin{lstlisting}[language=Java]
void insert(int val) {
    if(cnt == arr.length) {
        int sum = 0;
        for(int i = 0; i < arr.length; i++) {
            sum += arr[i];
        }
        arr[0] = sum;
        cnt = 1;
    }
    arr[cnt++] = val;
}
\end{lstlisting}

这段代码实现了一个往数组中插入数据的功能。当数组元素满时,就遍历数组求和,将元素和保存到数组的第0个位置,并清空数组,然后再将新的数据插入。但如果数组一开始就有空闲空间,则直接将数据插入数组。\\

最理想的情况下,数组中有空闲空间,最好情况时间复杂度为$ O(1) $;最坏的情况下,数组中没有空闲空间了,需要先做一次遍历求和,然后再将数据插入,所以最坏情况时间复杂度为$ O(n) $。\\

假设数组的长度是n,根据数据插入的位置的不同,可以分为n种情况,每种情况的时间复杂度是$ O(1) $。除此之外,还有一种情况,就是在数组没有空闲空间时插入一个数据,这个时候的时间复杂度是$ O(n) $。这n + 1种情况发生的概率一样,都是$ 1 \over n+1 $。\\

根据加权平均的计算方法,求得的平均时间复杂度:

\vspace{-0.5cm}

\begin{align*}
	 & 1 \times {1 \over n+1} + 1 \times {1 \over n+1} + \dots + 1 \times {1 \over n+1} + n \times {1 \over n+1} \\
	 & = {2n \over n+1}                                                                                          \\
	 & = O(1)
\end{align*}

对一个数据结构进行一组连续操作中,大部分情况下时间复杂度都很低,只有个别情况下时间复杂度比较高,而且这些操作之间存在前后连贯的时序关系,这个时候就可以将这一组操作放在一块分析,看是否能将较高时间复杂度那次操作的耗时,平摊到其它那些时间复杂度比较低的操作上。而且,在能够应用均摊时间复杂度分析的场合,一般均摊时间复杂度就等于最好情况时间复杂度。

\newpage

\section{递推方程}

\subsection{递推(Recurrence)}

如果数列$ \{a_n\} $的第n项与它前一项的关系可以用一个公式来表示,那么这个公式就叫做这个数列的递推方程。\\

算术级数的递推关系:

\vspace{-1cm}

\begin{align*}
	a_0 & = a           \\
	a_n & = a_{n-1} + d
\end{align*}

几何级数的递推关系:

\vspace{-1cm}

\begin{align*}
	a_0 & = a                \\
	a_n & = a_{n-1} \times r
\end{align*}

\vspace{0.5cm}

\subsection{斐波那契数列(Fibonacci Sequence)}

斐波那契数列$ f_0,\ f_1,\ f_2,\ \dots $的递推公式为:

\vspace{-0.5cm}

\begin{align*}
	f(n) = \begin{cases}
		1               & n = 1  \\
		1               & n = 2  \\
		f(n-1) + f(n-2) & n >= 3
	\end{cases}
\end{align*}

斐波那契数列的通项公式为:

$$
	f_n = {1 \over \sqrt{5}} \left({1 + \sqrt{5} \over 2} \right)^{n+1} - {1 \over \sqrt{5}} \left({1 - \sqrt{5} \over 2} \right)^{n+1}
$$

\vspace{0.5cm}

\mybox{斐波那契数列(递归)}

\begin{lstlisting}[language=C]
int fibonacci(int n) {
    if(n == 1 || n == 2) {
        return 1;
    }
    return fibonacci(n-2) + fibonacci(n-1);
}
\end{lstlisting}

\begin{figure}[H]
	\centering
	\begin{tikzpicture}[
			level distance=2.4cm,
			level 1/.style={sibling distance=6cm},
			level 2/.style={sibling distance=3cm},
			level 3/.style={sibling distance=2cm}
		]
		\node {$ f(5) $}
		child {
				node {$ f(3) $}
				child {node {$ f(1) $}}
				child {
						node {$ f(2) $}
						child {node {$ f(0) $}}
						child {node {$ f(1) $}}
					}
			}
		child {
				node {$ f(4) $}
				child {
						node {$ f(2) $}
						child {node {$ f(0) $}}
						child {node {$ f(1) $}}
					}
				child {
						node {$ f(3) $}
						child {node {$ f(1) $}}
						child {
								node {$ f(2) $}
								child {node {$ f(0) $}}
								child {node {$ f(1) $}}
							}
					}
			};
	\end{tikzpicture}
	\caption{递归树}
\end{figure}

\vspace{0.5cm}

\mybox{斐波那契数列(迭代)}

\begin{lstlisting}[language=C]
int fibonacci(int n) {
    int f[n];
    f[0] = f[1] = 1;
    for(int i = 2; i < n; i++) {
        f[i] = f[i-2] + f[i-1];
    }
    return f[n-1];
}
\end{lstlisting}

\vspace{0.5cm}

\subsection{汉诺塔(Hanoi Tower)}

有三根柱子A、B、C,A柱子上从下到上套有n个圆盘,要求将A柱子上的圆盘移动到C柱子上。每次只能移动一个圆盘,且大圆盘始终不能叠在小圆盘上面。\\

\begin{figure}[H]
	\centering
	\begin{tikzpicture}[scale=0.5]
		\draw[-, gray] (0,0) -- (24,0);
		\draw[-, very thick] (4,0) -- (4,7);
		\draw[-, very thick] (12,0) -- (12,7);
		\draw[-, very thick] (20,0) -- (20,7);

		\draw[fill=red] (2.5,3) rectangle (5.5,4);
		\draw[fill=green] (2,2) rectangle (6,3);
		\draw[fill=blue] (1.5,1) rectangle (6.5,2);
		\draw[fill=yellow] (1,0) rectangle (7,1);
	\end{tikzpicture}
	\caption{汉诺塔}
\end{figure}

递归算法求解汉诺塔问题:

\begin{enumerate}
	\item 将前n - 1个圆盘从A柱借助于C柱搬到B柱。
	\item 将最后一个圆盘直接从A柱搬到C柱。
	\item 将n - 1个圆盘从B柱借助于A柱搬到C柱。
\end{enumerate}

\vspace{0.5cm}

\mybox{汉诺塔}

\begin{lstlisting}[language=Python]
def hanoi(n, A, B, C):
    if n == 1
        move(1, A, C)
    else
        hanoi(n-1, A, C, B)
        move(n, A, C)
        hanoi(n-1, B, A, C)
\end{lstlisting}

假设每次移动花费1秒,解决一个64层的汉诺塔问题大约需要5800亿年。\\

汉诺塔递归算法的递推公式:

\vspace{-0.5cm}

\begin{align*}
	T(n) = \begin{cases}
		1             & n = 1 \\
		2T(n - 1) + 1 & n > 1
	\end{cases}
\end{align*}

利用迭代法,不断用递推方程的右部替换左部,直到出现初值停止迭代。

\vspace{-1cm}

\begin{align*}
	T(n) & = 2 * T(n - 1) + 1                     \\
	     & = 2 * [ 2 * T(n - 2) + 1] + 1          \\
	     & = 2 * [2 * [2 * T(n - 3) + 1] + 1] + 1 \\
	     & = \dots                                \\
	     & = 2^k * T(n - k) + 2^k - 1             \\
	\\
	     & \because n - k = 1                     \\
	     & \therefore k = n - 1                   \\
	\\
	T(n) & = 2^{n-1} * T(1) + 2^{n-1} - 1         \\
	     & = 2^{n-1} + 2^{n-1} - 1                \\
	     & = 2^n - 1                              \\
	     & = O(2^n)
\end{align*}

\begin{figure}[H]
	\centering
	\includegraphics[]{img/Chapter11/11-3/1.png}
\end{figure}

\vspace{0.5cm}

\subsection{插入排序}

插入排序的递推公式:

\vspace{-0.5cm}

\begin{align*}
	W(n) = \begin{cases}
		0                & n = 1 \\
		W(n - 1) + n - 1 & n > 1
	\end{cases}
\end{align*}

当数组中只有1个元素时,无须进行任何交换。当数组中的元素大于1时,最坏情况下需要进行n - 1次交换。

\vspace{-1cm}

\begin{align*}
	W(n) & = W(n - 1) + n - 1                             \\
	     & = [W(n - 2) + n - 2] + n - 1                   \\
	     & = [[W(n - 3) + n - 3] + n - 2] + n - 1         \\
	     & = \dots                                        \\
	     & = W(n - k) + 1 + 2 + \dots + (n - 2) + (n - 1) \\
	     & = W(1) + n(n - 1) / 2                          \\
	     & = n(n - 1) / 2                                 \\
	     & = O(n^2)
\end{align*}

\vspace{0.5cm}

\subsection{归并排序}

利用换元法,可以将对n个递推式转换成对其它变元k的递推式,通过对k直接迭代,最后将关于k的函数转换成关于n的函数。\\

假设$ n = 2^k $,二分归并排序的递推公式:

\vspace{-0.5cm}

\begin{align*}
	W(n) = \begin{cases}
		0               & n = 1 \\
		2W(n/2) + n - 1 & n > 1
	\end{cases}
\end{align*}

通过换元:

\vspace{-0.5cm}

\begin{align*}
	W(2^k) = \begin{cases}
		0                     & n = 1 \\
		2W(2^{k-1}) + 2^k - 1 & n > 1
	\end{cases}
\end{align*}

\vspace{-0.5cm}

\begin{align*}
	W(2^k) & = 2W(2^{k-1}) + 2^k - 1                                   \\
	       & = 2[2W(2^{k-2}) + 2^{k-1} - 1] + 2^k - 1                  \\
	       & = 2[2[2W(2^{k-3}) + 2^{k-2} - 1] + 2^{k-1} - 1] + 2^k - 1 \\
	       & = \dots                                                   \\
	       & = 2^k W(1) + k2^k - (2^{k-1} + 2^{k-2} + \dots + 2 + 1)   \\
	       & = k2^k - 2^k + 1                                          \\
	       & = nlogn - n + 1                                           \\
	       & = O(nlogn)
\end{align*}

\newpage

\section{Master Theorem}

\subsection{主定理(Master Theorem)}

Master Theorem是算法分析中非常重要的定理。对一个规模为n的问题通过分治,得到a个规模为$ n \over b $的子问题,分解和合并子问题的时间是$ f(n) $。

\vspace{-0.5cm}

$$
	T(n) = aT\left( {n \over b} \right) + f(n)
$$

\begin{enumerate}
	\item 若$ f(n) = O \left( n^{log_b{a-\epsilon}} \right),\ \epsilon > 0 $,那么$ T(n) = \Theta \left( n^{log_b{a}} \right) $。

	\item 若$ f(n) = \Theta \left( n^{log_b{a}} \right) $,那么$ T(n) = \Theta \left( n^{log_b{a}} * logn \right) $。

	\item 若$ f(n) = \Omega \left( n^{log_b{a+\epsilon}} \right),\ \epsilon > 0 $,且对于某个常数$ c < 1 $和充分大的$ n $满足$ af\left( {n \over b} \right) \le cf(n) $,那么$ T(n) = \Theta(f(n)) $。
\end{enumerate}

\vspace{0.5cm}

\mybox{$ T(n) = 9T(n/3) + n $}

\vspace{-1cm}

\begin{align*}
	 & a = 9,\ b = 3,\ f(n) = n                                             \\
	 & n^{log_3{9}} = n^2                                                   \\
	 & \text{取}\epsilon = 1\text{,}f(n) = O \left( n^{log_3{9-1}} \right) \\
	 & \text{满足Master Theorem 1,}T(n) = \Theta(n^2)
\end{align*}

\vspace{0.5cm}

\mybox{$ T(n) = T(2n / 3) + 1 $}

\vspace{-1cm}

\begin{align*}
	 & a = 1,\ b = 3/2,\ f(n) = 1                       \\
	 & n^{log_{3/2}{1}} = n^0 = 1                       \\
	 & \text{满足Master Theorem 2,}T(n) = \Theta(logn)
\end{align*}

\vspace{0.5cm}

\mybox{$ T(n) = 3T(n/4) + nlogn $}

\vspace{-1cm}

\begin{align*}
	 & a = 3,\ b = 4,\ f(n) = nlogn                                                                                                                                          \\
	 & \text{取}\epsilon = 0.2\text{,}f(n) = \Omega \left( n^{log_4{3+\epsilon}} \right) \approx \Omega \left( n^{0.793+\epsilon} \right) = \Omega \left( n^{0.993} \right) \\
	 & \text{要使}af\left( {n \over b} \right) \le cf(n)\text{成立:}3(n/4)log(n/4) \le cnlogn                                                                               \\
	 & \text{当}3/4 \le c < 1\text{,不等式对所有充分大的}n\text{成立}                                                                                                       \\
	 & \text{满足Master Theorem 3,}T(n) = \Theta(nlogn)
\end{align*}

\newpage

\section{旅行商问题}

\subsection{旅行商问题(TSP, Traveling Salesman Problem)}

小灰最近在工作中遇到了一个棘手的问题。公司正在开发一个物流项目,其中一个需求是为快递员自动规划送快递的路线。\\

有一个快递员,要分别给三家顾客送快递,他自己到达每个顾客家的路程各不相同,每个顾客之间的路程也各不相同。那么想要把快递依次送达这三家,并最终回到起点,哪一条路线所走的总距离是最短的呢?

\begin{figure}[H]
	\centering
	\begin{tikzpicture}
		\begin{scope}[every node/.style={circle,thick,draw}]
			\node (A) at (0,0) {出发};
			\node (B) at (-4,3) {客户};
			\node (C) at (1,7) {客户};
			\node (D) at (3,3) {客户};
		\end{scope}

		\begin{scope}[>={Stealth[black]},
			every node/.style={fill=white,circle},
			every edge/.style={draw=black,very thick}]
			\path [-] (A) edge node {2} (B);
			\path [-] (A) edge node {5} (C);
			\path [-] (A) edge node {6} (D);
			\path [-] (B) edge node {3} (C);
			\path [-] (B) edge node {4} (D);
			\path [-] (C) edge node {7} (D);
		\end{scope}
	\end{tikzpicture}
	\caption{快递客户路线}
\end{figure}

为了寻求最优路线,小灰研究了好久,可惜还是没有找到一个高效率的解决方案。\\

不只是小灰,当前的计算机科学家们也没有找到一个行之有效的优化方案,这是典型的旅行商问题。

\begin{figure}[H]
	\centering
	\includegraphics[scale=0.4]{img/Chapter11/11-5/1.png}
\end{figure}

有一个商品推销员,要去若干个城市推销商品。该推销员从一个城市出发,需要经过所有城市后,回到出发地。每个城市之间都有道路连通,且距离各不相同,推销员应该如何选择路线,使得总行程最短呢?\\

这个问题看起来很简单,却很难找到一个真正高效的求解算法。其中最容易想到的,是使用穷举法把所有可能的路线穷举出来,计算出每一条路线的总行程。\\

通过排列组合,从所有路线中找出总行程最短的路线。显然,这个方法的时间复杂度是$ O(n!) $,随着城市数量的增长,花费的运算时间简直不可想象!\\

后来,人们想出了许多相对优化的解决方案,比如动态规划法和分枝定界法等。但是,这些算法的时间复杂度仍然是指数级的,并没有让性能问题得到根本的解决。\\

像这样的问题有很多,旅行商问题仅仅是其中的一例。对于这类问题,我们统称为NP问题。

\newpage

\section{NP=P?}

\subsection{P和NP}

算法的设计与分析在计算机科学领域有着重要的应用背景。1966 $ \sim $ 2005年期间,Turing奖获奖50人,其中10人以算法设计,7人以计算理论、自动机和复杂性研究领域的杰出贡献获奖。计算复杂性理论中的P = NP?问题是世界七大数学难题之一。

\begin{table}[H]
	\centering
	\setlength{\tabcolsep}{5mm}{
		\begin{tabular}{|c|c|}
			\hline
			\textbf{算法} & \textbf{时间复杂度} \\
			\hline
			二分查找      & $ O(logn) $         \\
			\hline
			归并排序      & $ O(nlogn) $        \\
			\hline
			Floyd最短路径 & $ O(n^3) $          \\
			\hline
		\end{tabular}
	}
	\caption{常见算法时间复杂度}
\end{table}

尽管这些算法的运行时间有数量级上的差别,但是它们的时间复杂度都可以用$ O(n^k) $来表示,其中k是一个常数。因此,这些算法都是多项式时间算法,能用多项式时间算法解决的问题被称为P问题(Polynomial)。\\

人们常说,能用钱解决的问题都不是问题,在计算机科学家眼中,能用多项式时间解决的问题都不是问题。\\

然而,世间还存在许多变态的问题,是无法(至少是暂时无法)在多项式时间内解决的,比如一些算法的时间复杂度是$ O(2^n) $,甚至$ O(n!) $。随着问题规模$ n $的增长,计算量的增长速度是非常恐怖的。这类问题被称为NP问题(Non-deterministic Polynomial),意思是“不确定是否能用多项式时间解决”。\\

有些科学家认为,所有的NP问题终究都可以在多项式时间内解决,只是我们暂时还没有找到方法;也有些科学家认为,某些NP问题永远无法在多项式时间内解决。这个业界争论用NP=P?这个公式来表达。

想不到计算机领域还有这么多难以解决的问题!\\

这还不算完,在所有的NP问题当中,还存在着一些大BOSS,被称为NPC问题。\\

\subsection{规约与NPC}

这里所说的NPC问题可不是游戏当中的NPC,要想理解NPC问题,需要先了解归约的概念。\\

归约(reduction)可以简单理解成问题之间的转化。例如问题是一个一元一次方程的求解问题$ Q: 3x + 6 = 12 $,这个问题可以转化成一个一元二次问题$ Q': 0x^2 + 3x + 6 = 12 $。\\

显然,问题$ Q $并不比问题$ Q' $更难解决,只要有办法解决$ Q' $,就一定能够解决$ Q $。对于这种情况,我们可以说问题$ Q $归约于问题$ Q' $。\\

同时,这种归约可以逐级传递,比如问题A归约于问题B,问题B归约于问题C,问题C归约于问题D,那么可以说问题A归约于问题D。\\

在NP问题之间,也可以存在归约关系。我们把众多的NP问题层层归约,必定会得到一个或多个终极问题,这些归约的终点就是所谓的NPC问题(NP-Complete)。旅行商问题被科学家证明属于NPC问题。\\

就数量上而言,NP问题远比P问题要多,而NP之中的NPC问题也仅占极少数。

\begin{figure}[H]
	\centering
	\begin{tikzpicture}
		\draw (0,0) ellipse (4 and 2) node[yshift=1cm]{NP问题};
		\draw (-1.75,-0.5) ellipse (1.5 and 0.75) node{P问题};
		\draw (1.75,-0.5) ellipse (1.5 and 0.75) node{NPC问题};
	\end{tikzpicture}
	\caption{P、NP与NPC的关系}
\end{figure}

俗话说擒贼先擒王,只要有朝一日,我们能够找到NPC问题的多项式时间算法,就能够解决掉所有的NP问题!但遗憾的是,至今还没有人能够找到可行的方法,很多人认为这些问题是无解的。\\

\subsection{旅行商问题}

回到最初的快递路线规划问题,既然是工程问题,我们与其钻牛角尖寻求最优解,不如用小得多的代价寻求次优解。\\

最简单的办法是使用贪心算法,先选择距离起点最近的地点A,再选择距离A最近的地点B,以此类推,每一步都保证局部最优。\\

这样规划出的路线未必是全局最优,但平均情况下也不会比最优方案差多少。\\

除此之外,人们还找到了许多近似的解决方案,比如遗传算法、蚁群算法等。Google有一款开源工具OR-TOOL,其中也包含路线规划的实现。

\newpage