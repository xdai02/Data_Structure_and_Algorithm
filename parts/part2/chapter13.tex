\chapter{排序算法}

\section{希尔排序}

\subsection{希尔排序(Shell Sort)}

希尔排序本质上是直接插入排序的升级版。对于插入排序而言,在大多数元素已经有序的情况下,工作量会比较小。这个结论很明显,如果一个数组大部分元素都有序,那么数组中的元素自然不需要频繁地进行比较和交换。 \\

如何能够让待排序的数组中大部分元素有序呢?需要对原始数组进行预处理,使得原始数组的大部分元素变得有序。采用分组的方法,可以将数组进行一定程度地粗略调整。 \\

例如一个有8个数字组成的无序序列\{5, 8, 6, 3, 9, 2, 1, 7\},进行升序排序。让元素两两一组,同组两个元素之间的跨度为数组总长度的一半。 \\

接着让每组元素进行独立排序,排序方式使用直接插入排序即可。由于每一组的元素数量很少,所以插入排序的工作量很少。这样一来,仅仅经过几次简单的交换,数组整体的有序程序得到了显著提高,使得后续再进行直接插入排序的工作量大大减少。 \\

但是这样还不算完,还可以进一步缩小分组跨度,重复上述工作。