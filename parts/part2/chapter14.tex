\chapter{贪心算法}

\section{贪心算法}

\subsection{贪心算法(Greedy Algorithm)}

贪心算法,又称贪婪算法,是指在对问题求解时,总是做出在当前看来是最好的选择,也就是说不从整体最优上加以考虑。算法得到的是在某种意义上的局部最优解。贪心算法不是对所有问题都能得到整体最优解,关键在于贪心策略的选择。 \\

贪心算法没有固定的算法框架,算法设计的关键是贪心策略的选择。必须注意的是,贪心算法不是对所有问题都能得到整体最优解,选择的贪心策略必须具备无后效性,即某个状态以后的过程不会影响以前的状态,只与当前状态有关。 \\

在利用贪心算法求解问题之前,必须需要清楚什么样的问题适合用贪心算法。一般而言,能够利用贪心算法求解的问题都会具备以下两点性质:

\begin{enumerate}
    \item 贪心选择:当某一个问题的整体最优解可通过一系列局部最优解的选择达到,并且每次做出的选择可以依赖以前做出的选择,但不需要依赖后面需要做出的选择。

    \item 最优子结构:如果一个问题的最优解包含其子问题的最优解,则此问题具备最优子结构的性质。
\end{enumerate}

贪心算法的基本思路分为:

\begin{enumerate}
    \item 建立数学模型描述问题。
    \item 把求解的问题分成若干个子问题。
    \item 对每个子问题求解,得到子问题的局部最优解。
    \item 把子问题的局部最优解合成为原问题的解。
\end{enumerate}

