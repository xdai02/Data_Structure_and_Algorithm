\chapter{动态规划}

\section{动态规划}

\subsection{动态规划(Dynamic Programming)}

动态规划在数学上属于运筹学的分支,是求解决策过程最优化的数学方法,同时也是计算机科学与技术领域中一种常见的算法思想。 \\

动态规划算法的基本思想与分治法类似,也是将带求解的问题分解为若干个子问题,按顺序求解子问题。前一子问题的解,为后一子问题的求解提供了有用的信息。 \\

在求解任一子问题时,列出各种可能的局部解,通过决策保留那些有可能达到最优的局部解,丢弃其它局部解。依次解决各子问题,最后一个子问题就是初始问题的解。 \\

动态规划的本质是对问题状态的定义和状态转移方程的定义。动态规划通过拆分问题,定义问题状态和状态之间的关系,使得问题能够以递推的方式去解决。因此在一个典型的动态规划问题上,需要定义问题状态以及写出状态转移方程,这样对于问题的解答就会一目了然。

\subsection{爬楼梯}

有一座高度是10级台阶的楼梯,从下往上走,每跨一步只能向上1级或者2级台阶,要求求出一共有多少种走法。 \\

比如,每次走1级台阶,一共走10步,这是其中一种走法,可以简写成[1, 1, 1, 1, 1, 1, 1, 1, 1, 1]。再比如,每次走2级台阶,一共走5步,这是另一种走法,可以简写成[2, 2, 2, 2, 2]。当然,除此之外,还有很多很多种走法。 \\

暴力枚举的算法利用排列组合的思想,通过多重循环遍历出所有的可能性。但是暴力枚举的时间复杂度是指数级的,有没有更高效的解法呢? \\

要不找个楼梯走一下试试吧!正好能减肥! \\

动态规划是一种分阶段求解决策问题的数学思想,它不止用于编程领域,也应用于管理学、经济学、生物学等。总的来说就是大事化小,小事化了。 \\

在爬楼梯问题中,假设你只差最后一步就走到第10级台阶,这时候会出现几种情况? \\

当然是两种喽,因为每一步只许走1级或2级,所以最后一步要么是从第9级走到第10级,要么是从第8级走到第10级。 \\

接下来就引申出了一个新的问题,如果已知从第0级走到第9级的走法有X种,从第0级走到第8级的走法有Y种,那么从第0级走到第10级的走法就有X + Y种。 \\

