\chapter{模式匹配}

\section{BF}

\subsection{BF(Brute Force)}

在一个字符串中查找另一个字符串的操作称为模式匹配。其中,被查找的字符串被称为文本串,需要查找的子串称为模式串。\\

查找子串的最简单的算法就是采用暴力匹配的方式。暴力匹配的基本思想就是逐个比较相应位置的字符。\\

假设文本串为S,模式串为P:

\begin{enumerate}
	\item 如果当前字符匹配成功(即S[i] == P[j]),则i++, j++,继续匹配下一个字符。

	\item 如果失配(即S[i] != P[j]),令i = i - j + 1,j = 0,相当于每次匹配失败时,i回溯,j被置为0。
\end{enumerate}

例如当文本串S = "BBC ABCDAB ABCDABCDABDE",P = "ABCDABD"。\\

S[0]为B,P[0]为A,不匹配(情形2)。

\begin{table}[H]
	\centering
	\setlength{\tabcolsep}{1.3mm}{
		\begin{tabular}{|c|c|c|c|c|c|c|c|c|c|c|c|c|c|c|c|c|c|c|c|c|c|c|}
			\hline
			\textbf{0}         & \textbf{1} & \textbf{2} & \textbf{3} & \textbf{4} & \textbf{5} & \textbf{6} & \textbf{7} & \textbf{8} & \textbf{9} & \textbf{10} & \textbf{11} & \textbf{12} & \textbf{13} & \textbf{14} & \textbf{15} & \textbf{16} & \textbf{17} & \textbf{18} & \textbf{19} & \textbf{20} & \textbf{21} & \textbf{22} \\
			\hline
			\textcolor{red}{B} & B          & C          &            & A          & B          & C          & D          & A          & B          &             & A           & B           & C           & D           & A           & B           & C           & D           & A           & B           & D           & E           \\
			\hline
			\textcolor{red}{A} & B          & C          & D          & A          & B          & D          &            &            &            &             &             &             &             &             &             &             &             &             &             &             &             &             \\
			\hline
		\end{tabular}
	}
\end{table}

然后再判断S[1]和P[0]是否匹配,相当于模式串向右移动一位。S[1]与P[0]还是不匹配(情形2)。

\begin{table}[H]
	\centering
	\setlength{\tabcolsep}{1.3mm}{
		\begin{tabular}{|c|c|c|c|c|c|c|c|c|c|c|c|c|c|c|c|c|c|c|c|c|c|c|}
			\hline
			\textbf{0} & \textbf{1}         & \textbf{2} & \textbf{3} & \textbf{4} & \textbf{5} & \textbf{6} & \textbf{7} & \textbf{8} & \textbf{9} & \textbf{10} & \textbf{11} & \textbf{12} & \textbf{13} & \textbf{14} & \textbf{15} & \textbf{16} & \textbf{17} & \textbf{18} & \textbf{19} & \textbf{20} & \textbf{21} & \textbf{22} \\
			\hline
			B          & \textcolor{red}{B} & C          &            & A          & B          & C          & D          & A          & B          &             & A           & B           & C           & D           & A           & B           & C           & D           & A           & B           & D           & E           \\
			\hline
			           & \textcolor{red}{A} & B          & C          & D          & A          & B          & D          &            &            &             &             &             &             &             &             &             &             &             &             &             &             &             \\
			\hline
		\end{tabular}
	}
\end{table}

然后再判断S[2]和P[0]是否匹配,模式串再向右移动一位。直到S[4]与P[0]匹配成功(情形1)。

\begin{table}[H]
	\centering
	\setlength{\tabcolsep}{1.3mm}{
		\begin{tabular}{|c|c|c|c|c|c|c|c|c|c|c|c|c|c|c|c|c|c|c|c|c|c|c|}
			\hline
			\textbf{0} & \textbf{1} & \textbf{2} & \textbf{3} & \textbf{4}         & \textbf{5} & \textbf{6} & \textbf{7} & \textbf{8} & \textbf{9} & \textbf{10} & \textbf{11} & \textbf{12} & \textbf{13} & \textbf{14} & \textbf{15} & \textbf{16} & \textbf{17} & \textbf{18} & \textbf{19} & \textbf{20} & \textbf{21} & \textbf{22} \\
			\hline
			B          & B          & C          &            & \textcolor{red}{A} & B          & C          & D          & A          & B          &             & A           & B           & C           & D           & A           & B           & C           & D           & A           & B           & D           & E           \\
			\hline
			           &            &            &            & \textcolor{red}{A} & B          & C          & D          & A          & B          & D           &             &             &             &             &             &             &             &             &             &             &             &             \\
			\hline
		\end{tabular}
	}
\end{table}

此时i = 5,j = 1,接下来判断S[5]与P[1]是否匹配。S[5]与P[1]匹配成功(情形1)。可得i = 6,j = 2。

\begin{table}[H]
	\centering
	\setlength{\tabcolsep}{1.3mm}{
		\begin{tabular}{|c|c|c|c|c|c|c|c|c|c|c|c|c|c|c|c|c|c|c|c|c|c|c|}
			\hline
			\textbf{0} & \textbf{1} & \textbf{2} & \textbf{3} & \textbf{4}         & \textbf{5}         & \textbf{6} & \textbf{7} & \textbf{8} & \textbf{9} & \textbf{10} & \textbf{11} & \textbf{12} & \textbf{13} & \textbf{14} & \textbf{15} & \textbf{16} & \textbf{17} & \textbf{18} & \textbf{19} & \textbf{20} & \textbf{21} & \textbf{22} \\
			\hline
			B          & B          & C          &            & \textcolor{red}{A} & \textcolor{red}{B} & C          & D          & A          & B          &             & A           & B           & C           & D           & A           & B           & C           & D           & A           & B           & D           & E           \\
			\hline
			           &            &            &            & \textcolor{red}{A} & \textcolor{red}{B} & C          & D          & A          & B          & D           &             &             &             &             &             &             &             &             &             &             &             &             \\
			\hline
		\end{tabular}
	}
\end{table}

直到S[10]为空格,P[6]为D,不匹配(情形2)。

\begin{table}[H]
	\centering
	\setlength{\tabcolsep}{1.3mm}{
		\begin{tabular}{|c|c|c|c|c|c|c|c|c|c|c|c|c|c|c|c|c|c|c|c|c|c|c|}
			\hline
			\textbf{0} & \textbf{1} & \textbf{2} & \textbf{3} & \textbf{4}         & \textbf{5}         & \textbf{6}         & \textbf{7}         & \textbf{8}         & \textbf{9}         & \textbf{10}        & \textbf{11} & \textbf{12} & \textbf{13} & \textbf{14} & \textbf{15} & \textbf{16} & \textbf{17} & \textbf{18} & \textbf{19} & \textbf{20} & \textbf{21} & \textbf{22} \\
			\hline
			B          & B          & C          &            & \textcolor{red}{A} & \textcolor{red}{B} & \textcolor{red}{C} & \textcolor{red}{D} & \textcolor{red}{A} & \textcolor{red}{B} &                    & A           & B           & C           & D           & A           & B           & C           & D           & A           & B           & D           & E           \\
			\hline
			           &            &            &            & \textcolor{red}{A} & \textcolor{red}{B} & \textcolor{red}{C} & \textcolor{red}{D} & \textcolor{red}{A} & \textcolor{red}{B} & \textcolor{red}{D} &             &             &             &             &             &             &             &             &             &             &             &             \\
			\hline
		\end{tabular}
	}
\end{table}

此时i = 5,j = 0,相当于判断S[5]跟P[0]是否匹配。

\begin{table}[H]
	\centering
	\setlength{\tabcolsep}{1.3mm}{
		\begin{tabular}{|c|c|c|c|c|c|c|c|c|c|c|c|c|c|c|c|c|c|c|c|c|c|c|}
			\hline
			\textbf{0} & \textbf{1} & \textbf{2} & \textbf{3} & \textbf{4} & \textbf{5}         & \textbf{6} & \textbf{7} & \textbf{8} & \textbf{9} & \textbf{10} & \textbf{11} & \textbf{12} & \textbf{13} & \textbf{14} & \textbf{15} & \textbf{16} & \textbf{17} & \textbf{18} & \textbf{19} & \textbf{20} & \textbf{21} & \textbf{22} \\
			\hline
			B          & B          & C          &            & A          & \textcolor{red}{B} & C          & D          & A          & B          &             & A           & B           & C           & D           & A           & B           & C           & D           & A           & B           & D           & E           \\
			\hline
			           &            &            &            &            & \textcolor{red}{A} & B          & C          & D          & A          & B           & D           &             &             &             &             &             &             &             &             &             &             &             \\
			\hline
		\end{tabular}
	}
\end{table}

至此可以看出,按照暴力匹配算法的思路。同理,直到找到匹配的字符串或文本串遍历结束退出。\\

BF暴力匹配在最坏情况下时间复杂度是$ O(mn) $。\\

\mybox{BF}

\begin{lstlisting}[language=Python]
def brute_force(s, p):
    s_len = len(s)
    p_len = len(p)
    i = 0
    j = 0

    while i < s_len and j < p_len:
        if s[i] == p[j]:
            i += 1
            j += 1
        else:
            i = i - j + 1
            j = 0
    
    if j == p_len:
        return i - j
    else:
        return -1
\end{lstlisting}

\newpage

\section{Sunday}

\subsection{Sunday}

Sunday是从前往后匹配的算法,在匹配失败时重点关注的是文本串中参加匹配的最末位字符的下一位字符。如果该字符没有在模式串中出现则直接跳过,移动位数为模式串长度 + 1,否则移动位数为模式串长度 - 该字符最右出现的下标。\\

Sunday算法巧妙的地方在于它发现匹配失败之后可以直接考察文本串中参加匹配的最末尾字符的下一个字符。\\

例如当文本串S = "bcaitsnaxzfinihao",P = "nihao"。\\

S[0]为b,P[0]为n,匹配失败。关注文本串中参加匹配的最末位字符的下一位字符s,该字符并没在模式串中出现,因此将模式串向右移动模式串长度 + 1,即5 + 1 = 6。

\begin{table}[H]
	\centering
	\setlength{\tabcolsep}{3mm}{
		\begin{tabular}{|c|c|c|c|c|c|c|c|c|c|c|c|c|c|c|c|c|}
			\hline
			\textbf{0}         & \textbf{1} & \textbf{2} & \textbf{3} & \textbf{4} & \textbf{5}          & \textbf{6} & \textbf{7} & \textbf{8} & \textbf{9} & \textbf{10} & \textbf{11} & \textbf{12} & \textbf{13} & \textbf{14} & \textbf{15} & \textbf{16} \\
			\hline
			\textcolor{red}{b} & c          & a          & i          & t          & \textcolor{blue}{s} & n          & a          & x          & z          & f           & i           & n           & i           & h           & a           & o           \\
			\hline
			\textcolor{red}{n} & i          & h          & a          & o          &                     &            &            &            &            &             &             &             &             &             &             &             \\
			\hline
		\end{tabular}
	}
\end{table}

S[7]为a,P[1]为i,匹配失败。关注文本串中参加匹配的最末位字符的下一位字符i,该字符在模式串中最右出现出现下标为1,因此将模式串向右移动模式串长度 - 最右下标,即5 - 1 = 4。

\begin{table}[H]
	\centering
	\setlength{\tabcolsep}{3mm}{
		\begin{tabular}{|c|c|c|c|c|c|c|c|c|c|c|c|c|c|c|c|c|}
			\hline
			\textbf{0} & \textbf{1} & \textbf{2} & \textbf{3} & \textbf{4} & \textbf{5} & \textbf{6}         & \textbf{7}         & \textbf{8} & \textbf{9} & \textbf{10} & \textbf{11}         & \textbf{12} & \textbf{13} & \textbf{14} & \textbf{15} & \textbf{16} \\
			\hline
			b          & c          & a          & i          & t          & s          & \textcolor{red}{n} & \textcolor{red}{a} & x          & z          & f           & \textcolor{blue}{i} & n           & i           & h           & a           & o           \\
			\hline
			           &            &            &            &            &            & \textcolor{red}{n} & \textcolor{red}{i} & h          & a          & o           &                     &             &             &             &             &             \\
			\hline
		\end{tabular}
	}
\end{table}

\begin{table}[H]
	\centering
	\setlength{\tabcolsep}{3mm}{
		\begin{tabular}{|c|c|c|c|c|c|c|c|c|c|c|c|c|c|c|c|c|}
			\hline
			\textbf{0} & \textbf{1} & \textbf{2} & \textbf{3} & \textbf{4} & \textbf{5} & \textbf{6} & \textbf{7} & \textbf{8} & \textbf{9} & \textbf{10} & \textbf{11}         & \textbf{12} & \textbf{13} & \textbf{14} & \textbf{15} & \textbf{16} \\
			\hline
			b          & c          & a          & i          & t          & s          & n          & a          & x          & z          & f           & \textcolor{blue}{i} & n           & i           & h           & a           & o           \\
			\hline
			           &            &            &            &            &            &            &            &            &            & n           & \textcolor{blue}{i} & h           & a           & o           &             &             \\
			\hline
		\end{tabular}
	}
\end{table}

S[10]为f,P[0]为n,匹配失败。关注文本串中参加匹配的最末位字符的下一位字符a,该字符在模式串中最右出现出现下标为3,因此将模式串向右移动模式串长度 - 最右下标,即5 - 3 = 2。

\begin{table}[H]
	\centering
	\setlength{\tabcolsep}{3mm}{
		\begin{tabular}{|c|c|c|c|c|c|c|c|c|c|c|c|c|c|c|c|c|}
			\hline
			\textbf{0} & \textbf{1} & \textbf{2} & \textbf{3} & \textbf{4} & \textbf{5} & \textbf{6} & \textbf{7} & \textbf{8} & \textbf{9} & \textbf{10}        & \textbf{11} & \textbf{12} & \textbf{13} & \textbf{14} & \textbf{15}         & \textbf{16} \\
			\hline
			b          & c          & a          & i          & t          & s          & n          & a          & x          & z          & \textcolor{red}{f} & i           & n           & i           & h           & \textcolor{blue}{a} & o           \\
			\hline
			           &            &            &            &            &            &            &            &            &            & \textcolor{red}{n} & i           & h           & a           & o           &                     &             \\
			\hline
		\end{tabular}
	}
\end{table}

\begin{table}[H]
	\centering
	\setlength{\tabcolsep}{3mm}{
		\begin{tabular}{|c|c|c|c|c|c|c|c|c|c|c|c|c|c|c|c|c|}
			\hline
			\textbf{0} & \textbf{1} & \textbf{2} & \textbf{3} & \textbf{4} & \textbf{5} & \textbf{6} & \textbf{7} & \textbf{8} & \textbf{9} & \textbf{10} & \textbf{11} & \textbf{12} & \textbf{13} & \textbf{14} & \textbf{15}         & \textbf{16} \\
			\hline
			b          & c          & a          & i          & t          & s          & n          & a          & x          & z          & f           & i           & n           & i           & h           & \textcolor{blue}{a} & o           \\
			\hline
			           &            &            &            &            &            &            &            &            &            &             &             & n           & i           & h           & \textcolor{blue}{a} & o           \\
			\hline
		\end{tabular}
	}
\end{table}

此时,模式串匹配成功。\\

\mybox{Sunday}

\begin{lstlisting}[language=Python]
def sunday(s, p):
    s_len = len(s)
    p_len = len(p)
    i = 0
    j = 0
    result = 0
    
    while i < s_len and j < p_len:
        if s[i] == p[j]:
            i += 1
            j += 1
            continue

        idx = result + p_len
        if idx >= s_len:
            return -1

        k = p_len - 1
        while k >= 0 and s[idx] != p[k]:
            k -= 1

        i = result
        i += p_len - k
        result = i
        j = 0

        if result + p_len > s_len:
            return -1

    return result
\end{lstlisting}

\newpage

\section{RK}

\subsection{RK(Rabin-Karp)}

RK算法的命名由Rabin和Karp两位发明者的名字而来,它的实现方式有点与众不同。BF算法只是简单粗暴地对两个字符串的所有字符依次比较,而RK算法比较的是两个字符串的哈希值。\\

RK算法的基本思想就是将模式串P的哈希值跟文本串S中每一个长度为|S|的子串的哈希值进行比较。如果两个字符串哈希值不相同,则它们肯定不匹配。如果它们的哈希值相同,它们有可能匹配(因为可能存在哈希冲突)。\\

由于哈希函数有可能会产生哈希冲突,哈希值相等的两个字符串不一定相同。因此如果两个字符串的哈希值相等,就把这两个字符串本身进行一次比较即可。这种方法的前提是要控制冲突概率,达到可以接受的状态。\\

不过每次hash的时间复杂度为$ O(n) $,如果把全部子串都进行hash,总的时间复杂度不是和BF算法一样,都是$ O(mn) $了吗?\\

其实对子串的hash计算并不是独立的,从第二个子串开始,每一次子串的hash都可以由上一次子串进行简单的加减得到(减去第一个字符的hash,加上新字符的hash)。因此通过设计特殊的哈希算法,只需扫描一遍文本串就能计算出所有子串的哈希值。期间最多需要比较m - n + 1个子串的哈希值。\\

相比于BF算法,RK算法采用哈希值比较的方式,免去了许多无谓的字符比较,所以时间复杂度大大提高了。RK算法的缺点在于哈希冲突,每一次哈希冲突的时候,RK算法都要对子串和模式串进行逐个字符的比较。如果冲突太多了,RK算法就退化成了BF算法。\\

\mybox{RK}

\begin{lstlisting}[language=Java]
public static int rk(String s, String p) {
    int sLen = s.length();
    int pLen = p.length();

    int sHash = hash(s.substring(0, pLen)); // 文本串子串哈希值
    int pHash = hash(p);                    // 模式串哈希值

    for(int i = 0; i < sLen - pLen + 1; i++) {
        if(sHash == pHash) {
            if(match(s.substring(i, i + pLen), p)) {
                return i;
            }
        }
        if(i < sLen - pLen) {
            sHash = nextHash(s, sHash, i, pLen);
        }
    }

    return -1;
}

public static int hash(String s) {
    int hashCode = 0;
    for(int i = 0; i < s.length(); i++) {
        hashCode += s.charAt(i) - 'a';
    }
    return hashCode;
}

public static int nextHash(String s, int hash, int start, int n) {
    hash -= s.charAt(start) - 'a';
    hash += s.charAt(start + n) - 'a';
    return hash;
}

public static boolean match(String s, String p) {
    return s.equals(p);
}
\end{lstlisting}

\newpage

\section{BM}

\subsection{BM(Boyer-Moore)}

BM算法的名字取自于它的两位发明者,计算机科学家Bob Boyer和J Strother Moore。为了能减少比较,BM算法制定了两条规则:

\begin{itemize}
	\item 坏字符规则(bad character)
	\item 好后缀规则(good suffix)
\end{itemize}

\vspace{0.5cm}

\subsection{坏字符规则}

坏字符是指文本串与模式串不匹配的字符。

\begin{table}[H]
	\centering
	\setlength{\tabcolsep}{1.5mm}{
		\begin{tabular}{|c|c|c|c|c|c|c|c|c|c|c|c|c|c|c|c|c|c|c|c|c|}
			\hline
			\textbf{0} & \textbf{1} & \textbf{2} & \textbf{3} & \textbf{4} & \textbf{5} & \textbf{6} & \textbf{7} & \textbf{8}         & \textbf{9} & \textbf{10} & \textbf{11} & \textbf{12} & \textbf{13} & \textbf{14} & \textbf{15} & \textbf{16} & \textbf{17} & \textbf{18} & \textbf{19} & \textbf{20} \\
			\hline
			G          & T          & T          & A          & T          & A          & G          & C          & \textcolor{red}{T} & G          & G           & T           & A           & G           & C           & G           & G           & C           & G           & A           & A           \\
			\hline
			G          & T          & A          & G          & C          & G          & G          & C          & \textcolor{red}{G} &            &             &             &             &             &             &             &             &             &             &             &             \\
			\hline
		\end{tabular}
	}
\end{table}

咦?为什么坏字符不是主串中下标为2的字符T呢?那个位置不是先被检测的到吗?\\

BM算法的检测顺序是从字符串的最右侧向最左侧进行的。当检测到第一个坏字符后,并没有必要让模式串一位一位向后挪动并比较。因为只有模式串与坏字符对齐的位置相同的情况下,两者才有匹配的可能。由于模式串的第1位字符也是T,这样就可以直接把模式串中的T与文本串的坏字符对齐,进行下一轮比较。

\begin{table}[H]
	\centering
	\setlength{\tabcolsep}{1.5mm}{
		\begin{tabular}{|c|c|c|c|c|c|c|c|c|c|c|c|c|c|c|c|c|c|c|c|c|}
			\hline
			\textbf{0} & \textbf{1} & \textbf{2} & \textbf{3} & \textbf{4} & \textbf{5} & \textbf{6} & \textbf{7} & \textbf{8}          & \textbf{9} & \textbf{10} & \textbf{11} & \textbf{12} & \textbf{13} & \textbf{14} & \textbf{15} & \textbf{16} & \textbf{17} & \textbf{18} & \textbf{19} & \textbf{20} \\
			\hline
			G          & T          & T          & A          & T          & A          & G          & C          & \textcolor{blue}{T} & G          & G           & T           & A           & G           & C           & G           & G           & C           & G           & A           & A           \\
			\hline
			           &            &            &            &            &            &            & G          & \textcolor{blue}{T} & A          & G           & C           & G           & G           & C           & G           &             &             &             &             &             \\
			\hline
		\end{tabular}
	}
\end{table}

坏字符的位置越靠右,下一轮模式串的挪动跨度就可能越长,节省的比较次数也就越多。这就是BM算法从右向左检测的好处。\\

接着,从右向左成功匹配GCG,并遇到坏字符A。

\begin{table}[H]
	\centering
	\setlength{\tabcolsep}{1.5mm}{
		\begin{tabular}{|c|c|c|c|c|c|c|c|c|c|c|c|c|c|c|c|c|c|c|c|c|}
			\hline
			\textbf{0} & \textbf{1} & \textbf{2} & \textbf{3} & \textbf{4} & \textbf{5} & \textbf{6} & \textbf{7} & \textbf{8} & \textbf{9}          & \textbf{10} & \textbf{11} & \textbf{12}         & \textbf{13}        & \textbf{14}        & \textbf{15}        & \textbf{16} & \textbf{17} & \textbf{18} & \textbf{19} & \textbf{20} \\
			\hline
			G          & T          & T          & A          & T          & A          & G          & C          & T          & G                   & G           & T           & \textcolor{blue}{A} & \textcolor{red}{G} & \textcolor{red}{C} & \textcolor{red}{G} & G           & C           & G           & A           & A           \\
			\hline
			           &            &            &            &            &            &            & G          & T          & \textcolor{blue}{A} & G           & C           & \textcolor{red}{G}  & \textcolor{red}{G} & \textcolor{red}{C} & \textcolor{red}{G} &             &             &             &             &             \\
			\hline
		\end{tabular}
	}
\end{table}

按照类似的方式,找到模式串的第2位字符A,将它与文本串的坏字符对齐。

\begin{table}[H]
	\centering
	\setlength{\tabcolsep}{1.5mm}{
		\begin{tabular}{|c|c|c|c|c|c|c|c|c|c|c|c|c|c|c|c|c|c|c|c|c|}
			\hline
			\textbf{0} & \textbf{1} & \textbf{2} & \textbf{3} & \textbf{4} & \textbf{5} & \textbf{6} & \textbf{7} & \textbf{8} & \textbf{9} & \textbf{10}        & \textbf{11}        & \textbf{12}        & \textbf{13}        & \textbf{14}        & \textbf{15}        & \textbf{16}        & \textbf{17}        & \textbf{18}        & \textbf{19} & \textbf{20} \\
			\hline
			G          & T          & T          & A          & T          & A          & G          & C          & T          & G          & \textcolor{red}{G} & \textcolor{red}{T} & \textcolor{red}{A} & \textcolor{red}{G} & \textcolor{red}{C} & \textcolor{red}{G} & \textcolor{red}{G} & \textcolor{red}{C} & \textcolor{red}{G} & A           & A           \\
			\hline
			           &            &            &            &            &            &            &            &            &            & \textcolor{red}{G} & \textcolor{red}{T} & \textcolor{red}{A} & \textcolor{red}{G} & \textcolor{red}{C} & \textcolor{red}{G} & \textcolor{red}{G} & \textcolor{red}{C} & \textcolor{red}{G} &             &             \\
			\hline
		\end{tabular}
	}
\end{table}

如果出现坏字符在模式串中不存在的情况,就直接把模式串挪到主串坏字符的下一位。

\begin{table}[H]
	\centering
	\setlength{\tabcolsep}{1.5mm}{
		\begin{tabular}{|c|c|c|c|c|c|c|c|c|c|c|c|c|c|c|c|c|c|c|c|c|}
			\hline
			\textbf{0} & \textbf{1} & \textbf{2} & \textbf{3} & \textbf{4} & \textbf{5} & \textbf{6} & \textbf{7} & \textbf{8}         & \textbf{9} & \textbf{10} & \textbf{11} & \textbf{12} & \textbf{13} & \textbf{14} & \textbf{15} & \textbf{16} & \textbf{17} & \textbf{18} & \textbf{19} & \textbf{20} \\
			\hline
			G          & T          & T          & A          & T          & A          & G          & C          & \textcolor{red}{T} & G          & G           & T           & A           & G           & C           & G           & G           & C           & G           & A           & A           \\
			\hline
			G          & C          & A          & I          & C          & G          & G          & C          & \textcolor{red}{G} &            &             &             &             &             &             &             &             &             &             &             &             \\
			\hline
		\end{tabular}
	}
\end{table}

\begin{table}[H]
	\centering
	\setlength{\tabcolsep}{1.5mm}{
		\begin{tabular}{|c|c|c|c|c|c|c|c|c|c|c|c|c|c|c|c|c|c|c|c|c|}
			\hline
			\textbf{0} & \textbf{1} & \textbf{2} & \textbf{3} & \textbf{4} & \textbf{5} & \textbf{6} & \textbf{7} & \textbf{8}         & \textbf{9} & \textbf{10} & \textbf{11} & \textbf{12} & \textbf{13} & \textbf{14} & \textbf{15} & \textbf{16} & \textbf{17} & \textbf{18} & \textbf{19} & \textbf{20} \\
			\hline
			G          & T          & T          & A          & T          & A          & G          & C          & \textcolor{red}{T} & G          & G           & T           & A           & G           & C           & G           & G           & C           & G           & A           & A           \\
			\hline
			           &            &            &            &            &            &            &            &                    & G          & C           & A           & I           & C           & G           & G           & C           & G           &             &             &             \\
			\hline
		\end{tabular}
	}
\end{table}

\vspace{0.5cm}

\subsection{好后缀规则}

好后缀是指文本串与模式串当中相匹配的后缀。\\

例如对于这个例子,如果继续使用坏字符规则,模式串只能向后挪动一位。

\begin{table}[H]
	\centering
	\setlength{\tabcolsep}{2.5mm}{
		\begin{tabular}{|c|c|c|c|c|c|c|c|c|c|c|c|c|c|c|c|c|c|c|c|c|}
			\hline
			\textbf{0} & \textbf{1} & \textbf{2}          & \textbf{3}          & \textbf{4}         & \textbf{5}         & \textbf{6}         & \textbf{7} & \textbf{8} & \textbf{9} & \textbf{10} & \textbf{11} & \textbf{12} & \textbf{13} \\
			\hline
			C          & T          & G                   & \textcolor{blue}{G} & \textcolor{red}{G} & \textcolor{red}{C} & \textcolor{red}{G} & A          & G          & C          & G           & G           & A           & A           \\
			\hline
			G          & C          & \textcolor{blue}{G} & \textcolor{red}{A}  & \textcolor{red}{G} & \textcolor{red}{C} & \textcolor{red}{G} &            &            &            &             &             &             &             \\
			\hline
		\end{tabular}
	}
\end{table}

为了能真正减少比较次数,就需要使用好后缀规则。在第一轮比较中,文本串和模式串都有共同的后缀GCG,这就是所谓的好后缀。如果模式串的其它位置也包含与GCG相同的子串,那么就可以挪动模式串,让这个子串与好后缀对齐。

\begin{table}[H]
	\centering
	\setlength{\tabcolsep}{2.5mm}{
		\begin{tabular}{|c|c|c|c|c|c|c|c|c|c|c|c|c|c|c|c|c|c|c|c|c|}
			\hline
			\textbf{0}          & \textbf{1}          & \textbf{2}          & \textbf{3} & \textbf{4}          & \textbf{5}          & \textbf{6}          & \textbf{7} & \textbf{8} & \textbf{9} & \textbf{10} & \textbf{11} & \textbf{12} & \textbf{13} \\
			\hline
			C                   & T                   & G                   & G          & \textcolor{blue}{G} & \textcolor{blue}{C} & \textcolor{blue}{G} & A          & G          & C          & G           & G           & A           & A           \\
			\hline
			\textcolor{blue}{G} & \textcolor{blue}{C} & \textcolor{blue}{G} & A          & \textcolor{red}{G}  & \textcolor{red}{C}  & \textcolor{red}{G}  &            &            &            &             &             &             &             \\
			\hline
		\end{tabular}
	}
\end{table}

\begin{table}[H]
	\centering
	\setlength{\tabcolsep}{2.5mm}{
		\begin{tabular}{|c|c|c|c|c|c|c|c|c|c|c|c|c|c|c|c|c|c|c|c|c|}
			\hline
			\textbf{0} & \textbf{1} & \textbf{2} & \textbf{3} & \textbf{4}          & \textbf{5}          & \textbf{6}          & \textbf{7} & \textbf{8} & \textbf{9} & \textbf{10} & \textbf{11} & \textbf{12} & \textbf{13} \\
			\hline
			C          & T          & G          & G          & \textcolor{blue}{G} & \textcolor{blue}{C} & \textcolor{blue}{G} & A          & G          & C          & G           & G           & A           & A           \\
			\hline
			           &            &            &            & \textcolor{blue}{G} & \textcolor{blue}{C} & \textcolor{blue}{G} & A          & G          & C          & G           &             &             &             \\
			\hline
		\end{tabular}
	}
\end{table}

如果模式串中不存在其它与好后缀相同的片段,是不是可以直接把模式串挪到好后缀之后?\\

使不得!这里还要判断一种特殊情况,模式串的前缀是否和好后缀的后缀相匹配,免得挪过头了。

\begin{table}[H]
	\centering
	\setlength{\tabcolsep}{2.5mm}{
		\begin{tabular}{|c|c|c|c|c|c|c|c|c|c|c|c|c|}
			\hline
			\textbf{0}          & \textbf{1}          & \textbf{2} & \textbf{3} & \textbf{4}          & \textbf{5}          & \textbf{6} & \textbf{7} & \textbf{8} & \textbf{9} & \textbf{10} & \textbf{11} & \textbf{12} \\
			\hline
			T                   & G                   & G          & G          & \textcolor{blue}{C} & \textcolor{blue}{G} & A          & G          & C          & G          & G           & A           & A           \\
			\hline
			\textcolor{blue}{C} & \textcolor{blue}{G} & A          & G          & C                   & G                   &            &            &            &            &             &             &             \\
			\hline
		\end{tabular}
	}
\end{table}

\begin{table}[H]
	\centering
	\setlength{\tabcolsep}{2.5mm}{
		\begin{tabular}{|c|c|c|c|c|c|c|c|c|c|c|c|c|}
			\hline
			\textbf{0} & \textbf{1} & \textbf{2} & \textbf{3} & \textbf{4}          & \textbf{5}          & \textbf{6} & \textbf{7} & \textbf{8} & \textbf{9} & \textbf{10} & \textbf{11} & \textbf{12} \\
			\hline
			T          & G          & G          & G          & \textcolor{blue}{C} & \textcolor{blue}{G} & A          & G          & C          & G          & G           & A           & A           \\
			\hline
			           &            &            &            & \textcolor{blue}{C} & \textcolor{blue}{G} & A          & G          & C          & G          &             &             &             \\
			\hline
		\end{tabular}
	}
\end{table}

那应该什么时候用坏字符规则,什么时候用好后缀规则呢?\\

可以在每一轮字符比较之后,按照坏字符和好后缀规则分别计算相应的挪动距离,哪一种距离更长,就把模式串挪动对应的长度。比如坏字符可以让模式串在下一轮挪动3位,好后缀可以让模式串移动5位,那么就应该采用好后缀规则。

\newpage

\section{KMP}

\subsection{KMP(Knuth-Morris-Pratt)}

KMP算法是一个里程碑似的算法,它的出现宣告了人类找到了线性时间复杂度的字符串匹配算法。在此之后才出现了其它线性时间的字符串匹配算法,比如BM算法和Sunday算法。\\

KMP算法由三位计算机科学家D. E. Knuth、J. H. Morris和V. R. Pratt提出,KMP这个算法名字正是取自这三个人的姓氏首字母。\\

与BM算法类似,KMP算法也在试图减少无谓的字符比较,但KMP算法把专注点放在了已匹配的前缀。\\

在每一次匹配过程中,其实可以判断出后续几次匹配是否会成功。算法的核心就是每次匹配过程中推断出后续完全不可能匹配成功的部分,从而减少比较的次数。\\

例如主串中已匹配末尾的GTG是最长可匹配后缀,模式串中开头的GTG是最长可匹配前缀。

\begin{table}[H]
	\centering
	\setlength{\tabcolsep}{1.5mm}{
		\begin{tabular}{|c|c|c|c|c|c|c|c|c|c|c|c|c|c|c|c|c|c|c|c|c|}
			\hline
			\textbf{0}          & \textbf{1}          & \textbf{2}          & \textbf{3}          & \textbf{4}          & \textbf{5}         & \textbf{6} & \textbf{7} & \textbf{8} & \textbf{9} & \textbf{10} & \textbf{11} & \textbf{12} & \textbf{13} & \textbf{14} & \textbf{15} & \textbf{16} & \textbf{17} & \textbf{18} & \textbf{19} & \textbf{20} \\
			\hline
			G                   & T                   & \textcolor{blue}{G} & \textcolor{blue}{T} & \textcolor{blue}{G} & \textcolor{red}{A} & G          & C          & T          & G          & G           & T           & G           & T           & G           & T           & G           & C           & F           & A           & A           \\
			\hline
			\textcolor{blue}{G} & \textcolor{blue}{T} & \textcolor{blue}{G} & T                   & G                   & \textcolor{red}{C} & F          &            &            &            &             &             &             &             &             &             &             &             &             &             &             \\
			\hline
		\end{tabular}
	}
\end{table}

将最长可匹配后缀与最长可匹配前缀对齐。

\begin{table}[H]
	\centering
	\setlength{\tabcolsep}{1.5mm}{
		\begin{tabular}{|c|c|c|c|c|c|c|c|c|c|c|c|c|c|c|c|c|c|c|c|c|}
			\hline
			\textbf{0} & \textbf{1} & \textbf{2}          & \textbf{3}          & \textbf{4}          & \textbf{5}         & \textbf{6} & \textbf{7} & \textbf{8} & \textbf{9} & \textbf{10} & \textbf{11} & \textbf{12} & \textbf{13} & \textbf{14} & \textbf{15} & \textbf{16} & \textbf{17} & \textbf{18} & \textbf{19} & \textbf{20} \\
			\hline
			G          & T          & \textcolor{blue}{G} & \textcolor{blue}{T} & \textcolor{blue}{G} & \textcolor{red}{A} & G          & C          & T          & G          & G           & T           & G           & T           & G           & T           & G           & C           & F           & A           & A           \\
			\hline
			           &            & \textcolor{blue}{G} & \textcolor{blue}{T} & \textcolor{blue}{G} & \textcolor{red}{T} & G          & C          & F          &            &             &             &             &             &             &             &             &             &             &             &             \\
			\hline
		\end{tabular}
	}
\end{table}

主串中已匹配末尾的G是最长可匹配后缀,模式串中开头的G是最长可匹配前缀。

\begin{table}[H]
	\centering
	\setlength{\tabcolsep}{1.5mm}{
		\begin{tabular}{|c|c|c|c|c|c|c|c|c|c|c|c|c|c|c|c|c|c|c|c|c|}
			\hline
			\textbf{0} & \textbf{1} & \textbf{2}          & \textbf{3} & \textbf{4}          & \textbf{5} & \textbf{6} & \textbf{7} & \textbf{8} & \textbf{9} & \textbf{10} & \textbf{11} & \textbf{12} & \textbf{13} & \textbf{14} & \textbf{15} & \textbf{16} & \textbf{17} & \textbf{18} & \textbf{19} & \textbf{20} \\
			\hline
			G          & T          & G                   & T          & \textcolor{blue}{G} & A          & G          & C          & T          & G          & G           & T           & G           & T           & G           & T           & G           & C           & F           & A           & A           \\
			\hline
			           &            & \textcolor{blue}{G} & T          & G                   & T          & G          & C          & F          &            &             &             &             &             &             &             &             &             &             &             &             \\
			\hline
		\end{tabular}
	}
\end{table}

将最长可匹配后缀与最长可匹配前缀对齐。

\begin{table}[H]
	\centering
	\setlength{\tabcolsep}{1.5mm}{
		\begin{tabular}{|c|c|c|c|c|c|c|c|c|c|c|c|c|c|c|c|c|c|c|c|c|}
			\hline
			\textbf{0} & \textbf{1} & \textbf{2} & \textbf{3} & \textbf{4}          & \textbf{5} & \textbf{6} & \textbf{7} & \textbf{8} & \textbf{9} & \textbf{10} & \textbf{11} & \textbf{12} & \textbf{13} & \textbf{14} & \textbf{15} & \textbf{16} & \textbf{17} & \textbf{18} & \textbf{19} & \textbf{20} \\
			\hline
			G          & T          & G          & T          & \textcolor{blue}{G} & A          & G          & C          & T          & G          & G           & T           & G           & T           & G           & T           & G           & C           & F           & A           & A           \\
			\hline
			           &            &            &            & \textcolor{blue}{G} & T          & G          & T          & G          & C          & F           &             &             &             &             &             &             &             &             &             &             \\
			\hline
		\end{tabular}
	}
\end{table}

KMP算法的整体思路就是在已匹配的前缀当中寻找最长可匹配后缀子串和最长可匹配前缀子串,在下一轮直接把两者对齐,从而实现模式串的快速移动。\\

那么如何找到一个字符串前缀的最长可匹配后缀子串和最长可匹配前缀子串呢?难道在每一轮都要重新遍历吗?\\

要找到这两个子串没有必要每次都去遍历,可以事先缓存到一个集合当中,用的时候再去集合里面取。这个集合被称为next数组,如何生成next数组是KMP算法的最大难点。\\

\subsection{next数组}

next数组实质上就是找出模式串前后字符重复出现的个数,next[i]表示模式串T[0]到T[i]这个子串使得前k个字符等于后k个字符的最大值,其中k不能取i + 1,因为子串一共才i + 1个字符,自己跟自己相等毫无意义。\\

例如当文本串S = "ABABAABAABAC",P = "ABAABAC"。

\begin{table}[H]
	\centering
	\setlength{\tabcolsep}{5mm}{
		\begin{tabular}{|c|c|c|c|c|c|c|c|}
			\hline
			\textbf{模式串}   & \textbf{A} & \textbf{B} & \textbf{A} & \textbf{A} & \textbf{B} & \textbf{A} & \textbf{C} \\
			\hline
			\textbf{next数组} & 0          & 0          & 1          & 1          & 2          & 3          & 0          \\
			\hline
		\end{tabular}
	}
	\caption{模式串ABAABAC的next数组}
\end{table}

\begin{table}[H]
	\centering
	\setlength{\tabcolsep}{3mm}{
		\begin{tabular}{|c|c|c|c|c|c|c|c|c|c|c|c|}
			\hline
			\textbf{0}          & \textbf{1}          & \textbf{2}          & \textbf{3}         & \textbf{4} & \textbf{5} & \textbf{6} & \textbf{7} & \textbf{8} & \textbf{9} & \textbf{10} & \textbf{11} \\
			\hline
			\textcolor{blue}{A} & \textcolor{blue}{B} & \textcolor{blue}{A} & \textcolor{red}{B} & A          & A          & B          & A          & A          & B          & A           & C           \\
			\hline
			\textcolor{blue}{A} & \textcolor{blue}{B} & \textcolor{blue}{A} & \textcolor{red}{A} & B          & A          & C          &            &            &            &             &             \\
			\hline
		\end{tabular}
	}
\end{table}

已经成功匹配3个,查看next[3-1] = 1,表示最长匹配前后缀长度为1。模式串需要跳过的长度为匹配上的长度减去最长匹配前后缀长度,即3 - 1 = 2。

\begin{table}[H]
	\centering
	\setlength{\tabcolsep}{3mm}{
		\begin{tabular}{|c|c|c|c|c|c|c|c|c|c|c|c|}
			\hline
			\textbf{0} & \textbf{1} & \textbf{2}          & \textbf{3}          & \textbf{4}          & \textbf{5}          & \textbf{6}          & \textbf{7}          & \textbf{8}         & \textbf{9} & \textbf{10} & \textbf{11} \\
			\hline
			A          & B          & \textcolor{blue}{A} & \textcolor{blue}{B} & \textcolor{blue}{A} & \textcolor{blue}{A} & \textcolor{blue}{B} & \textcolor{blue}{A} & \textcolor{red}{A} & B          & A           & C           \\
			\hline
			           &            & \textcolor{blue}{A} & \textcolor{blue}{B} & \textcolor{blue}{A} & \textcolor{blue}{A} & \textcolor{blue}{B} & \textcolor{blue}{A} & \textcolor{red}{C} &            &             &             \\
			\hline
		\end{tabular}
	}
\end{table}

已经成功匹配6个,查看next[6-1] = 3,表示最长匹配前后缀长度为3。模式串需要跳过的长度为匹配上的长度减去最长匹配前后缀长度,即6 - 3 = 3。

\begin{table}[H]
	\centering
	\setlength{\tabcolsep}{3mm}{
		\begin{tabular}{|c|c|c|c|c|c|c|c|c|c|c|c|}
			\hline
			\textbf{0} & \textbf{1} & \textbf{2} & \textbf{3} & \textbf{4} & \textbf{5}          & \textbf{6}          & \textbf{7}          & \textbf{8}          & \textbf{9}          & \textbf{10}         & \textbf{11}         \\
			\hline
			A          & B          & A          & B          & A          & \textcolor{blue}{A} & \textcolor{blue}{B} & \textcolor{blue}{A} & \textcolor{blue}{A} & \textcolor{blue}{B} & \textcolor{blue}{A} & \textcolor{blue}{C} \\
			\hline
			           &            &            &            &            & \textcolor{blue}{A} & \textcolor{blue}{B} & \textcolor{blue}{A} & \textcolor{blue}{A} & \textcolor{blue}{B} & \textcolor{blue}{A} & \textcolor{blue}{C} \\
			\hline
		\end{tabular}
	}
\end{table}

在跳过不可能匹配的趟数后,并非再从头开始匹配,而是从之前不匹配的位置开始。\\

\subsection{算法分析}

假设模式串的长度为m,主串的长度为n。KMP算法唯一的额外空间是next数组,那么空间复杂度就是$ O(m) $。\\

至于时间复杂度,KMP算法包括两步。第一步生成next数组,时间复杂度为$ O(m) $;第二步是对主串的遍历,时间复杂度为$ O(n) $。因此KMP算法的整体时间复杂度就是$ O(m + n) $。\\

\mybox{KMP}

\begin{lstlisting}[language=Java]
public static int[] getNexts(String p) {
	int n = p.length();
	int[] next = new int[n];
	int j = 0;
	for(int i = 2; i < n; i++) {
		while(j != 0 && p.charAt(j) != p.charAt(i-1)) {
			j = next[j];
		}
		if(p.charAt(j) == p.charAt(i-1)) {
			j++;
		}
		next[i] = j;
	}
	return next;
}

public static int kmp(String s, String p) {
	int[] next = getNexts(p);
	int j = 0;
	for(int i = 0; i < s.length(); i++) {
		while(j > 0 && s.charAt(i) != p.charAt(j)) {
			j = next[j];
		}
		if(s.charAt(i) == p.charAt(j)) {
			j++;
		}
		if(j == p.length()) {
			return i - p.length() + 1;
		}
	}
	return -1;
}
\end{lstlisting}

\newpage