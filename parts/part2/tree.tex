\section{红黑树}

\subsection{红黑树(Red Black Tree)}

红黑树是一种自平衡的二叉查找树,除了符合二叉查找树的基本特性外,它还具有如下附加特性:

\begin{enumerate}
	\item 结点是红色或黑色的。
	\item 根结点是黑色的。
	\item 叶子结点都是黑色的空结点NIL。
	\item 红色结点的两个子结点都是黑色的,即从叶子到根的所有路径上不能有连续的两个红色结点。
	\item 从任一结点到其每个叶子的所有路径都包含相同数目的黑色结点。
\end{enumerate}

\begin{figure}[H]
	\centering
	\begin{tikzpicture}[font=\sffamily, very thick,
			level distance=1.5cm,
			level 1/.style={sibling distance=8cm},
			level 2/.style={sibling distance=4cm},
			level 3/.style={sibling distance=2cm},
			level 4/.style={sibling distance=1cm},
		]
		\node [blackVertex] (r){13}
		child {
				node [redVertex] {8}
				child {
						node [blackVertex] {1}
						child {node [nil] {NIL}}
						child {
								node [redVertex] {6}
								child {node [nil] {NIL}}
								child {node [nil] {NIL}}
							}
					}
				child {
						node [blackVertex] {11}
						child {node [nil] {NIL}}
						child {node [nil] {NIL}}
					}
			}
		child {
				node [redVertex] {17}
				child {
						node [blackVertex] {15}
						child {node [nil] {NIL}}
						child {node [nil] {NIL}}
					}
				child {
						node [blackVertex] {25}
						child {
								node [redVertex] {22}
								child {node [nil] {NIL}}
								child {node [nil] {NIL}}
							}
						child {
								node [redVertex] {27}
								child {node [nil] {NIL}}
								child {node [nil] {NIL}}
							}
					}
			};
	\end{tikzpicture}
	\caption{红黑树}
\end{figure}

天呐,这条条框框的太多了吧! \\

正是因为这些规则限制,才保证了红黑树的自平衡,红黑树从根到叶子的最长路径不会超过最短路径的2倍。 \\

红黑树的应用有很多,其中JDK的集合类TreeMap和TreeSet底层就是红黑树实现的。在Java8中,连HashMap也用到了红黑树。

\subsection{失衡调整}

当插入或删除结点时,红黑树的规则可能被破坏,需要调整使其重新符合规则。 \\

例如向红黑树中插入新结点14,由于父结点15是黑色结点,这种情况不会破坏红黑树的规则,无需做任何调整。

\begin{figure}[H]
	\centering
	\begin{tikzpicture}[font=\sffamily, very thick,
			level distance=1.5cm,
			level 1/.style={sibling distance=8cm},
			level 2/.style={sibling distance=4cm},
			level 3/.style={sibling distance=2cm},
			level 4/.style={sibling distance=1cm},
		]
		\node [blackVertex] (r){13}
		child {
				node [redVertex] {8}
				child {
						node [blackVertex] {1}
						child {node [nil] {NIL}}
						child {
								node [redVertex] {6}
								child {node [nil] {NIL}}
								child {node [nil] {NIL}}
							}
					}
				child {
						node [blackVertex] {11}
						child {node [nil] {NIL}}
						child {node [nil] {NIL}}
					}
			}
		child {
				node [redVertex] {17}
				child {
						node [blackVertex] {15}
						child {
								node [redVertex] {14}
								child {node [nil] {NIL}}
								child {node [nil] {NIL}}
							}
						child {node [nil] {NIL}}
					}
				child {
						node [blackVertex] {25}
						child {
								node [redVertex] {22}
								child {node [nil] {NIL}}
								child {node [nil] {NIL}}
							}
						child {
								node [redVertex] {27}
								child {node [nil] {NIL}}
								child {node [nil] {NIL}}
							}
					}
			};
	\end{tikzpicture}
	\caption{插入14}
\end{figure}

向红黑树中插入新结点21,由于父结点22是红色结点,违反了红黑树的规则4(红色结点的两个子结点都是黑色的)。

\begin{figure}[H]
	\centering
	\begin{tikzpicture}[font=\sffamily, very thick,
			level distance=1.5cm,
			level 1/.style={sibling distance=8cm},
			level 2/.style={sibling distance=4cm},
			level 3/.style={sibling distance=2cm},
			level 4/.style={sibling distance=1cm},
		]
		\node [blackVertex] (r){13}
		child {
				node [redVertex] {8}
				child {
						node [blackVertex] {1}
						child {node [nil] {NIL}}
						child {
								node [redVertex] {6}
								child {node [nil] {NIL}}
								child {node [nil] {NIL}}
							}
					}
				child {
						node [blackVertex] {11}
						child {node [nil] {NIL}}
						child {node [nil] {NIL}}
					}
			}
		child {
				node [redVertex] {17}
				child {
						node [blackVertex] {15}
						child {node [nil] {NIL}}
						child {node [nil] {NIL}}
					}
				child {
						node [blackVertex] {25}
						child {
								node [redVertex] {22}
								child {
										node [redVertex] {21}
										child {node [nil] {NIL}}
										child {node [nil] {NIL}}
									}
								child {node [nil] {NIL}}
							}
						child {
								node [redVertex] {27}
								child {node [nil] {NIL}}
								child {node [nil] {NIL}}
							}
					}
			};
	\end{tikzpicture}
	\caption{插入21}
\end{figure}

调整的方法有变色和旋转两种,而旋转又包含左旋转和右旋转两种方式。 \\

为了重新符合红黑树的规则,有时需要把红色结点变为黑色,或是把黑色结点变为红色。 \\

例如对于红黑树的一部分(子树),新插入的结点Y是红色结点,它的父结点X也是红色结点,不符合规则4(红色结点的两个子结点都是黑色的),因此可以把结点X变为黑色。

\begin{figure}[H]
	\centering
	\begin{tikzpicture}[font=\sffamily, very thick,
			level distance=1.5cm,
			level 1/.style={sibling distance=4cm},
			level 2/.style={sibling distance=2cm},
			level 3/.style={sibling distance=1cm}
		]
		\node [redVertex] (r){X}
		child {
				node [redVertex] {Y}
				child {node [nil] {NIL}}
				child {node [nil] {NIL}}
			}
		child {node [nil] {NIL}};
	\end{tikzpicture}
	\caption{违反规则4}
\end{figure}

\begin{figure}[H]
	\centering
	\begin{tikzpicture}[font=\sffamily, very thick,
			level distance=1.5cm,
			level 1/.style={sibling distance=4cm},
			level 2/.style={sibling distance=2cm},
			level 3/.style={sibling distance=1cm}
		]
		\node [blackVertex] (r){X}
		child {
				node [redVertex] {Y}
				child {node [nil] {NIL}}
				child {node [nil] {NIL}}
			}
		child {node [nil] {NIL}};
	\end{tikzpicture}
	\caption{变色}
\end{figure}

但是,如果这是简单的把一个结点变色,会导致相关路径凭空多出一个黑色结点,这样就会打破规则5(从任一结点到其每个叶子的所有路径都包含相同数目的黑色结点),因此还需要其它的调整策略。

\subsection{红黑树插入结点}

红黑树插入新结点时,可以分为五种不同的局面。每一种局面有不同的调整方法。

\subsubsection{局面1}

新结点(A)位于树根,没有父结点。

\begin{figure}[H]
	\centering
	\begin{tikzpicture}[font=\sffamily, very thick,
			level distance=1.5cm,
			level 1/.style={sibling distance=2cm},
			level 2/.style={sibling distance=1cm}
		]
		\node [redVertex] (r){A}
		child {node[rectangle,draw] {1}}
		child {node[rectangle,draw] {2}};
	\end{tikzpicture}
	\caption{局面1}
\end{figure}

这种局面,直接让新结点变色为黑色,规则2(根结点是黑色的)满足。同时黑色的根结点使每条路径上的黑色结点数目都增加了1,因此并没有打破规则5(从任一结点到其每个叶子的所有路径都包含相同数目的黑色结点)。

\begin{figure}[H]
	\centering
	\begin{tikzpicture}[font=\sffamily, very thick,
			level distance=1.5cm,
			level 1/.style={sibling distance=2cm},
			level 2/.style={sibling distance=1cm}
		]
		\node [blackVertex] (r){A}
		child {node[rectangle,draw] {1}}
		child {node[rectangle,draw] {2}};
	\end{tikzpicture}
\end{figure}

\subsubsection{局面2}

新结点(B)的父结点是黑色的。新插入的红色结点B并没有打破规则,无需调整。

\begin{figure}[H]
	\centering
	\begin{tikzpicture}[font=\sffamily, very thick,
			level distance=1.5cm,
			level 1/.style={sibling distance=2cm},
			level 2/.style={sibling distance=1cm}
		]
		\node [blackVertex] (r){A}
		child {
				node [redVertex] {B}
				child {node[rectangle,draw] {1}}
				child {node[rectangle,draw] {2}}
			}
		child {node[rectangle,draw] {3}};
	\end{tikzpicture}
	\caption{局面2}
\end{figure}

\subsubsection{局面3}

新结点(D)的父结点和叔叔结点都是红色。

\begin{figure}[H]
	\centering
	\begin{tikzpicture}[font=\sffamily, very thick,
			level distance=1.5cm,
			level 1/.style={sibling distance=2cm},
			level 2/.style={sibling distance=1cm},
			level 3/.style={sibling distance=1cm}
		]
		\node [blackVertex] (r){A}
		child {
				node [redVertex] {B}
				child {
						node [redVertex] {D}
						child {node[rectangle,draw] {1}}
						child {node[rectangle,draw] {2}}
					}
				child {node[rectangle,draw] {3}}
			}
		child {
				node [redVertex] {C}
				child {node[rectangle,draw] {4}}
				child {node[rectangle,draw] {5}}
			};
	\end{tikzpicture}
	\caption{局面3}
\end{figure}

这种局面,两个红色结点B和D连续,违反了规则4(红色结点的两个子结点都是黑色的),因此需要先让结点B变为黑色。

\begin{figure}[H]
	\centering
	\begin{tikzpicture}[font=\sffamily, very thick,
			level distance=1.5cm,
			level 1/.style={sibling distance=2cm},
			level 2/.style={sibling distance=1cm},
			level 3/.style={sibling distance=1cm}
		]
		\node [blackVertex] (r){A}
		child {
				node [blackVertex] {B}
				child {
						node [redVertex] {D}
						child {node[rectangle,draw] {1}}
						child {node[rectangle,draw] {2}}
					}
				child {node[rectangle,draw] {3}}
			}
		child {
				node [redVertex] {C}
				child {node[rectangle,draw] {4}}
				child {node[rectangle,draw] {5}}
			};
	\end{tikzpicture}
\end{figure}

但是这样一来,结点B所在路径凭空多出了一个黑色结点,打破了规则5(从任一结点到其每个叶子的所有路径都包含相同数目的黑色结点),因此再让结点A变为红色。

\begin{figure}[H]
	\centering
	\begin{tikzpicture}[font=\sffamily, very thick,
			level distance=1.5cm,
			level 1/.style={sibling distance=2cm},
			level 2/.style={sibling distance=1cm},
			level 3/.style={sibling distance=1cm}
		]
		\node [redVertex] (r){A}
		child {
				node [blackVertex] {B}
				child {
						node [redVertex] {D}
						child {node[rectangle,draw] {1}}
						child {node[rectangle,draw] {2}}
					}
				child {node[rectangle,draw] {3}}
			}
		child {
				node [redVertex] {C}
				child {node[rectangle,draw] {4}}
				child {node[rectangle,draw] {5}}
			};
	\end{tikzpicture}
\end{figure}

这时结点A和C又成为了连续的红色结点,再将结点C变为黑色。

\begin{figure}[H]
	\centering
	\begin{tikzpicture}[font=\sffamily, very thick,
			level distance=1.5cm,
			level 1/.style={sibling distance=2cm},
			level 2/.style={sibling distance=1cm},
			level 3/.style={sibling distance=1cm}
		]
		\node [redVertex] (r){A}
		child {
				node [blackVertex] {B}
				child {
						node [redVertex] {D}
						child {node[rectangle,draw] {1}}
						child {node[rectangle,draw] {2}}
					}
				child {node[rectangle,draw] {3}}
			}
		child {
				node [blackVertex] {C}
				child {node[rectangle,draw] {4}}
				child {node[rectangle,draw] {5}}
			};
	\end{tikzpicture}
\end{figure}

如果红色结点A是根结点,那么违反了规则2(根结点是黑色),参考局面1的方法,将其变为黑色。

\begin{figure}[H]
	\centering
	\begin{tikzpicture}[font=\sffamily, very thick,
			level distance=1.5cm,
			level 1/.style={sibling distance=2cm},
			level 2/.style={sibling distance=1cm},
			level 3/.style={sibling distance=1cm}
		]
		\node [blackVertex] (r){A}
		child {
				node [blackVertex] {B}
				child {
						node [redVertex] {D}
						child {node[rectangle,draw] {1}}
						child {node[rectangle,draw] {2}}
					}
				child {node[rectangle,draw] {3}}
			}
		child {
				node [blackVertex] {C}
				child {node[rectangle,draw] {4}}
				child {node[rectangle,draw] {5}}
			};
	\end{tikzpicture}
\end{figure}

\subsubsection{局面4}

新结点(D)的父结点是红色,叔叔结点是黑色或者没有叔叔,且新结点是父结点的右孩子,父结点是祖父结点的左孩子。

\begin{figure}[H]
	\centering
	\begin{tikzpicture}[font=\sffamily, very thick,
			level distance=1.5cm,
			level 1/.style={sibling distance=2cm},
			level 2/.style={sibling distance=1cm},
			level 3/.style={sibling distance=1cm}
		]
		\node [blackVertex] (r){A}
		child {
				node [blackVertex] {B}
				child {
						node [redVertex] {D}
						child {node[rectangle,draw] {1}}
						child {node[rectangle,draw] {2}}
					}
				child {node[rectangle,draw] {3}}
			}
		child {
				node [blackVertex] {C}
				child {node[rectangle,draw] {4}}
				child {node[rectangle,draw] {5}}
			};
	\end{tikzpicture}
    \caption{局面4}
\end{figure}